\section{Preliminaries}

We fix a field~$\kk$ of characteristic zero and an integer~$N\geq1$, and let $A$
be the algebra freely generated by two letters~$x$ and~$y$ subject to the
sole relation
  \[
  yx-xy = x^N.
  \]
If $N=0$, then this is a Weyl algebra, and if $N=1$ what we get is
isomorphic to the universal enveloping algebra of the non-abelian Lie
algebra of dimension~$2$. In general, we can view this algebra as the skew
polynomial algebra $\kk[x][y;\theta]$, with $\theta:\kk[x]\to\kk[x]$ the
derivation such that $\theta(x)=x^N$, and it follows from this that $A$ is
a noetherian domain, and that $\B=\{x^iy^j:i,j\geq0\}$ is a basis for~$A$.
There is a grading on~$A$ that makes~$x$ and~$y$ homogeneous of degrees~$1$
and~$N-1$, respectively. The homogeneous components of~$A$ are spanned by
the monomials in the basis~$\B$ that they contain, and they are all finite
dimensional exactly when $N\geq2$.

As $yx=xy+x^N=x(y+x^{N-1})$, it is easy to check that $Ax=xA$, that is,
that the element~$x$ is normal ---~in Lemma~\ref{lemma:normal} below we
will see that in fact it generates, together with the non-zero scalars, the
set of all the non-zero normal elements as a monoid. The set $S\coloneqq
\{x^i:i\geq0\}$ is then a left and right denominator set in~$A$ and we can
consider the localization $A_x\coloneqq S^{-1}A$ at~$S$. If $\kk[x^{\pm1}]$
is the algebra of Laurent polynomials and
$\tilde\theta:\kk[x^{\pm1}]\to\kk[x^{\pm1}]$ is the extension of the
derivation~$\theta$ to~$\kk[x^{\pm1}]$, then it is immediate that $A_x$ can
be viewed as the skew polynomial algebra~$\kk[x^{\pm1}][y;\tilde\theta]$.
The algebra~$A_x$ is graded, with $x^{-1}$ of degree~$-1$.

There is, on the other hand, an algebra filtration~$(F_k)_{k\geq-1}$ on~$A$
that has, for each $k\geq-1$, its $k$th layer $F_k$ spanned by the set of monomials
$\{x^iy^j:i\geq0,j\leq k\}$, and whose associated graded algebra $\gr A$ is
freely generated as a commutative algebra by the principal
symbols~$\overline x\coloneqq x+F_{-1}\in F_0/F_1$ and $\overline
y\coloneqq y+F_{0}\in F_1/F_0$  of~$x$ and~$y$, and these have degree~$0$
and~$1$, respectively. In particular ---~and we will use this fact all the
time without mentioning it~--- for all $i$,~$j\in\NN_0$ we have that
  \[
  y^ix^j \equiv x^iy^j \mod F_{i-1}.
  \]
As it is well-known, the non-commutativity of the algebra~$A$ gives rise to
a Poisson algebra structure on $\gr A$: this is the
biderivation~$\{\place,\place\}:\gr A\times\gr A\to\gr A$ uniquely
determined by the condition that
  \[
  \{\overline y,\overline x\} = \overline x^N.
  \]

Computing in~$A$ usually involves a significant amount of reordering
products, and the following lemma gives two important special cases of
this that are moreover related by a pleasing symmetry. If $k$ and~$x$ are
elements of a commutative ring~$\Lambda$ and $i\in\NN_0$, we define,
following Rafael Díaz and Eddy Pariguan \cite{DP}, the \newterm{Pochhammer
$k$-symbol} to be
  \[
  (x)_{k,i} \coloneqq x(x+k)(x+2k)\cdots(x+(i-1)k).
  \]
We will use a few times the equality
  \[ \label{eq:kpoch}
  \sum_{j\geq0}(a)_{k,j}\frac{t^j}{j!} = (1-kt)^{-a/k},
  \]
valid in the algebra~$\Lambda[[t]]$, provided, so that it actually makes
sense, that $\Lambda$ contains~$\QQ$. It can be proved by noticing that the
two sides are in the kernel of the differential operator
$(1-kt)\frac{\d}{\d t}-a$ and have the same constant term.

\begin{Lemma}\label{lemma:comm}
For each $i$,~$j\geq0$ we have that
  \begin{gather}
  y^jx^i = \sum_{t=0}^{j} (i)_{N-1,j-t} \binom{j}{t} x^{i+(j-t)(N-1)}y^t
     \label{eq:comm:1}
\shortintertext{and}
  x^iy^j = \sum_{t=0}^j(-i)_{N-1,j-t}\binom{j}{t}x^{(j-t)(N-1)}y^tx^i.
  \end{gather}
\end{Lemma}

These two equalities tell us how to move $x^i$ from one side of~$y^j$ to
the other.

\begin{proof}
Let us start by proving an identity involving Pochhammer $k$-symbols that
is a version of the well-known Zhu--Vandermonde identity. We
fix~$k\in\kk$ and for each $a$ in~$\kk$ consider, as above, the formal
series $f_a\coloneqq\sum_{i\geq0}(a)_{k,i}t^i/i!\in\kk[[t]]$. We have that
$f_af_b=f_{a+b}$ for all choices of~$a$ and~$b$ in~$\kk$: this can be checked
by showing that both sides of the equality are annihilated by the operator
$(1-kt)\frac{\d}{\d t}-(a+b)$ and have the same constant term. Writing this
equality in terms of coefficients we see that for all $j\geq0$ we have
that
  \[
  \sum_{i=0}^j \frac{(a)_{k,j-i}}{(j-i)!} \frac{(b)_{i,k}}{i!}
        = \frac{(a+b)_{k,j}}{j!}.
  \]
Let us now prove the lemma. An obvious induction proves that
  \[ \label{eq:comm:3}
  y^jx = \sum_{t=0}^{j} (1)_{N-1,j-t} \binom{j}{t}
                             x^{1+(j-t)(N-1)}y^t
  \]
for all $j\geq0$, and this is the special case of the
identity~\eqref{eq:comm:1} of the lemma in which $i=1$. Let us now
fix~$i\geq0$ and assume inductively that the identity~\eqref{eq:comm:1}
holds. Then
  \begin{align}
  y^jx^{i+1} 
    &= \sum_{t=0}^{j} (i)_{N-1,j-t} \binom{j}{t} x^{i+(j-t)(N-1)}y^tx 
\intertext{and using~\eqref{eq:comm:3} we see this is}
    &= \sum_{t=0}^{j} 
       \sum_{s=0}^{t} 
          (i)_{N-1,j-t} \binom{j}{t} 
          (1)_{N-1,t-s} \binom{t}{s}
          x^{i+1+(j-s)(N-1)}y^s
       \\
    &= \sum_{s=0}^{j} 
       \frac{j!}{s!}
       \left(
       \sum_{t=0}^{j-s} 
           \frac{(i)_{N-1,j-s-t}}{(j-s-t)!}
           \frac{(1)_{N-1,t}}{t!}
       \right)
          x^{i+1+(j-s)(N-1)}y^s
       \\
    &= 
       \sum_{s=0}^{j} 
          (i+1)_{N-1,j-s}
          \binom{j}{s}
          x^{i+1+(j-s)(N-1)}y^s.
  \end{align}
This proves~\eqref{eq:comm:1} for all values of~$j$. To prove the remaining
identity, we invert the one we already have: we fix $i$ and~$j$ in~$\NN_0$ and
compute that
  \begin{align}
  \MoveEqLeft[4]
  \sum_{t=0}^j(-i)_{N-1,j-t}\binom{j}{t}x^{(j-t)(N-1)}y^tx^i \\
    &= \sum_{t=0}^j\sum_{s=0}^{t}
                (-i)_{N-1,j-t} (i)_{N-1,t-s}
                \binom{j}{t} \binom{t}{s}
                x^{i+(j-s)(N-1)}
                y^s
        \\
    &= \sum_{s=0}^{j}
                \frac{j!}{s!}
                \left(
                \sum_{t=0}^{j-s}
                \frac{(-i)_{N-1,j-t-s}}{(j-t-s)!}
                \frac{(i)_{N-1,t}}{t!}
                \right)
                x^{i+(j-s)(N-1)}
                y^s
     = x^i y^j,
  \end{align}
which is what we want.
\end{proof}

\begin{Remark}\label{rem:laguerre}
When $N=1$ the identity~\eqref{eq:comm:1} tells us that
$y^jx^i=x^i(y+i)^j$, and we can write it, passing to the
localization~$A_x$, in the form
  \[ \label{eq:xyx:N1}
  x^{-i}y^jx^i = (y+i)^j.
  \]
Let us suppose now that $N\neq1$ and show how to view the
identity~\eqref{eq:comm:1} in a similar way also in this case. In the
localization~$A_x$ we have that
  \[
  x^{N-1}(x^{-N+1}y - y x^{-N+1})x^{N-1}
        = yx^{N-1} - x^{N-1}y
        = (N-1)x^{2N-2},
  \]
so that
  \[
  \left[ \frac{x^{-N+1}}{N-1}, y \right] = 1.
  \]
The subalgebra of~$A_x$ generated by $z\coloneqq x^{-N+1}/(N-1)$ and $y$ is
thus isomorphic to the first Weyl algebra. In this subalgebra we can
consider normal-order products ---~well-known in quantum field theory, for
example~--- and powers: we consider the symbol $\normalorder{zy}$ with the
property that $(\normalorder{zy})^i=z^iy^i$ for all $i\geq0$.

For each choice of $\alpha\in\CC$ and $j\in\NN_0$ we let
$L^{(\alpha)}_j\in\CC[t]$ be the $j$th generalized Laguerre polynomial with
parameter~$\alpha$. This is the unique solution of the Laguerre
differential equation
  \[
  tu'' + (\alpha+1-t)u' + ju = 0
  \]
that is a polynomial with leading coefficient $(-1)^j/j!$. It has
degree~$j$ and can be found to be given by the formula
  \[ \label{eq:laguerre}
  L^{(\alpha)}_j(t) = \sum_{i=0}^j(-1)^i\binom{j+\alpha}{j-i}
        \frac{t^i}{i!}.
  \]
When $\alpha>-1$, the sequence $(L^{(\alpha)}_j)_{j\geq0}$ is obtained from
$(t^j)_{j\geq1}$ by Gram--Schmidt orthogonalization over the
interval~$[0,+\infty)$ with respect to the weighting function~$t^\alpha
e^{-t}$ associated to the gamma distribution. We refer
to~\cite{AAR}*{\textsection 6.3} for more information on these
polynomials.

With this setup, we claim now that for all $i$,~$j\in\NN_0$ have that
  \[ \label{eq:xyx:NN}
  x^{-i}(\normalorder{zy})^j x^i
  = (-1)^j j! \, L^{(-i/(N-1)-j)}_j(\normalorder{zy}).
  \]
We view this as a reasonable generalization of~\eqref{eq:xyx:N1} for $N$
greater than~$1$ --- in any case, the fact that the commutation relations
of the algebra can be written in terms of hyper\-geometric functions is
interesting! Notice that the polynomial $(-1)^jj!L_j^{(-i/(N-1)-j)}$ is
monic, so the coefficient $(-1)^jj!$ appears here just as a
consequence of the standard normalization of Laguerre polynomials. We can
prove the equality~\eqref{eq:xyx:NN} by evaluating its right hand side using
the explicit formula~\eqref{eq:laguerre} for the Laguerre polynomials and
then using the first identity of Lemma~\ref{lemma:comm}. 
\end{Remark}

Another very useful observation that we will use many times is the
following one.

\begin{Lemma}\label{lemma:centralizer:x}
The centralizer of~$x$ in~$A$ is~$\kk[x]$.
\end{Lemma}

\begin{proof}
If $u$ is an element of~$A\setminus\kk[x]$ that commutes with~$x$, then
there exist $l\geq1$ and $a_0$,~\dots,~$a_l\in\kk[x]$ such that
$u=\sum_{i=0}^la_iy^i$ and $a_l\neq0$, and we have that
  \[
  0 = [u,x] = \sum_{i=0}^la_l[y^i,x]
        \equiv la_lx^Ny^{l-1} \mod F_{l-2},
  \]
which is absurd. As every element of~$\kk[x]$ obviously commutes with~$x$,
this proves the lemma.
\end{proof}

Going a bit further, we can describe the center of~$A$:

\begin{Proposition}\label{prop:center}
The center of~$A$ is~$\kk$.
\end{Proposition}

\begin{proof}
If $u$ is a central element of~$A$ then in particular $u$ commutes with~$x$
and we know from the lemma that this implies that $u\in\kk[x]$. As it also
commutes with~$y$, we also have that $0=[y,x]=x^Nu'$, so that $u$ is in
fact in~$\kk$.
\end{proof}

We will need further on the following related result, which has an entirely
similar proof.

\begin{Proposition}\label{prop:center:x}
The center of the localization~$A_x$ is~$\kk$.
\end{Proposition}

\begin{proof}
Let $z$ be a central element of~$A_x$. There is a positive integer
$l\in\NN$ such that $x^lz$ is in~$A$, and, as this product commutes
with~$x$, we know that it belongs to~$\kk[x]$. We thus see that $z$ is
in~$\kk[x^{\pm1}]$, and therefore that $0=[y,z]=x^nz'$: it is in fact a
scalar.
\end{proof}
