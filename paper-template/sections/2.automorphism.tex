\section{The automorphism group}

After the preliminaries of the previous section, our first objective is to
compute the group of automorphisms of the algebra~$A$, and to do that we
will start by finding the \newterm{normal} elements of~$A$, that is, those
elements~$u$ of~$A$ such that $uA=Au$.

\begin{Lemma}\label{lemma:normal}
The set of non-zero normal elements in~$A$ is $\N\coloneqq\{\lambda
x^i:\lambda\in\kk^\times,i\geq0\}$.
\end{Lemma}

\begin{proof}
The element~$x$ is normal in~$A$, since $yx=x(y+x^{N-1})$, and then it is
clear that all the elements of the set~$\N$ are normal in~$A$. Conversely,
let $u$ be a non-zero normal element in~$A$, and let $l\geq0$ and
$a_0$,~\dots,~$a_l\in\kk[x]$ be such that $u=\sum_{i=0}^la_iy^i$ and
$a_l\neq0$. As $u$ is normal, the right ideal~$uA$ is a bilateral ideal and
therefore $[u,x]\in uA$: there exists a $b\in A$ such that $[u,x]=ub$. As
$[u,x]=\sum_{i=0}^la_i[y^i,x]\in F_{l-1}$, this is only possible if in fact
$[u,x]=0$, so that $u\in\kk[x]$. We also have that $x^Nu'=[y,u]\in uA$ and,
since of course $x^Nu'$ is also in~$\kk[x]$, we have that $u$
divides~$x^Nu'$: this implies that $u$ is $\lambda x^i$ for
some~$\lambda\in\kk$ and some integer~$i\geq0$, so that $u$ belongs to the
set~$\N$.
\end{proof}

In view of this lemma, the element~$x$ generates the monoid of normal
elements of~$A$ up to non-zero scalars. Since $A$ is a domain, each
non-zero normal element~$u$ in~$A$ determines uniquely an automorphism of
algebras $\nu_u:A\to A$ such that $au=u\nu_u(a)$ for all $a\in A$. Let us
record for future use the description of the automorphism associated
to~$x$, which we will write~$\sigma_1$ for reasons that will be clear later.

\begin{Lemma}\label{lemma:sigma-1}
The automorphism~$\sigma_1:A\to A$ such that $ax=x\sigma_1(a)$ for all
$a\in A$ has $\sigma_1(x)=x$ and $\sigma_1(y)=y+x^{N-1}$. \qed
\end{Lemma}

This humble origin of~$\sigma_1$ as the automorphism associated to the
normal element~$x$ can actually be painted in a somewhat more impressive
way:

\begin{Proposition}\label{prop:CY}
The algebra~$A_N$ is twisted Calabi--Yau algebra of dimension~$2$, and its
modular automorphism is precisely the automorphism $\sigma_1:A_N\to A_N$ of
Lemma~\ref{lemma:sigma-1}.
\end{Proposition}

What we call here modular automorphism of a twisted Calabi--Yau algebra is
often called the Nakayama automorphism of the algebra.

\begin{proof}
That $A_N$ is twisted Calabi--Yau of dimension~$2$ is a consequence of the
fact that it is an Ore extension of~$\kk[x]$, which is a Calabi--Yau
algebra of dimension~$1$: this follows from a theorem of L.-Y.~Liu, S.~Wang
and Q.-S.~Wu proved in \cite{LWW}. That the modular automorphism of~$A_N$
is~$\sigma_1$ is also a consequence of their result.
\end{proof}

There is a right action of the multiplicative group~$\kk^\times$
on~$\kk[x]$ by algebra automorphisms such that $x\cdot\lambda = \lambda x$
for all $\lambda\in\kk^\times$. We denote $\kk[x]\bowtie\kk^\times$ the
group whose underlying set is the cartesian
product~$\kk[x]\times\kk^\times$ and whose product is such that
  \[
  (f,\lambda)\cdot(g,\mu) = (\mu^{N-1}f+g\cdot\lambda,\lambda\mu)
  \]
for all $f$,~$g\in\kk[x]$ and all~$\lambda$,~$\mu\in\kk^\times$. This
group\footnote{In general, if $G$ and $H$ are two groups, and $\lact$ and
$\triangleleft$ are a left action and a right action of $H$ on~$G$ by group
automorphisms such that $h\lact(g\triangleleft h')=(h\lact g)\triangleleft
h'$, then we can construct a group $G\bowtie H$ with underlying set
$G\times H$ and multiplication such that
$(g,h)\cdot(g',h')=((g\triangleleft h')(h\lact g'),hh')$. One can see that
this is isomorphic to a direct product of~$G$ and~$H$ with respect to an
action of $G$ on~$H$ constructed from~$\lact$ and~$\triangleleft$, but
surprisingly we have not found this direct construction in the literature.}
shows up in the following result.

\begin{Proposition}\label{prop:aut}
For each choice of $f\in\kk[x]$ and $\lambda\in\kk^\times$ there is unique
automorphism $\phi_{f,\lambda}:A\to A$ such that 
  \[
  \phi_{f,\lambda}(x) = \lambda x,
  \qquad
  \phi_{f,\lambda}(y) = \lambda^{N-1}y + f.
  \]
The function
$\Phi:(f,\lambda)\in\kk[x]\bowtie\kk^\times\mapsto\phi_{f,\lambda}\in\Aut(A)$
is an isomorphism of groups.
\end{Proposition}

Recall that we have the standing hypothesis that $N\geq1$: when $N=0$, the
algebra~$A$ is the first Weyl algebra and its automorphism group, which was
computed by Jacques Dixmier \cite{Dixmier:3} and Leonid Makar-Limanov
\cite{ML}, is both significantly larger and much less abelian. The case in
which $N=1$ was treated by Martha Smith in~\cite{Smith}, and much later the
case of a general Ore extension~$A_h$ by Jeffrey Bergen in~\cite{Bergen}. 
Our small class of extensions allows for a less complicated argument,
though.

\begin{proof}
An easy calculation shows that for each choice of $f\in\kk[x]$ and
$\lambda\in\kk^\times$ there is indeed an algebra endomorphism
$\phi_{f,\lambda}:A\to A$ mapping~$x$ and~$y$ to~$\lambda x$
and~$\lambda^{N-1}y+f$, respectively: we thus have a function
$(f,\lambda)\in\kk[x]\bowtie\kk^\times
\mapsto\phi_{f,\lambda}\in\End_{\Alg}(A)$. This is the map~$\Phi$ described
in the statement of the proposition. A direct calculation shows this map is
a morphism of monoids, so it actually takes values in~$\Aut(A)$. It is
obvious that it is injective, and we will now show that it is also
surjective.

Let $\phi:A\to A$ be an automorphism of algebras and, as before, let us
write $\N$ for the set of non-zero normal elements of~$A$, which is a
monoid with respect to multiplication. Of course, we have
that~$\phi(\N)=\N$ and that the restriction~$\phi|_\N:\N\to\N$ is an
automorphism of monoids which is the identity on the subset~$\kk^\times$
of~$\N$: it follows immediately from this that there is a
scalar~$\lambda\in\kk^\times$ such that $\phi(x)=\lambda x$. Moreover,
since $0=\phi([y,x]-x^N)=[\phi(y),\lambda x]-\lambda^Nx^N$ and
$[\lambda^{N-1}y,\lambda x]=\lambda^Nx^N$, we see that
$[\phi(y)-\lambda^{N-1}y,x]=0$ and, therefore, that
$\phi(y)-\lambda^{N-1}y\in\kk[x]$. This tells us that $\phi$ is in the
image of the map~$\Phi$, so that this mar is surjective.
\end{proof}

With the very explicit description of the group $\Aut(A)$ that this
proposition gives we can immediately compute its center:

\begin{Corollary}\label{coro:center}
For each $t\in\kk^\times$ there is a unique automorphism $\sigma_t:A\to A$
such that $\sigma_t(x)=x$ and $\sigma_t(y)=y+tx^{N-1}$, and it is central
in~$\Aut(A)$. The function
  \[
  t\in\kk\mapsto\sigma_t\in\Aut(A)
  \]
is an injective morphism of groups whose image is precisely the center
of~$\Aut(A)$. \qed
\end{Corollary}

The center of~$\Aut(A)$ is therefore a $1$-parameter subgroup which goes
through the automorphism~$\sigma_1$ of Lemma~\ref{lemma:sigma-1}. We will
find later an infinitesimal generator for this $1$-parameter subgroup as a
class in the first Hochschild cohomology of the algebra. 

\bigskip

Let $\Lambda$ be an algebra. If $u\in \Lambda$, then, as usual, the \newterm{inner
derivation} corresponding to~$u$ is   
  \[
  \ad(u):a\in \Lambda\mapsto[u,a]\in \Lambda,
  \]
and we say that $u$ is \newterm{locally $\ad$-nilpotent} if the
derivation~$\ad(u)$ is locally nilpotent. When that is the case we can
consider the exponential of~$\ad(u)$, namely the automorphism
  \[
  \exp\ad(u) : a\in\Lambda
               \mapsto 
               \sum_{i\geq0}\frac{\ad(u)^i(a)}{i!}\in \Lambda,
  \]
because for each $a$ in~$\Lambda$ the series appearing here is in fact a
finite sum. We put
  \[
  \Exp(\Lambda) \coloneqq \{ \exp\ad(u):\text{$u\in \Lambda$ is 
                        locally $\ad$-nilpotent} \}.
  \]
This is a conjugation-invariant subset of~$\Aut(\Lambda)$ but, in general,
not a subgroup. We refer to G.~Freudenburg's book \cite{Freudenburg} for
general information about locally nilpotent derivations, albeit in a
commutative setting.

\begin{Proposition}\label{prop:ad-nilpotent}
The set of $\ad$-locally nilpotent elements in~$A$ is $\kk[x]$ and for each
$f\in\kk[x]$ we have that
  \[
  \exp\ad(f)=\phi_{x^Nf',1}.
  \]
The set~$\Exp(A)$ coincides with $\{\phi_{f,1}:f\in x^N\kk[x]\}$ and is
a normal subgroup of~$\Aut(A)$.
\end{Proposition}

The locally $\ad$-nilpotent elements of the first Weyl algebra were
described by Dixmier in~\cite{Dixmier:3}*{Théorème 9.1}: when viewing the
algebra as that of differential operators on the line, they are the
elements that belong to the orbits of constant coefficient differential
operators under the action of the automorphism group of the algebra. Here
one could use the same description, but it is much less interesting: the
automorphism group of a Weyl algebra is much larger.

\begin{proof}
For each $a\in\kk[x]\setminus0$ let us write
$\nu(a)\coloneqq\max\{t\in\NN_0:a\in x^t\kk[x]\}$. On the other hand, if
$u$ is an element of~$A$ that is not in~$\kk[x]$, then there exist $l\geq1$
and $a_0$,~\dots,~$a_l\in\kk[x]$ such that $u=\sum_{i=0}^la_iy^i$ and
$a_l\neq0$, and we will call the rational number $\slope(u)\coloneqq
\nu(a_l)/l$ the \newterm{slope} of~$u$. We will start by showing that
  
\begin{claim}\label{eq:sl}
  If $u$ and $v$ are elements of $A\setminus\kk[x]$ and
  $\slope(v)>\slope(u)$, then $[u,v]\in A\setminus\kk[x]$ and
  $\slope([u,v])>\slope(u)$.
\end{claim}
To do so, let $u$,~$v\in A\setminus\kk[x]$, so that there are
$l$,~$m\geq1$, $a_0$,~\dots,~$a_l$, $b_0$,~\dots,~$b_m\in\kk[x]$ such that
$u=\sum_{i=0}^la_iy^i$, $v=\sum_{j=0}^mb_jy^j$, $a_l\neq0$ and $b_m\neq0$,
and let us suppose that $\slope(v)>\slope(u)$, so that
  \[\label{eq:sl:1}
  \nu(b_m)/m>\nu(a_l)/l.
  \]
We have that
  \begin{align}
  [u,v] 
       &= \sum_{i=0}^l\sum_{j=0}^m
          \Bigl(
          a_i[y^i,b_j]y^j
          +
          b_j[a_i,y^j]y^i
          \Bigr)
       \equiv
          a_l[y^l,b_m]y^m
          +
          b_m[a_l,y^m]y^l
          \\
     &\equiv x^N(la_lb_m' - mb_ma_l') y^{l+m-1} \mod F_{l+m-2}.
        \label{eq:sl:2}
  \end{align}
There are $c$,~$d\in\kk[x]$ not divisible by~$x$ such that
$a_l=x^{\nu(a_l)}c$ and $b_m=x^{\nu(b_m)}d$, and 
  \[ \label{eq:sl:3}
  la_lb_m' - mb_ma_l' 
  = x^{\nu(a_l)+\nu(b_m)-1}
    \Bigl(
    \bigl(l\nu(b_m)-m\nu(a_l)\bigr)cd
    +
    x\bigl(lcd'-mdc'\bigr)
    \Bigr).
  \]
If the left hand side of this equality is~$0$, then ---~since $x$ does not
divide the product~$cd$~--- we have that $l\nu(b_m)-m\nu(a_l)=0$: this is
absurd, since we are assuming that the inequality~\eqref{eq:sl:1} holds.
Going back to~\eqref{eq:sl:2}, we see that $[u,v]\in F_{l+m-1}\setminus
F_{l+m-2}$ and, in particular, that $[u,v]\in A\setminus\kk[x]$, since
$l+m-1\geq1$. Moreover, in view of~\eqref{eq:sl:2} and~\eqref{eq:sl:3} and
the fact that $x$ does not divide the product~$cd$, we have
  \begin{align}
  \slope([u,v]) 
        &= \frac{\nu(a_l)+\nu(b_m)+N-1}{l+m-1}
\intertext{and this is easily seen to be}
        &> \frac{\nu(a_l)}{l} = \slope(u)
  \end{align}
using~\eqref{eq:sl:1} and the fact that $N\geq1$. This proves the
claim~\eqref{eq:sl} above.

Suppose now that $u$ is an element of~$A\setminus\kk[x]$, and let $m$ be an
integer such that $m>\slope(u)$. If we put $v_i\coloneqq\ad(u)^i(x^my)$ for
each non-negative integer~$i$, then an obvious induction
using~\eqref{eq:sl}, starting with the observation that $v_0\in
A\setminus\kk[x]$ and $\slope(v_0)=m>\slope(u)$, shows that $v_i\neq0$ for
all $i\in\NN_0$. This proves that the element~$u$ is not locally
$\ad$-nilpotent and, therefore, that the set of locally $\ad$-nilpotent
element of~$A$ is contained in~$\kk[x]$.

Conversely, if $f$ is an element of~$\kk[x]$ then we have that
$\ad(f)(F_i)\subseteq F_{i-1}$ for all $i\in\NN_0$, so that
$\ad(f)^{i+1}(F^i)=0$ for all $i\in\NN_0$: this shows that~$f$ is locally
$\ad$-nilpotent and, with that, the first claim of the proposition.
Moreover, a direct calculation show that $(\exp\ad(f))(x)=x$ and
$(\exp\ad(f))(y)=y+x^Nf'$, so that $\exp\ad(f)$ is the
automorphism~$\phi_{x^Nf',1}$. That $\Exp(A)$ is $\{\phi_{f,1}:f\in
x^N\kk[x]\}$ is now clear, that it is a subgroup of~$\Aut(A)$ follows
another simple calculation, and its normality is obvious.
\end{proof}

The subgroup of~$\Aut(A)$ generated by the exponentials of the locally
$\ad$-nilpotent elements of~$A$, which we have just shown to be
precisely~$\Exp(A)$, is cognate to the group of inner automorphisms of a
Lie algebra ---~as given by Roger Carter in~\cite{Carter}*{\textsection 3.2}, for
example~--- so it makes sense to view the quotient $\Aut(A)/\Exp(A)$ as a
Lie-theoretic version of the \newterm{outer automorphism group} of~$A$. We
will describe this quotient below. For comparison, the \emph{usual} inner
automorphism group of~$A$, in the sense of associative algebras, is
trivial, as the units of~$A$ are central, so that the usual outer
automorphism group of~$A$ is just $\Aut(A)$. On the other hand, the X-inner
automorphism group of~$A$ ---~that is, the group of automorphisms of~$A$
that are restrictions of inner automorphisms of the Martindale left
quotient ring of~$A$, considered originally by Vladislav K.\,Har\v{c}enko
in~\citelist{\cite{Harchenko:1}\cite{Harchenko:2}}~--- is isomorphic
to~$\ZZ$: this was computed by Jeffrey Bergen in~\cite{Bergen}*{Theorem~2.6}. 

\begin{Corollary}\label{coro:aut-quot}
Let $\xi$ be the class of $x$ in the quotient $Q\coloneqq\kk[x]/(x^N)$, let
$\kk^\times$ act on the right on this quotient by algebra automorphisms in
such a way that $\xi\cdot\lambda = \lambda\xi$ for all
$\lambda\in\kk^\times$, and let $Q\bowtie\kk^\times$ be the group that as a
set coincides with $Q\times\kk^\times$ and whose product is such that
  \[
  (p,\lambda)\cdot(q,\mu) = (\mu^{N-1}p+q\cdot\lambda,\lambda\mu)
  \]
whenever $(p,\lambda)$ and $(q,\mu)$ are two elements
of~$Q\times\kk^\times$. The function
  \[
  Q\bowtie\kk^\times \to \frac{\Aut(A)}{\Exp(A)} 
  \]
that maps each pair $(f+(x^N),\lambda)$ to the class of the
automorphism~$\phi_{f,\lambda}$ is an isomorphism of groups. The quotient
$\Aut(A)/\Exp(A)$ has therefore a natural structure of Lie group over~$\kk$
of dimension $N+1$, solvable of class~$2$ and, in fact, an extension of
$\kk^N$ by $\kk^\times$. \qed
\end{Corollary}

The most interesting part of this is, of course, that dividing by the group
of inner automorphisms allows us to go from the infinite-dimensional group
$\Aut(A)$ to a finite-dimensional one. As elements of~$\Exp(A)$ has a
tendency to act trivially on objects associated to~$A$, this is useful. The
proof of the corollary is immediate given the description of have
of~$\Aut(A)$ and of~$\Exp(A)$, and we omit it.

\bigskip

Let us determine now the locally nilpotent derivations of our algebra ---
we already know those which are inner.

\begin{Proposition}\mbox{}\label{prop:lnd}
\begin{thmlist}

\item If $g\in\kk[x]$, then there is a unique derivation $d_g:A\to A$ such
that $d_g(x)=0$ and $d_g(y)=g$, it is locally nilpotent, it is inner
exactly when $g\in x^N\kk[x]$, and for all $t\in\kk$ we have that
$\exp(td_g) = \phi_{tg,1}$.

\item If $d:A\to A$ is a locally nilpotent, there is exactly one
polynomial~$g\in\kk[x]$ such that $d=d_g$.

\end{thmlist}
\end{Proposition}

\begin{proof}
Let $g$ be an element of~$\kk[x]$. That there is
indeed a derivation~$d_g:A\to A$ such that $d_g(x)=0$ and $d_g(y)=g$
follows by a trivial calculation using the presentation of~$A$. Since it is
a derivation, for all $i$,~$j\in\NN_0$ we have that
  \[
  d_g(x^iy^j) 
        = \sum_{s+1+t=j}x^iy^sgy^t
        \equiv jgx^{i}y^{j-1} \mod F_{j-2}.
  \]
This implies that $d_g(F_j)\subseteq F_{j-1}$ for all $j\in\NN_0$ and, in
particular, that $d_g$ is locally nilpotent. If $t\in\kk$, then
$\exp(td_g)(x)=x$ because $d_g(x)=0$ and $\exp (td_g)(y)=y+td_g(y)=y+tg$
because $d_g^2(y)=0$, and this tells us that
$\exp(t\partial_0)=\phi_{tg,1}$.

If there is an element~$a\in A$ such that $d_g=\ad(a)$, then
$0=d_g(x)=[a,x]$, so that $a\in\kk[x]$ because of
Lemma~\ref{lemma:centralizer:x}, and therefore $g=d_g(y)=[a,y]\in x^NA$.
Conversely, if $g=x^Nh$ for some $h\in\kk[x]$ and we chose $k\in\kk[x]$
such that $k'=-h$, then $d_g=\ad(k)$, so that $d_g$ is inner. With
this we have proved all the claims in part~\thmitem{1} of the proposition.

Next, let $d:A\to A$ be a locally nilpotent derivation of~$A$, so that, in
particular, there are two non-negative integers~$l$ and~$m$ such that
$d^{l+1}(x)=0$ and $d^{m+1}(y)=0$. As~$d$~is locally nilpotent, for each
$t\in\kk$ we can consider the automorphism $\exp(td):A\to A$. In view of
Proposition~\ref{prop:aut}, for each $t\in\kk$ there exist a non-zero
scalar~$\lambda_t\in\kk^\times$ and a polynomial~$f_t\in\kk[x]$ such that
$\exp(td)=\phi_{f_t,\lambda_t}$ and therefore
  \[ \label{eq:dxy}
  \sum_{i=0}^lt^i\frac{d^i(x)}{i!} = \lambda_t x, 
  \qquad
  \sum_{i=0}^mt^i\frac{d^i(y)}{i!} = \lambda_t^{N-1}y + f_t. 
  \]
Let $t_0$,~\dots,~$t_l$ be $l$ pairwise different elements of~$\kk$. The
Vandermonde matrix built out of those $l+1$ scalars is invertible: it
follows that from the $l+1$ equalities that we get from the first
one in~\eqref{eq:dxy} by replacing~$t$ by each of~$t_0$,~\dots,~$t_l$ we
can solve for~$d(x)$ and find that there is a scalar~$\alpha$ such that
  \[ \label{eq:dxx}
  d(x)=\alpha x.
  \]
Proceeding similarly with the second equation in~\eqref{eq:dxy} we see that
there are a scalar~$\beta\in\kk$ and a polynomial~$g\in\kk[x]$ such that 
  \[ \label{eq:dyy}
  d(y)=\beta y+g.
  \]

Of course, we must have that $d(yx-xy-x^N)=0$, and writing this out we find
that $\beta=(N-1)\alpha$. On the other hand, from~\eqref{eq:dxx}
and~\eqref{eq:dyy} we can see immediately that there is a sequence of
polynomial~$(g_n)_{n\geq0}$ in~$\kk[x]$ such that
  \[
  d^n(y) = (N-1)^n\alpha^n y + g_n
  \]
for all $n\in\NN_0$. As $d^m(y)=0$, this implies that $\alpha=0$. As $g$ is
uniquely determined by the derivation~$d$, this proves the
claim~\thmitem{2} of the proposition.
\end{proof}

This proposition allows us to give a particularly important example of a
locally nilpotent derivation:

\begin{Corollary}\label{coro:expp0}
There is a derivation $\partial_0:A\to A$ such that $\partial_0(x)=0$ and
$\partial_0(y)=x^{N-1}$, it is locally nilpotent, not inner, and
  \[ \label{eq:expp0}
  \exp(t\partial_0) = \sigma_t.
  \]
for all $t$. \qed
\end{Corollary}

Here $\sigma_t$ is the central automorphism of~$A$ that we described in
Corollary~\ref{coro:center}, so the derivation~$\partial_0$ is an
infinitesimal generator for the $1$-parameter subgroup of~$\Aut(A)$ that is
its center. We will show later that the class of the
derivation~$\partial_0$ generates the center of~$\HH^1(A)$. The point of
the following remark is that this derivation is also of Poisson-theoretic
interest.

\begin{Remark}\label{rem:modular}
In proving this proposition we have noted that $\partial_0(F_j)\subseteq
F_{j-1}$ for all $j\in\NN_0$, and this implies immediately that the
derivation $\partial_0$ induces a derivation $\overline\partial_0:\gr
A\to\gr A$ on the associated graded algebra~$\gr A$ of~$A$ such that
  \[
  \overline\partial_0(\overline x)=0,
  \qquad
  \overline\partial_0(\overline y)=\overline x^{N-1}.
  \]
This is a Poisson derivation of~$\gr F$ and, in fact, it is the
\newterm{modular derivation} of that Poisson algebra, in the sense of Alan
Weinstein in~\cite{Weinstein}, that is, the map
  \[ \label{eq:md}
  f\in\gr A\mapsto \div H_f\in\gr A.
  \]
As a consequence of this, the morphism in Corollary~\ref{coro:center} that,
according to Corollary~\ref{coro:expp0}, arises by exponentiation
from~$\partial_0$, is precisely the \newterm{modular flow} of the Poisson
algebra $\gr A$. In~\eqref{eq:md} we have written
$H_f\coloneqq\{f,\place\}$ for the Hamiltonian derivation corresponding to
an element~$f$ of~$\gr A$, and $\div$ for the divergence operator, so that 
  \[
  X\lact\d\overline x\wedge\d\overline y 
        = \div L\cdot\d\overline x\wedge\d\overline y
  \]
for each $X\in\Der(\gr A)$, with $\lact$ the action of vector fields on
differential forms.
\end{Remark}

\begin{Remark}
The derivation~$\partial_0$ is not inner, but it is a «logarithmic
derivation» in that
  \[
  \partial_0(a) = \frac{1}{x}[x,a]
  \]
for all $a\in A$. Here the right hand side of the equality is, in
principle, an element of the localization~$A_x$ of~$A$ at~$x$, but it turns
out to be in~$A$.
\end{Remark}

Knowing the locally nilpotent derivations we obtain another nice subgroup
of~$\Aut(A)$.

\begin{Corollary}\mbox{}\label{coro:aut-0}
\begin{thmlist}

\item The set $\LND(A)$ of all locally nilpotent derivations of~$A$ is an
abelian subalgebra of the Lie algebra~$\Der(A)$ of all derivations of~$A$.

\item The set of the exponentials of the elements of~$\LND(A)$ is the
normal abelian subgroup 
  \[
  \Aut_0(A) \coloneqq \{\phi_{g,1}:g\in\kk[x]\}
  \]
of~$\Aut(A)$. 

\item The function $\det:\Aut(A)\to\kk^\times$ such that
$\det(\phi_{f,\lambda})=\lambda$ whenever $f\in\kk[x]$ and
$\lambda\in\kk^\times$ is a morphism of groups. The sequence
  \[
  \begin{tikzcd}
  0 \arrow[r]
    & \Aut_0(A) \arrow[hook, r]
    & \Aut(A) \arrow[r, "\det"]
    & \kk^\times \arrow[r]
    & 1
  \end{tikzcd}
  \]
is an extension of groups that is split by the morphism
$\lambda\in\kk^\times\mapsto\phi_{0,\lambda}\in\Aut(A)$. \qed

\end{thmlist}
\end{Corollary}

All this follows immediately from Proposition~\ref{prop:lnd}. This
corollary describes the algebraic actions of the additive group $\GG_a$ on
the algebra~$A$. Indeed, such a thing is the same thing as a right
algebra-comodule structure on~$A$ over the coordinate Hopf algebra~$\kk[t]$
of~$\GG_a$, which is a coassociative morphism of algebras $\phi:A\to
A\otimes\kk[t]$. In fact, a linear map $A\to A\otimes\kk[t]$ is determined
by a sequence $(\phi_i)_{i\geq0}$ of linear maps~$A\to A$ such that for
each $a\in A$ the sequence $(\phi_i(a))_{i\geq0}$ has almost all its
components equal to zero and $\phi(a)=\sum_{i=0}^\infty\phi_i(a)$, and it
is a algebra-comodule structure on~$A$ over~$\kk[x]$ exactly when
$\phi_1:A\to A$ is a locally nilpotent derivation and $\phi_i=\phi_1^i/i!$
for all $i\in\NN_0$.

It should be remarked that the space of locally nilpotent derivations of an
algebra is most often not a subalgebra of the Lie algebra of
derivations of the algebra.

\bigskip

On the end of the spectrum opposite to where the $\GG_a$-actions lie are
the actions of finite groups. With the description of the automorphism
group of our algebra that we have we can easily find these. Later,
in Section~\ref{sect:taft}, we will consider more generally coactions of
some finite-dimensional Hopf algebras --- and that will require considerably
more work.

\begin{Proposition}\label{prop:finite-subgroups}\mbox{}
\begin{thmlist}

\item If $\phi$ is an element of~$\Aut(A)$ of finite order~$m$, then exists
a unique $\lambda\in\kk^\times$ of order exactly~$m$ such that $\phi$ is
conjugated in~$\Aut(A)$ to~$\phi_{0,\lambda}$.

\item If $\lambda$ is an element of order~$m$ in~$\kk^\times$, then the
automorphism $\phi_{0,\lambda}$ has order~$m$, and if 
$G\coloneqq\gen{\phi_{0,\lambda}}$ is the subgroup of~$\Aut(A)$ generated
by it, then the subalgebra~$A^G$ of invariants
under~$G$ is the $m$th Veronese subalgebra
$A^{(m)}=\bigoplus_{i\geq0}A_{mi}$.

\item Every finite subgroup of~$\Aut(A)$ is cyclic, and conjugated to the
subgroup generated by~$\phi_{0,\lambda}$, with $\lambda$ a root of unity
in~$\kk$. 

\end{thmlist}
\end{Proposition}

The first two parts of the proposition imply that for all $m\in\NN$ the
number of conjugacy classes of~$\Aut(A)$ of elements of order~$m$ coincides
with the number of elements of order~$m$ in~$\kk^\times$. On the other
hand, the third part tells us that set of conjugacy classes of finite
subgroups of~$\Aut(A)$ are in bijection with the set of integers~$m$ such
that there is a primitive $m$th root of unity in~$\kk$, the bijection being
given by taking the order.

\begin{proof}
Let $(f,\lambda)\in\kk[x]\bowtie\kk^\times$, let $k\in\NN$ and let us
suppose that the automorphism~$\phi_{f,\lambda}$ has order~$k$. Since
  \[
  \phi_{f,\lambda}^k(x) = \lambda^kx,
  \qquad
  \phi_{f,\lambda}^k(y) = \lambda^{k(N-1)}y
        + \sum_{i=0}^{k-1}\lambda^{i(N-1)}f(\lambda^{k-1-i}x),
  \]
we see that $\lambda^k=1$, so that $\lambda$ has finite order
in~$\kk^\times$, and that 
  \[ \label{eq:fxk}
  \sum_{i=0}^{k-1}\lambda^{i(N-1)}f(\lambda^{k-1-i}x) = 0.
  \]
Let $m$ be the order of~$\lambda$ in~$\kk^\times$, so that $m$ divides~$k$.
If $h\in\kk[x]$, then
  \begin{align}
  (\phi_{h,1}\circ\phi_{f,\lambda}\circ\phi_{h,1}^{-1})(x)
    &= \lambda x,
    \\
  (\phi_{h,1}\circ\phi_{f,\lambda}\circ\phi_{h,1}^{-1})(y)
    &= \lambda^{N-1}y+f(x)+\lambda^{N-1}h(x)-h(\lambda x)
  \end{align}
This tells us that replacing~$f$ by
$f+\lambda^{N-1}h(x)-h(\lambda x)$ does not change the conjugacy class
of~$\phi_{f,\lambda}$ in~$\Aut(A)$ and,
as the subspace 
  \(
  \gen[\Big]{\lambda^{N-1}h(x)-h(\lambda x):h\in\kk[x]}
  \)
of~$\kk[x]$ is spanned by the monomials $x^d$ with $d\not\equiv N-1\mod m$,
that up to conjugacy we can suppose that the polynomial~$f$ is a linear
combination of monomials~$x^d$ with $d\equiv N-1\mod m$. It is easy to
check now that when that is the case the left hand side of the
equality~\eqref{eq:fxk} is equal to~$kf(\lambda^{k-1}x)$, and therefore
that equality implies that $f=0$, since our ground field has
characteristic zero. This shows that every element of~$\Aut(A)$ of finite
order is conjugated to one of the form~$\phi_{0,\lambda}$ with $\lambda$ an
element of finite order in~$\kk^\times$, and that of course the order
of~$\phi_{0,\lambda}$ is equal to the order of~$\lambda$. To complete the
proof of the first part of the proposition we need only notice that if
$\lambda$ and~$\mu$ are two elements of~$\kk^\times$ the automorphisms
$\phi_{0,\lambda}$ and~$\phi_{0,\mu}$ are conjugated in~$\Aut(A)$ exactly
when $\lambda=\mu$: this is an immediate consequence of the fact that the
map 
  \[ \label{eq:pi}
  \pi:\phi_{f,\lambda}\in\Aut(A)\mapsto\lambda\in\kk^\times
  \]
is a morphism of groups.

If $m\in\NN$ and $\lambda$ is an element of order~$m$ in~$\kk^\times$, then
$\phi_{0,\lambda}(x^iy^j)=\lambda^{i+j(N-1)}x^iy^j$. We see immediately
from this that the subalgebra fixed by~$\phi_{0,\lambda}$, and by the
cyclic group it generates, is spanned by the monomials~$x^iy^j$ with $m\mid
i+j(N-1)$, that is, whose degree is divisible by~$m$. The claim~\thmitem{2}
of the proposition follows at once from this.

Suppose now that $G$ is a finite subgroup of $\Aut(A)$. If  $f$ and $g$ are
two elements of~$\kk[x]$ and $\lambda$ one of~$\kk^\times$ such that
$\phi_{f,\lambda}$ and~$\phi_{g,\lambda}$ are both in~$G$, then the
composition $\phi_{f,\lambda}\circ\phi_{g,\lambda}$ has finite order and
maps~$x$ and~$y$ to $x$ and to~$y+\lambda^{-N+1}(f-g)$, respectively: it
follows immediately from this that $f=g$. This means that the map~$\pi$
from~\eqref{eq:pi} is injective when restricted to~$G$ and therefore $G$ is
isomorphic to a finite subgroup of~$\kk^\times$ and, in particular, cyclic.
If $\phi$ is a generator of~$G$ and $m$ is the order of~$G$, we have seen
above that there is a primitive $m$th root of unity~$\lambda$ in~$\kk$ such
that $\phi$ is conjugated to~$\phi_{0,\lambda}$ and, therefore, $G$ is
conjugated to~$\gen{\phi_{0,\lambda}}$.
\end{proof}

Proposition~\ref{prop:finite-subgroups} gives us a classification of the
finite subgroups of~$\Aut(A)$ and their corresponding invariant
subalgebras, and it is natural to consider the classical problem of classifying
these up to isomorphism. This seems to be a rather complicated problem, as
even giving presentations for them is not easy. We will content ourselves
with describing two «extreme» instances of this problem.

Let $\lambda$ be a root of unity in~$\kk$, let $m$ be its order, and let us
suppose that $m>1$. We let $G$ be the subgroup generated
by~$\phi_{0,\lambda}$ in~$\Aut(A)$, which has order~$m$, so that the
invariant subalgebra~$A^G$ is the $m$th Veronese
subalgebra~$A^{(m)}$ of~$A$. 
\begin{thmlist}

\item\label{it:ag:1} Suppose first that $m$ divides $N-1$, and let $k$ be
the integer such that $N-1=km$. The invariant subalgebra~$A^G$ is easily
seen to be generated by~$x^n$ and $y$ and, since
  \[
  yx^n-x^ny=nx^{n+N-1}=n(x^n)^k,
  \]
isomorphic to $A_k=\kk\lin{x,y}/(yx-xy-x^k)$. We will see below, in
Corollary~\ref{coro:n-inv}, that as $m$ varies among the $\phi(N-1)$
positive divisors of~$N-1$ the algebras that we obtain in this way are
pairwise non-isomorphic. These algebras can obviously be generated by two
elements and not by one.

\item Suppose now that $N-1$ divides properly~$m$, and let $l$ be the
integer such that $m=l(N-1)$, which is at least~$2$. In this situation the
subalgebra~$A^G$ is generated by the $m$th homo\-geneous component of~$A$,
which has dimension $l+1$, so that that $A^G$ can be generated by~$l+1$
elements. On the other hand, $A^G$ cannot be generated by~$l$ elements:
indeed, if there existed $l$ elements~$a_1$,~\dots,~$a_l$ in~$A^G$ that
generate it as an algebra, the homogeneous components of those elements of
degree~$m$ would generate the $m$th homo\-geneous component of~$A$, and
this is absurd. We thus see that $A^G$ is minimally generated by~$l+1$
elements and that therefore the subalgebras that we obtain in this way are
pairwise non-isomorphic and, since $l\geq2$, also non-isomorphic to any of
the algebras that we found in~\ref{it:ag:1}.

\end{thmlist}
The smallest case not covered by these considerations is that in which
$N=3$ and $m=3$.
