\section{The Lie algebra structure on \texorpdfstring{$\HH^1(A)$}{HH1(A)}}
\label{sect:lie}

Now that we have explicit derivations whose classes freely span  $\HH^1(A)$
we can compute the canonical Lie algebra structure of this vector space. In
what follows we will write $\sim$ for the relation of cohomology between
derivations of~$A$.

\begin{Lemma}\label{lemma:brackets}
Let $l$ and $m$ be two integers such that $l\geq -N+1$, $m\geq -N+1$, and
let $i$, $j$, $u$ and~$v$ be the unique integers such that $l+1=i+j(N-1)$,
$m+1=u+v(N-1)$, $i$,~$u\in\inter{1,N-1}$ and $j$,~$v\geq-1$. Suppose that,
moreover, $l\leq m$.
\begin{thmlist}

\item If $i+u>N$, then $[\partial_l,\partial_m]\sim0$.

\item If $i+u\leq N$, then $[\partial_l,\partial_m]$ is cohomologous to
  \[
  \begin{c-dcases*}
    \left(u-i+\frac{N-i}{j+1}v-\frac{N-u}{v+1}j\right)\partial_{l+m}
          & if $l\geq1$ and $m\geq1$; \\
    0     & if $l\leq-1$ and~$m\leq-1$ or $lm=0$ \\
    v\partial_{l+m}
          & if $lm<0$ and $l+m\geq1$; \\
    -(l+m)\partial_{l+m}
          & if $lm<0$ and $l+m\leq-1$; \\
    E     & if $l=-N+1$ and~$m=N-1$.
  \end{c-dcases*}
  \]
Here we cannot have that $lm<0$, $l+m=0$ and $l>-N+1$.

\item Finally, $[E,\partial_m]=m\partial_m$.

\end{thmlist}
\end{Lemma}

\begin{proof}
The third claim of the lemma can be proved by a very simple direct
calculation that we omit. We will split the calculation that proves the
first two claims in several parts, and the following table describes in
which part we consider each particular combination of indices~$l$ and~$m$.
  \[
  \renewcommand{\arraystretch}{1.2}
  \begin{tabular}{c@{\hskip3em}c@{\hskip2em}c@{\hskip2em}c}
  {} & $l\leq-1$ & $l=0$ & $l\geq1$ \\ \toprule
  $m\leq-1$ & \textsc{Second} & \textsc{Third} & \textsc{Fourth} \\
  $m=0$     & \textsc{Third}  & \textsc{Third} & \textsc{Third} \\
  $m\geq1$  & \textsc{Fourth} & \textsc{Third} & \textsc{First}
  \end{tabular}
  \]

\textsc{First part.} Let us suppose first that $l\geq1$ and~$m\geq1$, so
that $j\geq0$ and $v\geq0$. If~$j\geq1$, then in view of
Proposition~\ref{prop:hh1-high} and Lemma~\ref{lemma:phi} we have that
modulo $F_{j-1}$
  \begin{align}
  \partial_l(x) 
        &\equiv x^iy^j
\shortintertext{and}
  \partial_l(y) 
        &=\sum_{s+2+t=N}(s+1)x^{s+i}y^jx^t + (N-i)x^{i-1}\Phi_{j+1} \\
        &\equiv \frac{N(N-1)}{2}x^{i+N-2}y^j
                +(N-i)x^{i-1}
                 \left(
                 \frac{1}{j+1}y^{j+1}-\frac{N}{2}x^{N-1}y^{j}
                 \right) \\
        &= \frac{N-i}{j+1}x^{i-1}y^{j+1} + \frac{N(i-1)}{2}x^{i+N-2}y^j,
  \end{align}
and using this we can see that
  \[ \label{eq:dldm}
  \partial_l(\partial_m(x))
        = \partial_l(x^uy^v)
        \equiv \left(u+\frac{N-i}{j+1}v\right) x^{i+u-1}y^{j+v}
        \mod F_{j+v-1}.
  \]
If instead $j=0$, then Lemma~\ref{lemma:phi} gives us a slightly different
formula for~$\Phi_{j+1}$ and what we have is that $\partial_l(x)=x^i$ and
  \[
  \partial_l(y) 
         =\sum_{s+2+t=N}(s+1)x^{s+i+t} + (N-i)x^{i-1}\Phi_{1} 
         = \frac{N(N-1)}{2}x^{i+N-2}
                +(N-i)x^{i-1}
                 y
  \]
and using this we can see that the congruence~\eqref{eq:dldm} also holds
when $j=0$.

By symmetry we get from~\eqref{eq:dldm} a formula for
$\partial_m(\partial_l(x))$, and finally conclude that
  \[ \label{eq:plpm:x}
  [\partial_l,\partial_m](x) 
        \equiv
                \left(
                u-i+\frac{N-i}{j+1}v-\frac{N-u}{v+1}j
                \right) x^{i+u-1}y^{j+v}
                \mod F_{j+v-1}.
  \]
There are now two cases.
\begin{itemize}

\item If $i+u>N$, then we have 
  \[
  l+m+1=(i+u-N)+(j+v+1)(N-1), 
  \qquad
  1\leq i+u-N\leq N-1,
  \]
and the above formula~\eqref{eq:plpm:x} tells us that the
derivation~$[\partial_l,\partial_m]$, which is homogeneous of degree~$l+m$,
maps~$x$ into~$F_{j+v}$: it follows from Lemma~\ref{lemma:cut} that
$[\partial_l,\partial_m]$ is inner in this situation.

\item Suppose now that $i+u\leq N$, so that
  \[
  l+m+1=(i+u-1)+(j+v)(N-1),
  \qquad
  1\leq i+u-1\leq N-1,
  \]
and let $\alpha$ be the scalar that appears between parentheses
in the right hand side of the congruence~\eqref{eq:plpm:x}. We then have that
the derivation $[\partial_l,\partial_m]-\alpha\partial_{l+m}$, which is
homogeneous of degree~$l+m$, maps~$x$ into~$F_{j+v-1}$: again using
Lemma~\ref{lemma:cut} we see that that difference is inner. 
We thus have that $[\partial_l,\partial_m]\sim\alpha\partial_{l+m}$ in this
situation.

\end{itemize}

\textsc{Second part.} If $l\leq-1$ and $m\leq-1$, then it follows immediately
from the description of~$\partial_l$ and~$\partial_m$ given in
Proposition~\ref{prop:hh1-low} that $[\partial_l,\partial_m]=0$,
independently of whether the inequality $i+u\geq N$ holds or not.

\medskip

\textsc{Third part.} We now want to prove that the
derivation~$[\partial_l,\partial_m]$ is inner if one of~$l$ or~$m$ is zero
and, of course, we can suppose that it is $l$ that is zero. Let us notice
that in this situation we have that 
$i=1$, $j=0$ and $i+u\leq N$.

If $m\geq1$, then $\partial_m(\partial_0(x))=0$ and
  \[
  \partial_0(\partial_m(x))
        = \partial_0(x^uy^v)
        = \sum_{s+1+t=v}x^uy^sx^{N-1}y^t
        \in F_{v-1},
  \]
so that $[\partial_0,\partial_m]$ is a homogeneous derivation of degree~$m$
that maps~$x$ into~$F_{v-1}$: as $m+1=u+v(N-1)$ and $1\leq u\leq N-1$, we know
from Lemma~\ref{lemma:cut} that that commutator is inner. If instead
$m\leq0$, then using the description of~$\partial_m$ given in
Proposition~\ref{prop:hh1-low} can compute directly that
$[\partial_0,\partial_m]=0$.

\medskip

\textsc{Fourth part.} Finally, let us consider the case in which $lm<0$
and, without any loss of generality thanks to anti-symmetry, $l\leq-1$ and
$m\geq1$. We have that 
  \begin{gather}
  j = -1, \qquad
  i = l+N,  \qquad
  v \geq 0, \\
  l+m+1=l+u+v(N-1), \qquad
  -N+2\leq l+u\leq N-2, \label{eq:r2}
  \end{gather}
and that, as $\partial_m(\partial_l(x))=0$, 
  \[
  [\partial_l,\partial_m](x)
        = \partial_l(\partial_m(x))
        = \partial_l(x^uy^v)
        = \sum_{s+1+t=v}x^uy^sx^{l+N-1}y^t. \label{eq:vx0}
  \]
Let us suppose first that $i+u>N$.
\begin{itemize}

\item If $v=0$ and $i+u>N+1$, then 
$l+m = i-N+u-1 \geq 1$, so that the degree of~$[\partial_l,\partial_m]$ is
positive, and, since $1\leq i-N+u= l+u\leq N-2$, it follows from the
first equality in~\eqref{eq:r2}, the formula~\eqref{eq:vx0} tells us 
that $[\partial_l,\partial_m](x)=0\in F_{v-1}$, and Lemma~\ref{lemma:cut},
that $[\partial_l,\partial_m]\sim0$.

\item If $v=0$ and $i+u=N+1$, then the formula~\eqref{eq:vx0} tells us
again that $[\partial_l,\partial_m](x)=0$, and using the definitions
of~$\partial_l$ and~$\partial_m$ we see that
  \[
  \partial_l(\partial_m(y))
        = \partial_l
                \left(
                \frac{N(N-1)}{2}x^{u+N-2}+(N-u)x^{u-1}y
                \right)
        = (N-u)x^{N-1}
  \]
and
  \[
  \partial_m(\partial_l(y))
        = \partial_m(x^{l+N-1})
        = (l+N-1)x^{N-1}
        = (i-1)x^{N-1}
  \]
so that $[\partial_l,\partial_m](y)=(N-u-i+1)x^{N-1}=0$ and, therefore,
$[\partial_l,\partial_m]=0$.

\item If $v\geq1$, then $l+m=i-N+u+v(N-1)-1>N-2\geq0$, so that the degree
of the derivation~$[\partial_l,\partial_m]$ is positive, and since
$l+m+1=i-N+u+v(N-1)$, $1\leq i-N+u\leq N-2$, and
$[\partial_l,\partial_m](x)\in F_{v-1}$ because of~\eqref{eq:vx0},
Lemma~\ref{lemma:cut} tells us that $[\partial_l,\partial_m]\sim0$.

\end{itemize}
Let us suppose, finally, that $i+u\leq N$. We then have that 
  \[ \label{eq:vy0}
  l+m+1 = (i+u-1)+(v-1)(N-1), \qquad
  1\leq i+u-1\leq N-1,
  \]
and, as before, we consider several cases.
\begin{itemize}

\item If $l+m\geq1$, then \eqref{eq:vy0} implies that
$\partial_{l+m}(x)=x^{i+u-1}y^{v-1}$ while \eqref{eq:vx0} implies that
$[\partial_l,\partial_m](x)\equiv vx^{i+u-1}y^{v-1}\mod F_{v-1}$: it
follows from this that $[\partial_l,\partial_m]-v\partial_{l+m}$, a
homogeneous derivation of degree~$l+m$, maps~$x$ into~$F_{v-2}$, so that
Lemma~\ref{lemma:cut} and~\eqref{eq:vy0} let us conclude that it is inner
and, therefore, that $[\partial_l,\partial_m]\sim v\partial_{l+m}$.

\item Suppose now that $l+m\leq -1$. We have
that $v=0$, for otherwise $v\geq1$ and 
  \[
  0 \geq l+m+1 = i-N+u+v(N-1) \geq i-N+u+N-1 = i+u-1 \geq1,
  \]
which is absurd. From~\eqref{eq:vx0} we see then that
$[\partial_l,\partial_m](x)=0$ and we can compute directly that
  \begin{gather}
  \partial_l(\partial_m(y))
        = \partial_l
                \left(
                \frac{N(N-1)}{2}x^{u+N-2} + (N-u)x^{u-1}y
                \right)
        = (N-u)x^{l+m+N-1}
\shortintertext{and}
  \partial_m(\partial_l(y))
        = \partial_m(x^{l+N-1})
        = (l+N-1)x^{l+m+N-1},
  \end{gather}
so that $[\partial_l,\partial_m](y)=-(l+m)x^{l+m+N-1}$, and therefore
$[\partial_l,\partial_m]=-(l+m)\partial_{l+m}$.

\item It is easy to check that we cannot have $l+m=0$ and $l>-N+1$.

\item We have one last case to consider: that in which $l+m=0$ and $l=-N+1$.
We then have that $m=N-1$, $u=1$ and $v=1$, and computing directly we see
that $[\partial_l,\partial_m]$ maps~$x$ and~$y$ to~$x$ and~$(N-1)y$, so
that it is equal to~$E$.

\end{itemize}
We have proved all the claims in the lemma.
\end{proof}

It will be convenient to work with a different basis of~$\HH^1(A)$ with
respect to which the structure constants of the Lie bracket are simpler. If
$l$ is an integer such that $l\geq-N+1$, there is a unique way of choosing
integers~$i$ and~$j$ such that $i\in\inter{1,N-1}$, $j\geq-1$ and
$l+1=i+j(N-1)$, and we define
  \[
  L_l \coloneqq
        \begin{r-dcases*}
        -\frac{j+1}{N-1}\partial_l & if $l\geq1$; \\ 
        -\frac{1}{N-1}E & if $l=0$; \\
        \frac{l}{N-1} \partial_l & if $l\leq -1$.
        \end{r-dcases*}
  \]
Clearly the classes of $\partial_0$ and the derivations $L_l$ with
$l\geq-N+1$ freely span~$\HH^1(A)$.

\begin{Corollary}\label{coro:mlmm}
Let $l$ and $m$ be two integers such that $l\geq -N+1$, $m\geq -N+1$, and
let $i$, $j$, $u$ and~$v$ be the unique integers such that $l+1=i+j(N-1)$,
$m+1=u+v(N-1)$, $i$,~$u\in\inter{1,N-1}$ and $j$,~$v\geq-1$. We have that
  \[ \label{eq:mlmm}
  [L_l, L_m] \sim
    \begin{dcases*}
    0 & if $i+u>N$ or $l+m<-N+1$; \\
    \frac{l(v+1)-m(j+1)}{N-1}L_{l+m} & if $i+u\leq N$.
    \end{dcases*}
  \]
\end{Corollary}

\begin{proof}
Let us consider first the situation in which $i+u>N$. Neither~$i$ nor~$u$
is equal to~$1$: were $i=1$, for example, we would have that $i+u\leq
1+(N-1)=N$. It follows then, in particular, that $l\neq0$ and $m\neq0$, so
there are scalars~$a$ and~$b$ such that $L_l=a\partial_l$ and
$L_m=b\partial_m$, and therefore that
$[L_l,L_m]=ab[\partial_l,\partial_m]\sim0$ because of the first part of
Lemma~\ref{lemma:brackets}. This proves the first line in~\eqref{eq:mlmm}.

In order to prove the second line let us suppose from now on that 
$i+u\leq N$ and, as if bound by a hex cast upon us, handle each of the 
several following special cases separately.
\begin{itemize}

\item Suppose first that $l\leq-1$ and $m\leq-1$. As before, there are
scalars~$a$ and~$b$ such that $L_l=a\partial_l$ and $L_m=b\partial_m$, and
therefore $[L_l,L_m]=ab[\partial_l,\partial_m]\sim0$ according to
Lemma~\ref{lemma:brackets}. On the other hand, we have that $j=-1$ and
$v=-1$, so that the numerator of the fraction appearing in~\eqref{eq:mlmm}
is~$0$ and that cohomology holds in this case.

\item Suppose now that $l\geq1$ and $m\geq1$. Using
Lemma~\ref{lemma:brackets} we see that $[L_l,L_m]$ is 
  \[
  \frac{(j+1)(v+1)}{(N-1)^2} [\partial_l,\partial_m] 
        \sim \frac{(j+1)(v+1)}{(N-1)^2}
          \left(u-i+\frac{N-i}{j+1}v-\frac{N-u}{v+1}j\right)\partial_{l+m},
  \]
and a little calculation shows that this is equal to 
  \[
  \frac{l(v+1)-m(j+1)}{N-1}L_{l+m}.
  \]

\item Suppose next that $lm=0$, so that one of~$l$ or~$m$ is zero --- and
by anti-symmetry we can suppose additionally that $l$ is, so that $j=0$. If
$m\geq1$, then
  \[
    [L_l,L_m]
      = \left[-\frac{1}{N-1}E,-\frac{v+1}{N-1}\partial_m\right] 
      \sim m\frac{v+1}{(N-1)^2}\partial_m
      = -\frac{m}{N-1}L_m,
  \]
if $m=0$, then~$[L_l,L_m]=0$, and if $m\leq-1$, then
  \[
    [L_l,L_m]
      = \left[-\frac{1}{N-1}E,\frac{m}{N-1}\partial_m\right] 
      \sim -\frac{m^2}{(N-1)^2}\partial_m
      = -\frac{m}{N-1}L_m 
  \]
In any case, we see that \eqref{eq:mlmm} holds.

\end{itemize}
At this point we are left with considering the case in which $lm<0$ and,
thanks to anti-symmetry, we can further suppose that in fact $l\leq-1$ and
$m\geq1$.
\begin{itemize}

\item If $l\leq-1$, $m\geq1$ and $l+m\geq1$, then
$j=-1$, $1\leq i+u-1\leq N-1$, and $l+m+1=(i+u-1)+(v-1)(N-1)$, so that
$L_{l+m}=-v\partial_{l+m}/(N-1)$ and
  \[
  [L_l,L_m]
        = \left[\frac{l}{N-1}\partial_l,-\frac{v+1}{N-1}\partial_m\right]
        = -\frac{l(v+1)v}{(N-1)^2}\partial_{l+m}
        = \frac{l(v+1)}{N-1}L_{l+m}.
  \]

\item Suppose now that $l\leq-1$, $m\geq1$ and $l+m=0$. We then have that
$j=-1$, that $0=l+m=(i+u-2)+(v-1)(N-1)$, and that $0\leq i+u-2\leq N-2$, so
that in fact $i=1$, $u=1$ and $v=1$: this tells us that $l=-N+1$ and
$m=N-1$, so that
  \[
  [L_l,L_m]
        = \left[
                \frac{-N+1}{N-1}\partial_{-N+1},
                -\frac{2}{N-1}\partial_{N-1}
          \right]
        = \frac{2}{N-1} E
        = -2L_0
  \]

\item Finally, if $l\leq-1$, $m\geq1$ and $l+m\leq-1$, then
  \[
  [L_l,L_m]
        = \left[
                \frac{l}{N-1}\partial_{l},
                -\frac{v+1}{N-1}\partial_{m}
          \right]
        = \frac{l(v+1)(l+m)}{(N-1)^2}\partial_{l+m}
        = \frac{l(v+1)}{N-1}L_{l+m}.
  \]

\end{itemize}
In all cases what we have found is a specialization of the formula that
appears in~\eqref{eq:mlmm}.
\end{proof}

For each integer~$l$ let us write $\rho(l)$ for the element
of~$\inter{0,N-2}$ that is the remainder of the division of~$l$ by~$N-1$.
Above we have used many times the fact that $l$ determines uniquely
integers~$i$ and~$j$ such that $l+1=i+j(N-1)$: we have $\rho(l)=i-1$.
Clearly, whenever $l$ and $m$ are integers we have that
\begin{claim}
  $\rho(l+m)=\rho(l)+\rho(m)$ if $\rho(l)+\rho(m)\leq N-2$.
\end{claim}
From now on we will, in contexts where this does not lead to confusion,
take the liberty of not making an explicit difference between a derivation
of~$A$ and its class in~$\HH^1(A)$.

Let us recall that the Lie algebra~$\Der(\kk[t])$ of derivations of the
polynomial algebra~$\kk[t]$ is freely spanned as a vector space by the
derivations
  \[
  - t^{j+1}\frac{\d}{\d t}, \qquad j\in\ZZ,\; j\geq-1,
  \]
which are such that
  \[ \label{eq:witt}
  \left[- t^{j+1}\frac{\d}{\d t}, - t^{v+1}\frac{\d}{\d t}\right] 
        = (j-v)L_{j+v}
  \]
whenever $j$ and~$v$ are integers such that $j$,~$v\geq-1$. This Lie
algebra is often called the \newterm{Witt algebra}, although the same name
is more commonly applied to the Lie algebra~$\Der(\kk[t^{\pm1}])$, which is
different --- there is a \emph{third} Lie algebra called the Witt algebra,
but it only occurs in positive characteristic. The Lie algebra
$\Der(\kk[x])$ is simple --- in fact, David Jordan proved in~\cite{Jordan}
that the Lie algebra of derivations on an affine variety is simple exactly
when the variety is smooth, and that applies here (see also Thomas
Siebert's \cite{Siebert}), but one can prove this particular case very
easily «by hand».

\begin{Proposition}\label{prop:nil}
For each $r\in\NN_0$ let $\Nil_r$ be the span of $\{L_l:\rho(l)\geq r\}$
in~$\HH^1(A)$, so that, in particular, $\Nil_r=0$ when $r\geq N-2$ and
there is a descending chain of subspaces
  \[ \label{eq:nils}
        \Nil_0 \supsetneq 
          \underbracket{
              \Nil_1
              \supsetneq \cdots
              \supsetneq \Nil_{N-2}
              \supsetneq \Nil_{N-1} = 0
              }
        .
  \]
\begin{thmlist}

\item For each $r$,~$s\in\inter{0,N-1}$ we have that
$[\Nil_r,\Nil_s]\subseteq\Nil_{r+s}$.

\item For each $r\in\inter{1,N-2}$ we have that
$\Nil_{r+1}\subseteq[L_{-N+2},\Nil_r]\subseteq[\Nil_1,\Nil_r]$.

\item $\Nil_1$ is a nilpotent ideal in~$\HH^1(A)$, the chain underlined
in~\eqref{eq:nils} is its lower central series and, in particular, its
nilpotency index is exactly~$N-2$.

\item The center of~$\HH^1(A)$ is the subspace~$\kk\partial_0$ spanned
by~$\partial_0$, the derived subalgebra of~$\HH^1(A)$ is~$\Nil_0$, which is
a perfect subalgebra, and 
  \[
  \HH^1(A) = Z \oplus \Nil_0.
  \]

\item There is an injective morphism of Lie algebras
$\Phi:\Der(\kk[t])\to\Nil_0$ that for all integers~$j$ such that $j\geq-1$
has
  \[
  \Phi\left(-t^{j+1}\frac{\d}{\d t}\right) = L_{j(N-1)}.
  \]

\item $\Nil_0=\Phi(\Der(\kk[t]))\oplus\Nil_1$ and $\Nil_1$ is the unique
maximal ideal of~$\Nil_0$.

\end{thmlist}
\end{Proposition}

\begin{proof}
\thmitem{1} Let $r$ and~$s$ be elements of~$\inter{0,N-1}$, let $l$ and~$m$
be integers such that $l\geq-N+1$, $m\geq-N+1$, $\rho(l)\geq r$
and~$\rho(m)\geq s$, and let $i$,~$j$,~$u$,~$v$ be the integers such that
$i$,~$u\in\inter{1,N-1}$, $j$,~$v\geq-1$, $l+1=i+j(N-1)$ and
$m+1=u+v(N-1)$, so that $i=\rho(l)+1$ and $u=\rho(m)+1$. If
$\rho(l)+\rho(m)>N-2$, then we know from Corollary~\ref{coro:mlmm} that
$[L_l,L_m]=0\in\Nil_{r+s}$. If instead $\rho(l)+\rho(m)\leq N-2$, then
$\rho(l+m)=\rho(l)+\rho(m)\geq r+s$ and that corollary tells us now that
there is a scalar~$\lambda\in\kk$ such that $[L_l,L_m]=\lambda
L_{l+m}\in\Nil_{r+s}$. This shows that
$[\Nil_r,\Nil_s]\subseteq\Nil_{r+s}$.

\thmitem{2} Let $r$ be an element of~$\inter{1,N-2}$, so that in particular
we have $N\geq3$, and let $m$ be an integer such that $m\geq-N+1$ and
$\rho(m)\geq r+1$. There is an integer $v$ such that $v\geq-1$ and
$m+1=\rho(m)+1+v(N-1)$, and we set $m'\coloneqq\rho(m)-1+(v+1)(N-1)$. Since
$2\leq r+1\leq \rho(m)\leq N-2$, we have that $\rho(m')=\rho(m)-1\geq r$ and,
therefore, that $L_{m'}\in\Nil_{r+1}$. Since $\rho(m)\geq2$ and
$v\geq-1$, we have $m'\geq1$. Since $(-N+2)+m'=m\neq0$ because
$\rho(m)\geq r+1>0$, it follows from Lemma~\ref{lemma:brackets} that
  \[
  [L_{-N+2},L_{m'}]
        = -\frac{N-2}{N-1}(v+2)L_{m}
  \]
and, in any case, that $L_m\in[L_{-N+2},\Nil_r]$. We have
proved that $\Nil_{r+1}\subseteq[L_{-N+2},\Nil_r]$.

\bigskip

Lemma~\ref{lemma:brackets} tells us that the class of the
derivation~$\partial_0$ is central in~$\HH^1(A)$. On the other hand, a
central element of~$\HH^1(A)$ has to commute with $E$, so has
degree~$0$: as $E$ is not central, we see that the center of~$\HH^1(A)$
es precisely~$\kk\partial_0$. Clearly $\HH^1(A)=\kk\partial_0\oplus\Nil_0$
and the derived subalgebra of~$\HH^1(A)$ is~$[\Nil_0,\Nil_0]$.

The map~$\Phi$ described in the lemma is immediately seen to be an
injective morphism of Lie algebras --- thanks to Corollary~\ref{coro:mlmm}
and the formulas in~\eqref{eq:witt} --- and clearly
$\Nil_0=\Phi(\Der(\kk[t]))\oplus\Nil_1$.
As the algebra~$\Der(\kk[x])$ is simple, it is perfect, and thus
$\Phi(\Der(\kk[t]))$ is contained in~$[\Nil_0,\Nil_0]$. As also
$\Nil_1=[M_0,\Nil_1]\subseteq[\Nil_0,\Nil_0]$, we see that $\Nil_0$ is
perfect and equal to the derived subalgebra of~$\HH^1(A)$.

From~\thmitem{1} we have that $[\Nil_0,\Nil_1]\subseteq\Nil_1$, so that the
subspace $\Nil_1$ is an ideal of~$\Nil_0$ and in~$\HH^1(A)$.
From~\thmitem{1} and~\thmitem{2} we see that for each $r\in\inter{1,N-2}$
we have $[\Nil_1,\Nil_r]=\Nil_{r+1}$, so that
$\Nil_1\supsetneq\Nil_2\supsetneq\cdots\supsetneq\Nil_{N-2}$ is the
beginning of the lower central series of~$\Nil_1$. As $\Nil_{N-2}=0$ and
$\Nil_{N-1}\neq0$, we see that $\Nil_1$ is nilpotent of index
exactly~$N-2$. As $\Nil_0/\Nil_1\cong\Der(\kk[t]))$, a simple Lie
algebra, $\Nil_1$ is a maximal ideal in~$\Nil_0$. 

Suppose, finally, that
$I$ is an ideal of~$\Nil_0$: as $I$ is $\ad(E)$-invariant and
$\ad(E):\Nil_0\to\Nil_0$ is diagonalizable and its eigenspaces are the
homogeneous components of~$\Nil_0$, we see that $I$ is a \emph{homogeneous}
ideal of~$\Nil_0$. The intersection $I\cap\Phi(\Der(\kk[t])$ is zero,
because it is an ideal in~$\Phi(\Der(\kk[t]))$, and this tells us that $I$
is generated by homogeneous elements of degree not divisible by~$N-1$ and
thus that it is contained in~$\Nil_1$. This proves that~$\Nil_1$ is the
unique maximal ideal of~$\Nil_0$.
\end{proof}

An immediate consequence of this is that the number~$N$ is a derived
invariant of the algebra~$A$:

\begin{Corollary}\label{coro:n-inv}
If $N$ and $N'$ are two non-negative integers such that the algebras~$A_N$
and~$A_{N'}$ are derived equivalent, then $N=N'$.
\end{Corollary}

\begin{proof}
Let $N$ and~$N'$ be two non-negative integers such that $N\leq N'$ and the
algebras $A_N$ and~$A_{N'}$ are derived equivalent. According to a theorem
of Bernhard Keller in~\cite{Keller} we then have that the Lie
algebras~$\HH^1(A_N)$ and~$\HH^1(A_{N'})$ are isomorphic as Lie algebras.
If $N\geq2$, then Proposition~\ref{prop:nil} tells us that we can compute
the number~$N-2$ from the Lie algebra~$\HH^1(A_N)$ as the nilpotency index
of the unique maximal ideal of the quotient of $\HH^1(A_N)$ by its center
is $N-2$, and we therefore have that $N=N'$.

It is easy to deal with the remaining possibilities. If $N=0$, then $A_N$
is the Weyl algebra and $\HH^1(A)=0$: from
Propositions~\ref{prop:hh1-series} and~\ref{prop:hh1-n1} we see that then
$N'=0$. If $N=1$, then we cannot have $N'>1$, for in that case
$\dim\HH^1(A_N)=1<\infty=\dim\HH^1(A_{N'})$, so again $N=N'$.
\end{proof}

The next thing we want to do is to obtain a computational criterion for a
derivation of~$A$ to be the restriction of an inner derivation of the
algebra~$A_x$ obtained by localizing at~$x$. This is possible because $A_x$
can also be obtained as a localization of a Weyl algebra, with which we can
work with ease. The result we need is the following:

\begin{Proposition}\label{prop:xinner:weyl}
Let $\W$ be the algebra freely generated by letters~$p$ and~$q$ subject to
the relation
  \[
  pq-qp = 1,
  \]
and let $\W_q$ be the localization of~$\W$ at~$q$. The first Hochschild
cohomology space~$\HH^1(\W_q)$ is one-dimensional and is spanned by the
cohomology class of the derivation $\partial:\W_q\to\W_q$ such that
  \[
  \partial(p) = q^{-1}, 
  \qquad
  \partial(q) = 0.
  \]
There is a representation of~$\W_q$ on the algebra~$\kk[t^{\pm1}]$ of
Laurent polynomials such that 
  \[
  p\lact f(t) = \frac{\d}{\d t}f(t),
  \qquad
  q\lact f(t) = tf(t)
  \]
for all~$f\in\kk[t^{\pm1}]$. A derivation $\delta:\W_q\to\W_q$ is inner if
and only if
  \[ \label{eq:res:inn}
  \Res_0\left(\delta(pq)\lact\frac{1}{t}\right) = 0.
  \]
\end{Proposition}

Here $\Res_0:\kk[t^{\pm1}]\to\kk$ is the usual residue map that picks the
coefficient of $t^{-1}$ in its argument. Just as the Weyl algebra~$\W$ is
the algebra of regular differential operators on the affine line~$\AA^1$,
the localization~$\W_q$ is the algebra of regular differential operators on
the punctured affine line~$\AA^1\setminus\{0\}$ --- and this is the
representation~$\lact$ that appears in the proposition. As this is a smooth
affine scheme and our ground field has characteristic zero, we know that
the Hochschild cohomology~$\HH^*(\W_q)$ is isomorphic to the algebraic De
Rham cohomology~$\H^*(\AA^1\setminus\{0\})$. In particular, cohomology
classes of derivations of~$\W_q$ can be viewed as cohomology classes of
$1$-forms on~$\AA^1\setminus\{0\}$: the condition~\eqref{eq:res:inn} above
for a derivation to be inner corresponds to the condition that a $1$-form
«integrate to zero along a curve around the puncture» for it to be a
exact, so it is a version of Poincaré duality in our context. We will prove
the proposition in a purely algebraic way, but geometry and
various comparison maps could be used instead.

\begin{proof}
Let $W$ be the subspace of~$\W$ spanned by~$p$ and~$q$, let $\hat W$ be its
dual space, and let $(\hat p,\hat q)$ be the ordered basis of~$\hat W$ dual
to~$(p,q)$. There is a projective resolution of~$\W$ as a $\W$-bimodule of
the form
  \[
  \begin{tikzcd}
  \W\otimes\Lambda^2W\otimes\W \arrow[r, "d"]
    & \W\otimes W\otimes\W \arrow[r, "d"]
    & \W\otimes\W \arrow[r, dashed, "\epsilon"]
    & \W
  \end{tikzcd}
  \]
with differentials such that
  \begin{align}
  &d(1\otimes 1) = 1, \\
  &d(1\otimes w\otimes 1) = w\otimes 1-1\otimes w, \qquad\forall w\in W, \\
  &d(1\otimes p\wedge w\otimes 1) =
        p\otimes q\otimes1 + 1\otimes p\otimes q
        - q\otimes p\otimes 1 - 1\otimes q\otimes p,
  \end{align}
and augmentation $\epsilon:\W\otimes\W\to\W$ given by the multiplication of
the algebra~$\W$. The localization $\W_q$ of~$\W$ at~$q$ is a $\W$-bimodule
that is flat on both sides, so applying the functor
$\W_q\otimes_\W(\mathord-)\otimes_\W\W_q$ to the resolution above gives a
projective resolution 
  \[
  \begin{tikzcd}
  \W_q\otimes\Lambda^2W\otimes\W_q \arrow[r, "d"]
    & \W_q\otimes W\otimes\W_q \arrow[r, "d"]
    & \W_q\otimes\W_q \arrow[r, dashed, "\epsilon"]
    & \W_q\otimes_\W\W_q
  \end{tikzcd}
  \]
of the $\W_q$-bimodule $\W_q\otimes_\W\W_q$. As the map 
$\W_q\otimes_\W\W_q\to\W_q$ induced by the multiplication of~$\W_q$ is an
isomorphism of~$\W_q$-bimodules, what we have is a projective resolution
of~$\W_q$ as a bimodule over itself.
Applying to it the functor $\hom_{\W_q^e}(\place,\W_q)$ and doing
standard identifications we obtain the complex
  \[ \label{eq:w:comp}
  \begin{tikzcd}
  \W_q \arrow[r, "\delta"]
    & \W_q\otimes W \arrow[r, "\delta"]
        \arrow[l, dashed, shift left=1ex, "s_1"]
    & \W_q\otimes\Lambda^2\hat W 
        \arrow[l, dashed, shift left=1ex, "s_2"]
  \end{tikzcd}
  \]
with differentials such that for all $a$ and~$b$ in~$\W_q$ have
  \[
    \delta(a) 
        = [p,a]\otimes\hat p + [q,a]\otimes\hat q, 
    \qquad
    \delta(a\otimes\hat p+b\otimes\hat q)    
        = \bigl([p,b]-[q,a]\bigr)\otimes\hat p\wedge\hat q.
  \]
Let us write $\X$ for this complex of vector spaces, whose cohomology
is~$\HH^*(\W_q)$, the Hochschild cohomology we are trying to compute, up
to canonical isomorphisms.

There is a $\ZZ$-grading on the algebra~$\W_q$ that assigns to~$p$ and~$q$
degrees~$-1$ and~$1$, respectively. On the other hand, the inner derivation
$\ad(pq):\W_q\to\W_q$ is diagonalizable with spectrum~$\ZZ$, and for each
$n\in\ZZ$ the eigenspace of~$\ad(pq)$ corresponding to the eigenvalue~$n$
is precisely the homogeneous component of~$\W_q$ of degree~$n$ --- this
follows at once from the facts that $\ad(pq)(p)=-p$ and $\ad(pq)=q$, on one
hand, and, on the other, that the set $\{p^iq^j:i\in\NN_0,j\in\ZZ\}$ is a
basis of~$\W_q$.

We consider on the complex~$\X$ the maps~$s_1$ and~$s_2$ indicated with
dashed arrows in the diagram~\eqref{eq:w:comp} above that are given by
  \[
  s_1(a\otimes\hat p+b\otimes\hat q) = aq+pb, 
  \qquad
  s_2(a\otimes\hat p\wedge\hat q) = -pa\otimes\hat p+aq\otimes\hat q
  \]
for all $a$ and~$b$ in~$A$, and define maps
  \begin{align}
  & r_0 \coloneqq s_1\circ\delta 
        : \W_q\to\W_q, \\
  & r_1 \coloneqq \delta\circ s_1 + s_2\circ\delta
        : \W_q\otimes W\to\W_q\otimes W, \\
  & r_2 \coloneqq \delta\circ s_1
        : \W_q\otimes\Lambda^2W \to \W_q\otimes\Lambda^2W.
  \end{align}
Of course, in this way we obtain an endomorphism $r_*:\X\to\X$ of the
complex~$\X$ that is homotopic to zero. A simple calculation shows that
  \begin{align}
  & r_0(a) 
        = \ad(pq)(a),  \\
  & r_1(a\otimes\hat p+b\otimes\hat q)
        = (\ad(pq)(a)-a)\otimes\hat p + (\ad(pq)(b)+b)\otimes\hat q, \\
  & r_2(a\otimes\hat p\wedge\otimes\hat q)
        = \ad(pq)(q)\otimes\hat p\otimes\hat q
  \end{align}
for all choices of~$a$ and~$b$ in~$\W_q$, and it follows from this that the
endomorphism~$r_*$ is diagonalizable, since $\ad(pq)$ is. As a consequence
of all this, the subcomplex~$\X'$ of~$\X$ spanned by all eigenvectors of
the map~$r_*$ corresponding to non-zero eigenvalues is exact and a
complement to~$\ker r_*$, so that 
  \[ \label{eq:xx:desc}
  \X=\X'\oplus\ker r_*.
  \]
In particular, the inclusion $\ker r_*\hookrightarrow\X$ is a quasi-isomorphism.

The kernel~$\ker r_*$ is easily computed to be the complex
  \[ \label{eq:w:comp:red}
  \begin{tikzcd}
  \kk[pq] \arrow[r, "\delta"]
    & \bigl(\kk[pq]q^{-1}\otimes\kk\hat p\bigr)
        \oplus 
      \bigl(\kk[pq]q\otimes\kk\hat q\bigr)
      \arrow[r, "\delta"]
    & \kk[pq]\otimes\kk\hat p\wedge\hat q
  \end{tikzcd}
  \]
with differentials such that
  \begin{align}
  & \delta(p^iq^i) 
        = ip^iq^{i-1}\otimes\hat p - ip^{i-1}q^i\otimes\hat q, \\
  & \delta(p^iq^{i-1}\otimes\hat p)
        = ip^{i-1}q^{i-1}\otimes\hat p\wedge\hat q, \\
  & \delta(p^iq^{i+1}\otimes\hat q)
        = (i+1)p^iq^i\otimes\hat p\wedge\hat q
  \end{align}
for all $i\geq0$. In writing this we have used the fact that the set
$\{p^iq^i:i\in\NN_0\}$ is a basis for the subalgebra~$\kk[pq]$ of~$\W_q$.
A standard calculation now shows that the cohomology of the
complex~\eqref{eq:w:comp:red} is of total dimension~$2$: in degree~$0$ it
is spanned by the cohomology class of~$1\in\kk[pq]$, and in degree~$1$ by
the cohomology class of~$q^{-1}\otimes\hat p$. The class of this last
cocycle thus freely spans $\HH^1(\W_q)$, and translating back we see that
it corresponds to the derivation $\partial:\W_q\to\W_q$ such that
$\partial(p)=q^{-1}$ and $\partial(q)=0$.

If $i\in\NN_0$ and $j\in\ZZ$, then one can easily check that
  \[ \label{eq:res:delta}
  \Res_0\left(p^iq^j\lact\frac{1}{t}\right) 
        = \begin{cases*}
          1 & if $i=j=0$; \\
          0 & in any other case.
          \end{cases*}
  \]
Let $\delta:\W_q\to\W_q$ be a derivation. There is a sequence of
derivations $(\delta_l)_{l\in\ZZ}$, all of which apart from a finite number
are zero, such that $\delta=\sum_{l\in\ZZ}\delta_l$ and for each $l\in\ZZ$
the derivation~$\delta_l:\W_q\to\W_q$ is homogeneous of degree~$l$. Since
$pq$ has degree~$0$, for each $l\in\ZZ\setminus0$ the element
$\delta_l(pq)$ is a linear combination of monomials of the form $p^iq^j$
with $i\in\NN_0$, $j\in\ZZ$ and $j-i\neq0$: in view
of~\eqref{eq:res:delta}, we then have that 
  \[
  \Res_0\left(\delta(pq)\lact\frac{1}{t}\right) 
        = \Res_0\left(\delta_0(pq)\lact\frac{1}{t}\right).
  \]
According to the decomposition~\eqref{eq:xx:desc}, the derivation~$\delta$
is inner if and only if the homogeneous derivation~$\delta_0$ is inner.
Now, if $\delta_0$ is inner, then there is a $u\in\W_q$ of degree~$0$ such
that $\delta_0=\ad(u)$ and, in particular,
$\delta_0(pq)=\ad(u)(pq)=[u,pq]=0$. Putting everything together, we see the
map
  \[ \label{eq:map:res}
  \delta\in\Der(\W_q)\mapsto\Res_0\left(\delta(pq)\lact\frac{1}{t}\right)\in\kk
  \]
vanishes on the subspace~$\InnDer(\W_q)$ of all inner derivations. Our
calculation of~$\HH^1(\W_q)$ implies that
$\Der(A)=\kk\partial\oplus\InnDer(\W_q)$:  as the map~\eqref{eq:map:res}
takes the value~$1$ on the derivation~$\partial$, we conclude that its
kernel is exactly~$\InnDer(\W_q)$.
\end{proof}

Going back to our algebra~$A$ is just a matter of changing coordinates.

\begin{Corollary}
There is a representation of~$A$ on the algebra~$\kk[t^{\pm1}]$ of Laurent
polynomials such that
  \[
  y \lact f(t) = t^N\frac{\d}{\d t}f(t),
  \qquad
  x \lact f(t) = tf(t)
  \]
for all~$f\in\kk[t^{\pm1}]$. If $\delta:A\to A$ is a derivation and
$\tilde\delta:A_x\to A_x$ is its extension to~$A_x$, then the
derivation~$\delta$ is $A_x$-inner if and only if
  \[
  \Res_0\left(\tilde\delta(x^{-N+1}y)\lact \frac{1}{t} \right) = 0.
  \]
\end{Corollary}

\begin{proof}
There is an injective algebra map $\iota:A\to\W$ such that $\iota(x)=q$ and
$\iota(y)=q^Np$ --- this is easy to check and is done in detail in
\cite{BLO:1}*{Section 3} --- and clearly this map extends to one on
localizations $\tilde\iota:A_x\to\W_q$ that is an isomorphism. The
representation of~$A$ on~$\kk[t^{\pm1}]$ that is mentioned in the corollary
is the restriction to~$A$ along~$\tilde\iota$ of the representation
of~$\W_q$ given in Proposition~\ref{prop:xinner:weyl}. Let $\delta:A\to A$
be a derivation of~$A$. It extends uniquely to a derivation
$\tilde\delta:A_x\to A_x$ and, conjugating by~$\tilde\iota$, gives a
derivation on~$\W_q$: the claim of the corollary thus follows immediately 
from the last part of Proposition~\ref{prop:xinner:weyl}.
\end{proof}
