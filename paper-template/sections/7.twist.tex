\section{Twisted cohomology}
\label{sect:twist}

In Proposition~\ref{prop:finite-subgroups} we described the finite
subgroups of~$\Aut(A)$ or, equivalently, the finite groups that act
faithfully on our algebra~$A$ by algebra automorphisms, and showed that
they are all cyclic. A natural next step is to describe the finite
dimensional Hopf algebras that act on~$A$ faithfully in the appropriate
sense. In that generality this is a problem that we do not know how to
approach, so we will settle for the much smaller problem of describing the
faithful actions of Taft Hopf algebras, motivated by the fact that these
Hopf algebras can be viewed as «quantum thickenings» of cyclic groups and
are therefore very close to them. We will do this in the next section.

We will recall the details about Taft algebras later. For now, let us
mention simply that an action of a Taft Hopf algebra on our algebra~$A$ is
determined by an algebra automorphism~$\phi:A\to A$ of finite order and a
$\phi$-twisted derivation of~$A$. We know the automorphisms very well
already, and our task in the present section will be to obtain a
description of the twisted derivations.

\bigskip

Let $\omega\in\kk$ be different from~$0$ and~$1$, and let
$\phi=\phi_{0,\omega}:A\to A$ be the algebra automorphism such that
$\phi(x)=\omega x$ and $\phi(y)=\omega^{N-1}y$; clearly $\phi$ has finite
order exactly when $\omega$ has finite order, but we do not need to impose
that restriction yet. We write ${}_\phi A$ for the $A$-bimodule which
coincides with $A$ as a right $A$-module and whose left action is such that
$a\lact x=\phi(a)x$ for all $a\in A$ and all $x\in{}_\phi A$. As the
automorphism~$\phi$ is homogeneous, the $A$-bimodule~${}_\phi A$ is a
graded one. 

We are interested in computing the space of \newterm{$\phi$-twisted
derivations}: such a map is a linear function $\partial:A\to A$ such that 
  \[ 
  \partial(ab) = \partial(a)\cdot b+\phi(a)\cdot\partial(b) 
  \] 
for all $a$ and~$b$ in~$A$. It is immediate that a $\phi$-twisted
derivation $A\to A$ is exactly the same thing as a derivation $A\to{}_\phi A$
into the bimodule~${}_\phi A$, and because of this we will write
$\Der(A,{}_\phi A)$ for the space of all $\phi$-twisted derivations and
conflate the two notions.

Now, if $m$ is an element of~$A$, then the map $\ad_\phi(m):a\in A\mapsto
ma-\phi(a)m\in A$ is an element of~$\Der(A,{}_\phi A)$: we say that such a
$\phi$-twisted derivation is an \newterm{inner} one. We can thus split the
calculation of~$\Der(A,{}_\Phi A)$ in two steps: compute the $\phi$-twisted
derivations modulo inner ones, and then glue back the inner ones into the
result. The first step amounts, precisely, to the calculation of the first
Hochschild cohomology space $\H^1(A,{}_\phi A)$, and this is what we will
do --- and just as in Section~\ref{sect:cohomology} we will do this by
first computing the spaces $\H^0(A,{}_\phi A)$ and~$\H^2(A,{}_\phi A)$.

\bigskip

The Hochschild cohomology~$\H^*(A,{}_\phi A)$ can be identified with the
cohomology of the complex~$\hom_{A^e}(P_*,{}_\phi A)$, with $P_*$ the
projective resolution described in Section~\ref{sect:cohomology}, and this
complex, in turn --- just as we built the complex~\eqref{eq:comp} in that
section --- is naturally isomorphic to the complex
  \[ \label{eq:comp:phi}
  \begin{tikzcd}
    & A \arrow[r, "\delta_0"]
    & A\otimes V^* \arrow[r, "\delta_1"]
    & A\otimes \Lambda^2V^* \arrow[r]
    & 0
  \end{tikzcd}
  \]
with differentials such that
  \begin{align}
  \delta_0(a) &= (\omega xa-ax)\otimes\hat x+(\omega^{N-1}ya-ay)\otimes\hat y, 
        \label{eq:d:phi:0} \\
  \delta_1(b\otimes\hat x+c\otimes\hat y) &=
    \left((\omega^{N-1}yb-by) + (cx-\omega xc) - \sum_{s+1+t=N}\omega^sx^sbx^t\right)
    \otimes\hat x\wedge\hat y
        \label{eq:d:phi:1} 
  \end{align}
for all choices of~$a$,~$b$ and~$c$ in~$A$. We remark that we are writing
«$A$» here and in~\eqref{eq:comp:phi} and not~«${}_\phi A$»: we will write
the twisted left action explicitly, as in the formulas above for the
differentials --- we hope this is less confusing than the alternative.

Just as when we dealt with the «untwisted» Hochschild cohomology, we start
our calculation with a lemma that tells us how some specific commutators
---~twisted commutators now\footnote{So we have not broken our promise not
to do any more commutation formulas!}~--- behave. In its statement and in
the rest of this section we will use the convention that if
$\mathit{snark}$ is an object related to~$A$ then we will write
$\widetilde{\smash{\mathit{snark}}}$ the `corresponding' object related to
the localization~$A_x$.

\begin{Lemma}\label{lemma:comm:phi}
The map $\alpha:a\in A\mapsto \omega xa-ax\in A$ is injective and has image
equal to~$xA$, while the map $\tilde\alpha:a\in A_x\mapsto \omega xa-ax\in
A_x$ is bijective.
\end{Lemma}

\begin{proof}
Let $a$ be a non-zero element of~$A$. There are $d\in\NN_0$ and
$a_0$,\dots,~$a_d\in\kk[x]$ such that $a_d\neq0$ and
$a=\sum_{i=0}^da_iy^i$, and we have that
  \[ \label{eq:alpha-a}
  \alpha(a) = \sum_{i=0}^da_i(\omega xy^i-y^ix)
        \equiv (\omega-1)a_dxy^d \mod F_{d-1}.
  \]
Since $\omega\neq1$, it follows immediately from this that
$\alpha(a)\neq0$, so that $\ker\alpha=0$.

Let us now show that $xa$ is in the image of~$\alpha$: we will proceed by
induction with respect to the integer~$d$. If $d=0$, then $a=a_0\in\kk[x]$
and we have that $xa=\alpha(a/(\omega-1))\in\img\alpha$. If instead
$d\geq1$, then we have, according to~\eqref{eq:alpha-a}, that
  \[ \label{eq:alpha-a:2}
  xa - \alpha\left(\frac{a_dy^d}{\omega-1}\right) \in F_{d-1}.
  \]
As $xA=Ax$, the image of~$\alpha$ is certainly contained in~$xA$, and
therefore the difference in~\eqref{eq:alpha-a:2} is both in~$F_{d-1}$ and
in~$xA$: the inductive hypothesis therefore implies that it is in the image
of~$\alpha$, and so is $xa$, as we wanted. With this we have proved the
claims of the lemma about the map~$\alpha$.

Let now $a$ be an element of~$A_x$. There is a non-negative integer
$l\in\NN_0$ such that $x^la\in A$. If $\tilde\alpha(a)=0$, then
$\alpha(x^la)=x^l\tilde\alpha(a)=0$ and what we have already proved implies
that $x^la=0$, so that $a=0$: this shows that the map~$\tilde\alpha$ is
injective. On the other hand, as $x^{l+1}a\in xA$, we already know there is
a $b\in A$ such that $\alpha(b)=x^{l+1}a$ and then
$\tilde\alpha(x^{-l-1}b)=a$: the map~$\tilde\alpha$ is also surjective.
\end{proof}

With this lemma we can now easily determine the twisted cohomology
$\HH^*(A,{}_\phi A)$ in degrees~$0$ and~$2$. Recall that when $N\geq2$ the
homogeneous components of~$A$ are finite-dimensional, so we can work with
Hilbert series in that case.

\begin{Proposition}
We have that $\HH^0(A,{}_\phi A) = 0$. If $N\geq2$, then the Hilbert series of
$\HH^2(A,{}_\phi A)$ is
  \[
  h_{\HH^2(A,{}_\phi A)} = 
        \begin{dcases*}
        % debatable: the first item in the frst line is centered here...
        \hfil t^{-N} & if $\omega^{N-1}\neq1$; \\[5pt]
        \frac{t^{-N}}{1-t^{N-1}} & if $\omega^{N-1}=1$.
        \end{dcases*}
  \]
If $N=1$, then $\HH^2(A,{}_\phi A)=0$.
\end{Proposition}

In the proof of the corollary we will exhibit concrete
representatives for all cohomology classes in~$\HH^2(A,{}_\phi A)$.

\begin{proof}
Since the map~$\alpha$ of the lemma is injective, it follows 
from the formula~\eqref{eq:d:phi:0} for the differential~$\delta_0$ that
the latter is itself injective and, therefore, that $\HH_0(A,{}_\phi A)=0$.

Let us suppose that $N\geq2$ and that $\omega^{N-1}\neq1$. If
$f\in\kk[y]$ and $v\in A$, then there is $c\in A$ such that
  \[
  \omega xc-cx 
        = \alpha(c) 
        = - (\omega^{N-1}-1)xv - \sum_{s+1+t=N}\omega^sx^sfx^t,
  \]
because the right hand side of this equality is in~$xA$ --- recall that
$xA=Ax$ --- and therefore
  \begin{align}
  \delta_1(f\otimes\hat x+c\otimes y) 
    &= \left(
        (\omega^{N-1}yf-fy) 
        + (cx-\omega xc) 
        - \sum_{s+1+t=N}\omega^sx^sfx^t
        \right)
       \otimes\hat x\wedge\hat y \\
    &= (\omega^{N-1}-1)(fy + xv) \otimes\hat x\wedge\hat y.
  \end{align}
This tells us that 
  \[ \label{eq:d1:phi}
  (\kk[y]y+xA)\otimes\hat x\wedge\hat y\subseteq\img\delta_1.
  \]
The subspace~$\kk[y]y+xA$ that appears here is the bilateral ideal~$I$ of~$A$
generated by~$x$ and~$y$. It is clear from the formula~\eqref{eq:d:phi:1}
and our hypothesis that $N\geq2$ that the image of~$\delta_1$ is contained
in~$I\otimes\hat x\wedge\hat y$, so in~\eqref{eq:d1:phi} we actually have
an equality. It follows from this that $\HH^2(A,{}_\phi A)$ can be
identified with $A/I\otimes\hat x\wedge\hat y$, so it has dimension~$1$,
and is generated by the class of the element $1\otimes\hat x\wedge\hat y$,
which is homogeneous of degree~$-N$. We thus have that
$h_{\HH^2(A,{}_\phi A)} = t^{-N}$.

Next, we suppose that $N\geq2$ and that $\omega^{N-1}=1$. According
to~\eqref{eq:d:phi:1}, if $b$ and~$c$ are in~$A$ we have now
  \[
  \delta_1(b\otimes\hat x+c\otimes\hat y)  =
    \left([y,b] + (cx-\omega xc) - \sum_{s+1+t=N}\omega^sx^sbx^t\right)
    \otimes\hat x\wedge\hat y.
  \]
The three terms in the sum wrapped with big parentheses are clearly in~$xA$,
so 
  \[ \label{eq:d1:phi:2}
  \img\delta_1 \subseteq xA\otimes\hat x\wedge\otimes\hat y.
  \]
As $\delta_1(c\otimes 1)=-\alpha(c)\otimes\hat x\wedge\hat y$ for each $c\in
A$, Lemma~\ref{lemma:comm:phi} implies us that in fact we have an equality
in~\eqref{eq:d1:phi:2} and therefore that $\HH^2(A,{}_\phi A)$ is in this
case isomorphic to $A/xA\otimes\hat x\wedge\hat y$. This is isomorphic as a
graded vector space to~$\kk[y]\otimes\hat x\wedge\hat y$, and therefore in
this case the Hilbert series we are after is
  \[
  h_{\HH^2(A,{}_\phi A)} = \frac{t^{-N}}{1-t^{N-1}}.
  \]

Finally, let us suppose that $N=1$. For all $b$ and $c$ in~$A$ we
have now that
  \[
  \delta_1(b\otimes\hat x+c\otimes\hat y) =
    \Bigl([y,b] + (cx-\omega xc) - b\Bigr)
    \otimes\hat x\wedge\hat y.
  \]
If $u$ is an element of~$A$, then there are $f\in\kk[y]$ and $v\in A$ such
that $u=f+xv$, and there is a $g\in A$ such that $gx-\omega
xg=-\alpha(g)=xv$, so that
  \(
  \delta_1(-f\otimes\hat x+g\otimes\hat y) = u\otimes\hat x\wedge\hat y
  \).
This tells us that the map~$\delta_1$ is in this case surjective, so that
we have $\HH^2(A,{}_\phi A)=0$.
\end{proof}

We can now compute $\HH^1(A,{}_\phi A)$ using the same strategy that
we used for the regular cohomology in Section~\ref{sect:cohomology}.

\begin{Proposition}\label{prop:h1:phi}
If $N\geq2$, then the Hilbert series of $\HH^1(A,{}_\phi A)$ is
  \[
  h_{\HH^1(A,{}_\phi A)}(t)
        = \begin{dcases*}
          0 & if $\omega^{N-1}\neq1$; \\ 
          \frac{1}{t(1-t^{N-1})} & if $\omega^{N-1}=1$.
          \end{dcases*}
  \]
If $N=1$, the $\HH^1(A,{}_\phi A)=0$.
\end{Proposition}

\begin{proof}
Let us suppose first that~$N\geq2$.
The Hilbert series of our algebra~$A$, as we saw before, is
  \(
  h_A(t) = (1-t)^{-1}(1-t^{N-1})^{-1}
  \),
and the Euler characteristic of the complex~\eqref{eq:comp:phi}~is
  \[
  h_A(t) - (t^{-1}+t^{-(N-1)})h_A(t)+t^{-N}h_A(t) = t^{-N},
  \]
so the invariance of the Euler characteristic when passing to cohomology
implies that
  \[
  h_{\HH^0(A,{}_\phi A)}(t)
  -h_{\HH^1(A,{}_\phi A)}(t)
  +h_{\HH^2(A,{}_\phi A)}(t)
  = t^{-N}.
  \]
The Hilbert series~$h_{\HH^1(A,{}_\phi A)}(t)$ can be computed immediately
from this and the preceding proposition, and turns out to be what is
described in the statement.

Let us now suppose that $N=1$ and let $\eta=b\otimes\hat x+c\otimes\hat y$,
with $b$ and~$c$ in~$A$, be a $1$-cocycle in the
complex~\eqref{eq:comp:phi}. If $f\in A$, then 
  \[
  \eta+\delta_0(f) = (b+\omega xf-xf)\otimes\hat x +
  (\text{something})\otimes\hat y,
  \]
and Lemma~\ref{lemma:comm:phi} implies that we can choose~$f$ so that
$b+\omega xf-xf$ is in~$\kk[y]$. This means that, up to replacing~$\eta$ by
a cohomologous $1$-cocycle, we can suppose that $b$ itself is in~$\kk[y]$.
In that case we have, according to~\eqref{eq:d:phi:1}, that
  \[
  0 = \delta_1(\eta) = 
    (cx-\omega xc - b) \otimes\hat x\wedge\hat y.
  \]
Since $cx-\omega xc\in xA$ and $b\in\kk[y]$, this implies that $cx-\omega
xc=b=0$ and the injectivity of the map~$\alpha$ that, in fact, also $c=0$.
This shows that every $1$-cocycle in our complex~\eqref{eq:comp:phi} is
cohomologous to zero and therefore that $\HH^1(A,{}_\phi A)=0$.
\end{proof}

The proposition we have just proved shows that all $\phi$-twisted derivations
$\Der(A,{}_\phi A)$ are in fact inner except in the special case in which
$N\geq2$ and $\omega^{N-1}=1$. In this special case we can make the
following observations. We will work with the localization~$A_x$ of~$A$ at
the powers of the normal element~$x$, in which we have the following
commutation rule:
  \[
  yx^{-1}-x^{-1}y = -x^{N-2}.
  \]
The canonical map $A\to A_x$ is injective, so we view $A$ as a subalgebra
of~$A_x$.
The automorphism $\phi:A\to A$ such that $\phi(x)=\omega x$ and
$\phi(y)=\omega^{N-1}y$ clearly extends to an automorphism
$\tilde\phi:A_x\to A_x$, and every $\phi$-twisted derivation
$\partial:A\to A$ extends uniquely to a $\tilde\phi$-twisted derivation
$\tilde\partial:A_x\to A_x$ that has $\tilde\partial(x^{-1}) =
-\omega^{-1}x^{-1}\partial(x)x^{-1}$.

The following technical result will allow us to construct twisted
derivations.

\begin{Proposition}\label{prop:nin}
Suppose that $N\geq2$ and $\omega^{N-1}=1$.
\begin{thmlist}

\item There is a $\phi$-twisted derivation $\partial_0^\phi:A\to A$ such
that
  \[
  \partial_0^\phi(x) = 1-\omega, 
  \qquad
  \partial_0^\phi(y) = x^{N-2},
  \]
and it is homogeneous of degree~$-1$ and not inner. The inner
$\tilde\phi$-twisted derivation $\ad_{\tilde\phi}(x^{-1}):A_x\to A_x$
of~$A_x$ corresponding to~$x^{-1}$ preserves the subalgebra~$A$ and, in
fact, the map $\partial_0^\phi$ is the restriction
of~$\ad_{\tilde\phi}(x^{-1})$ to~$A$.

\item There is a derivation $\partial_{N-1}:A\to A$ such that
  \[
  \partial_{N-1}(x) = xy, 
  \qquad
  \partial_{N-1}(y) 
    = \hskip-1em\sum_{s+2+t=N}(s+1)x^{s+1}yx^t 
         +  \frac{N-1}{2}
            \left(
              y^{2}
              -N
              x^{N-1}y
            \right).
  \]
This derivation commutes with~$\phi:A\to A$.

\item If $\partial:A\to{}_\phi A$ is a twisted derivation that is
homogeneous of some degree~$l$, then the map
  \[
  \partial_{N-1}\circ\partial-\partial\circ\partial_{N-1}:A\to A
  \]
is also a $\phi$-twisted derivation that is homogeneous, now of
degree~$l+N-1$. There is therefore a map  
  \[
  \ad(\partial_{N-1}):
        \partial
        \in\Der(A,{}_\phi A)
        \longmapsto
        \partial_{N-1}\circ\partial-\partial\circ\partial_{N-1}
        \in\Der(A,{}_\phi A)
  \]
that is homogeneous of degree~$N-1$.

\item If $u$ is an element of~$A_x$ such that the inner
$\tilde\phi$-derivation $\ad_{\tilde\phi}(u):A_x\to A_x$ preserves the
subalgebra~$A$, so that the restriction
$\ad_{\tilde\phi}(u)|_A:A\to A$ is an element of~$\Der(A,{}_\phi A)$, then
the map $\ad_{\tilde\phi}(\tilde\partial_{N-1}(u)):A_x\to A_x$ also
preserves~$A$ and
  \[ \label{eq:adad}
  \ad(\partial_{N-1})\Bigl(\ad_{\tilde\phi}(u)|_A\Bigr)
        = \ad_{\tilde\phi}(\tilde\partial_{N-1}(u))|_A.
  \]
Here $\tilde\partial_{N-1}(u)$ denotes the value at~$u$ of the
extension~$\tilde\partial_{N-1}$ of the derivation~$\partial_{N-1}$
of~\thmitem{2} to~$A_x$.

\end{thmlist}
\end{Proposition}

\begin{proof}
\thmitem{1} Let $\kk\gen{X,Y}$ be the free algebra on two generators~$X$
and~$Y$. There is an automorphism $\hat\phi:\kk\gen{X,Y}\to\kk\gen{X,Y}$
such that $\hat\phi(X)=\omega X$ and
$\hat\phi(Y)=Y$, and it preserves the bilateral ideal
$I=(YX-XY-X^N)$, since $\omega^{N-1}=1$. The quotient $\kk\gen{X,Y}/I$ is
of course isomorphic to~$A$, and we may thus view $A$ as a
$\kk\gen{X,Y}$-bimodule. Since the algebra $\kk\gen{X,Y}$ is free, there is
a unique derivation $\hat\partial:\kk\gen{X,Y}\to{}_{\hat\phi} A$ such that
$\hat\partial(X)=1-\omega$ and $\hat\partial(Y)=x^{N-2}$. This derivation
vanishes on the ideal~$I$: to check this, it is sufficient to show that it
vanishes on the generator $YX-XY-X^N$ of that ideal, and that is easy. One
need only keep in mind that $\hat\partial$ is a \emph{$\hat\phi$-twisted}
derivation, so that
  \[
  \hat\partial(X^N)
        = (1+\omega+\cdots+\omega^{N-1})(1-\omega)x^{N-1}
        = (1-\omega)x^{N-1}.
  \]
It follows from this that $\hat\partial$ induces a derivation
$\partial_0^\phi:A\to{}_\phi A$ such that
$\partial_0^\phi(x)=1-\omega$ and $\partial_0^\phi(y)=x^{N-2}$. It
is manifestly homogeneous of degree~$-1$, and it is not inner because there
are no elements of negative degree in~$A$.

Now, we certainly have an inner $\tilde\phi$-twisted derivation
$\ad_{\tilde\phi}(x^{-1}):A_x\to A_x$, and we can compute that
  \begin{align}
  \ad_{\tilde\phi}(x^{-1})(x)
       &= x^{-1}x-\tilde\phi(x)x^{-1}
        = 1-\omega, \\
  \ad_{\tilde\phi}(x^{-1})(y)
       &= x^{-1}y-\tilde\phi(y)x^{-1}
        = x^{N-2}.
  \end{align}
Since $\tilde\phi(A)=A$, the facts that $x$ and~$y$ generate~$A$ and
that $\ad_{\tilde\phi}(x^{-1})$ is a $\tilde\phi$-twisted derivation imply
together that $\ad_{\tilde\phi}(x^{-1})$ maps the subalgebra~$A$ into
itself. Moreover, its restriction to~$A$ is a $\phi$-twisted derivation
that coincides with $\partial^\phi_0$ at~$x$ and at~$y$, so
we have that $\ad_{\tilde\phi}(x^{-1})|_A=\partial^\phi_0$.

\thmitem{2} The derivation~$\partial_{N-1}$ described here is simply the
one given by Proposition~\ref{prop:hh1-high} when $l=N-1$, which we know to be
homogeneous of degree~$N-1$ and not inner. To make this explicit we have
used the fact that $\Phi_2 = \frac{1}{2}(y^{2} -N x^{N-1}y)$. A simple
direct calculation shows that $\partial_{N-1}$ commutes with~$\phi$.

\thmitem{3} This follows from an easy direct calculation. They key fact
that makes things work is that the derivation $\partial_{N-1}:A\to A$
commutes with the automorphism $\phi:A\to A$.

\thmitem{4} A direct calculation shows that we have an equality of
maps~$A_x\to A_x$
  \[
  \ad(\tilde\partial_{N-1})\Bigl(\ad_{\tilde\phi}(u)\Bigr)
        = \ad_{\tilde\phi}\Bigl(\tilde\partial_{N-1}(u)\Bigr)
  \]
The left hand side preserves~$A$ because $\ad_{\tilde\phi}(u)$
and~$\tilde\partial_{N-1}$ do, so the right hand side also preserves it.
The equality~\eqref{eq:adad} that appears in the statement of the
proposition is now immediate.
\end{proof}

Combining the different parts of the proposition we have just proved we can
construct $\phi$-twisted derivations $A\to{}_\phi A$ of all degrees of the
form $-1+l(N-1)$ with $l\in\NN_0$.

\begin{Corollary}\label{coro:partial:phi}
Suppose that $N\geq2$ and $\omega^{N-1}=1$, and let $\phi:A\to A$ be the
automorphism such that $\phi(x)=\omega x$ and $\phi(y)=y$.
Let $\tilde\phi:A_x\to A_x$ and $\tilde\partial_{N-1}:A_x\to A_x$ be the
extensions of~$\phi$ and of the $\phi$-twisted
derivation~$\partial_{N-1}:A\to A$ of Proposition~\ref{prop:nin}. 
For each $l\in\NN_0$ the $\tilde\phi$-twisted derivation
  \[
  \ad_{\tilde\phi}\bigl(\tilde\partial_{N-1}^l(x^{-1})\bigr):A_x\to A_x
  \]
preserves~$A$ and its restriction to~$A$ is a map 
  \[
  \partial^\phi_l:A\to A
  \]
that is a homogeneous $\phi$-twisted derivation of degree~$-1+l(N-1)$ and
that coincides with the map
  \[
  \ad(\partial_{N-1})^l(\partial_0^\phi) \in \Der(A,{}_\phi A).
  \]
\end{Corollary}

\begin{proof}
This can be proved immediately by induction starting with~\thmitem{1} of
the proposition and using part~\thmitem{4} for the inductive step.
\end{proof}

\begin{Remark}\label{rem:general-nonsense}
There is a general-nonsense way of looking at our last few results. We will
explain it, as it helps to make sense of them.

Suppose that $N\geq2$ and that $\omega^{N-1}=1$, and let $\nu$ be the order
of~$\omega$ in~$\kk^\times$, which is a divisor of~$N-1$. General
considerations imply that the automorphism~$\phi$ acts in a canonical way
on the Hochschild cohomology~$\HH^*(A)$ of~$A$, the invariant
subspace~$\HH^*(A)^\phi$ is a sub-Gerstenhaber algebra of~$\HH^*(A)$, and
there is a «Gerstenhaber module structure»
  \[
  \HH^*(A)^\phi \otimes \HH^*(A,{}_\phi A)
  \to
  \HH^*(A,{}_\phi A)
  \]
that in particular restricts to a Lie module structure
  \[ \label{eq:hhhh}
  \HH^1(A)^\phi \otimes \HH^1(A,{}_\phi A)
  \to
  \HH^1(A,{}_\phi A),
  \]
which itself is induced by the map
  \[ \label{eq:derder}
  f\otimes g\in\Der(A)\otimes\Der(A,{}_\phi A)
  \mapsto
  \ad(f)(g) \in\Der(A,{}_\phi A),
  \]
using the notation of part~\thmitem{3} of Proposition~\ref{prop:nin}. 

If we view~$\HH^1(A)$ as $\Der(A)/\InnDer(A)$ then the action of~$\phi$
on~$\HH^1(A)$ is induced by the map  
  \[
  \phi^\sharp:d\in\Der(A)\mapsto\phi\circ d\circ\phi^{-1}\in\Der(A),
  \]
and a direct calculation proves that if $d:A\to A$ is a homogeneous
derivation of degree~$l$, then $\phi^\sharp(d)=\omega^l d$. It follows from
this that the invariant subspace~$\HH^1(A)^\phi$ is spanned by the classes
of the derivations of the form~$L_{j\nu}$, with~$j\in\ZZ$ such that
$j\geq-1$, and $\partial_0$, using the notation of
Section~\ref{sect:explicit}. On the other hand,
Proposition~\ref{prop:h1:phi} tells us that the only homogeneous components
of~$\HH^1(A,{}_\phi A)$ that are non-zero are those of degrees~$l$ such
that $l\geq-1$ and $l\equiv-1\mod N-1$: as the action~\eqref{eq:hhhh} is
homogeneous, this implies that the only $L_l$ in~$\HH^1(A)^\phi$ that can
act non-trivially on~$\HH^1(A,{}_\phi A)$ are those with those with
$l\equiv0\mod N-1$. What Corollary~\ref{coro:partial:phi} does is,
essentially, to describe the action of~$L_{N-1}$, the simplest of those:
in part~\thmitem{1} of Proposition~\ref{prop:nin} we exhibited an
element~$\partial_0^\phi$ in~$\Der(A,{}_\phi A)$, and in
Corollary~\ref{coro:partial:phi} we produced many others by using the
action~\eqref{eq:derder}. We will see below that in this way we 
obtain a basis for~$\HH^1(A,{}_\phi A)$.
\end{Remark}

We are just a hop, skip and jump away from the description of
$\H^1(A,{}_\phi A)$ in the exceptional case in which it is non-zero ---
apart, alas, for one exception.

\begin{Proposition}\label{prop:h1:phi:base}
Suppose that $N\geq4$ and that $\omega^{N-1}=1$, and let $\phi:A\to A$ be
the automorphism of~$A$ such that $\phi(x)=\omega x$ and $\phi(y)=y$. The
cohomology classes of the $\phi$-twisted derivations $\partial^\phi_l$ with
$l\in\NN_0$ that we described in Corollary~\ref{coro:partial:phi} freely
span the vector space~$\H^1(A,{}_\phi A)$.
\end{Proposition}

This excludes the cases in which $2\leq N\leq 3$. Now if $N=2$ then the
hypothesis on~$\omega$ is that $\omega=1$ and, since we are assuming
throughout that $\omega\neq1$, this case cannot actually occur. The case in
which $N=3$, on the other hand, is a real possibility that is excluded in
this proposition --- and for good reason, as we will see below.

\begin{proof}
It is enough that we show that the cohomology classes of those
$\phi$-twisted derivations are linearly independent in~$\H^1(A,{}_\phi A)$,
because if that is the case then they span the whole space: this follows
immediately from our calculation of the Hilbert series of~$\H^1(A,{}_\phi
A)$ in Proposition~\ref{prop:h1:phi} and the fact that for all $l\in\NN_0$
the twisted derivation $\partial^\phi_l$ has degree~$-1+l(N-1)$.

Let us suppose that the cohomology classes of the twisted derivations in
the statement are linearly dependent. There are then
an integer $d\geq0$ and scalars $a_0$,~\dots,~$a_d$ in~$\kk$
such that $a_d\neq0$ and the $\phi$-twisted derivation 
  \[
  \delta\coloneqq\sum_{i=0}^da_i\partial_i^\phi:A\to A
  \]
is inner, so that there is an element~$u$ in~$A$ such that
$\delta=\ad_\phi(u)$. We let $P$ be the polynomial
$\sum_{i=0}^da_iT\in\kk[T]$ and consider the $\tilde\phi$-twisted
derivation
  \[ \label{eq:urgh}
  \ad_{\tilde\phi}\bigl(P(\tilde\partial_{N-1})(x^{-1})\bigr):A_x\to A_x
  \]
It follows from Corollary~\ref{coro:partial:phi} that this map
preserves~$A$ and its restriction to~$A$ is precisely~$\delta$. In
particular, the map in~\eqref{eq:urgh} and~$\ad_{\tilde\phi}(u):A_x\to A_x$
take the same value on~$x$ and~$y$ and therefore, since they are
$\tilde\phi$-twisted derivations, they are in fact equal --- in other
words, we have that
  \[ \label{eq:urgh:1}
  \ad_{\tilde\phi}\Bigl(P(\tilde\partial_{N-1})(x^{-1})-u\Bigr) = 0.
  \]
We know form Lemma~\ref{lemma:comm:phi} that the map $\tilde\alpha:a\in
A_x\mapsto\omega xa-xa\in A_x$ is injective, and as a consequence of that

\begin{claim}\label{eq:urgh:2}
    If $v\in A_x$ and the map $\ad_{\tilde\phi}(v):A_x\to A_x$
  vanishes at~$x$, then $v=0$.
\end{claim}

From this and the equality~\eqref{eq:urgh:1} we can conclude that
  \[
  P(\tilde\partial_{N-1})(x^{-1}) = u \in A.
  \]

Let us next check that for all $l\in\NN_0$ we have that
  \[ \label{eq:urgh:3}
  \tilde\partial_{N-1}^l(x^{-1})
        \equiv \prod_{i=0}^{l-1}\left(\frac{t(N-1)}{2}-1\right)\cdot x^{-1}y^{l} 
        \mod \tilde F_{l-1}.
  \]
This is clear when $l=0$, and when $l=1$ we have that
  \[
  \tilde\partial_{N-1}(x^{-1})
        = -x^{-1}\tilde\partial_{N-1}(x)x^{-1}
        = -yx^{-1}
        \equiv -x^{-1}y\mod \tilde F_0,
  \]
in accordance with the formula we want to prove. The general case follows
by induction and the observation that for all $l\in\NN_0$ we have
  \begin{align}
  \MoveEqLeft
  \tilde\partial_{N-1}(x^{-1}y^l)
        = \tilde\partial_{N-1}(x^{-1})y^l + x^{-1}\tilde\partial_{N-1}(y^l) 
        = -yx^{-1}y^l + x^{-1}\sum_{i=0}^{l-1}y^i\tilde\partial_{N-1}(y)y^{l-1-i} \\
       &= -yx^{-1}y^l + x^{-1}\sum_{i=0}^{l-1}y^i
                \left[
                \sum_{s+2+t=N}(s+1)x^{s+1}yx^t 
                         +  \frac{N-1}{2}
                            \left(
                              y^{2}
                              -N
                              x^{N-1}y
                            \right)
               \right]
               y^{l-1-i}
          \\
       &\equiv \left(\frac{l(N-1)}{2}-1\right)
                x^{-1}
               y^{l+1} \mod \tilde F_l.
  \end{align}

Now, using the congruence~\eqref{eq:urgh:3} we have just proved we see that
  \[
  P(\tilde\partial_{N-1})(x^{-1})
    \equiv \underbrace{a_d \prod_{i=0}^{d-1}\left(\frac{t(N-1)}{2}-1\right)}
        {}\cdot x^{-1}y^{d} 
    \mod \tilde F_{d-1}
  \]
As $N\geq4$, the scalar marked with a brace is non-zero. If we call it
$\gamma$ for brevity, then what we have is that 
  \[
  A \ni P(\tilde\partial_{N-1})(x^{-1}) = \gamma\cdot x^{-1}y^d + w
  \]
for some $w\in\tilde F_d$, and this is absurd. This contradiction proves
the proposition.
\end{proof}

When $N=3$, in the situation of Proposition~\ref{prop:h1:phi:base} we can
compute that 
  \[
  \tilde\partial_2(x)=xy, 
  \qquad
  \tilde\partial_2(x^{-1})=-x^{-1}y+x,
  \qquad
  \tilde\partial_2(y)=x^4+y^2
  \]
and using that that
  \[
  \tilde\partial_2^2(x^{-1}) = -x^3 \in A.
  \]
This tells us that the $\phi$-twisted derivation $\partial^\phi_2:A\to A$
is inner, and implies that in fact the $\phi$-twisted derivation
$\partial^\phi_l:A\to A$ is inner whenever $l$ is an integer such that
$l\geq2$. The conclusion of Proposition~\ref{prop:h1:phi:base} is therefore
very false when $N=3$. We can fix it as follows:

\begin{Proposition}
Suppose that $N=3$ and that $\omega^{N-1}=1$, so that $\omega=-1$, and let
$\phi:A\to A$ be the automorphism of~$A$ such that $\phi(x)=-x$ and
$\phi(y)=y$. The $\tilde\phi$-twisted
derivations of~$A_x$  
  \[
  \ad_{\tilde\phi}(x^{-1}),
  \qquad\quad
  \ad_{\tilde\phi}(\tilde\partial_2(x^{-1})),
  \qquad\quad
  \ad_{\tilde\phi}(\tilde\partial_2^l(x^{-1}y^2)), \quad l\geq0
  \]
preserve the subalgebra~$A$ and the cohomology classes of their
restrictions to~$A$ freely span the vector space~$\H^1(A,{}_\phi A)$.
\end{Proposition}

\begin{proof}
This can be proved in essentially the same way as the previous proposition,
and we therefore omit the details.
\end{proof}

With this we have concluded the calculation of the twisted Hochschild
cohomology $\H^*(A,{}_\phi A)$. One nice observation we can immediately
make with the result at hand is:

\begin{Proposition}\label{prop:der:xinner}
Suppose that $N\geq1$ and that $\omega\in\kk^\times\setminus\{1\}$, and let
$\phi:A\to A$ be the automorphism of~$A$ such that $\phi(x)=\omega x$ and
$\phi(y)=\omega^{N-1}y$. Every $\phi$-twisted derivation $A\to{}_\phi A$ is
the restriction to~$A$ of an inner $\tilde\phi$-twisted derivation
$A_x\to{}_{\tilde\phi} A_x$ that preserves~$A$. \qed
\end{Proposition}

By analogy with the notion of X-inner automorphisms of Har\v{c}enko
\citelist{\cite{Harchenko:1}\cite{Harchenko:2}}, we can rephrase this
proposition saying that all $\phi$-twisted derivations of~$A$ are X-inner.

\bigskip

When $N=0$, so that the algebra $A$ is the first Weyl
algebra, the calculation of the twisted Hochschild cohomology
$\H^*(A,{}_\phi A)$ for an automorphism $\phi:A\to A$ such that
$\phi(x)=\omega x$ and $\phi(y)=\omega^{-1}y$ for some
$\omega\in\kk^\times$ was done in~\cite{AFLS} by Jacques Alev, Thierry
Lambre, Marco Farinati and Andrea Solotar: the end result is that
  \[
  \dim\H^p(A,{}_\phi A) \cong 
    \begin{cases*}
    1 & if $p=0$ and $\omega=1$, or if $p=2$ and $\omega\neq1$; \\
    0 & in any other case.
    \end{cases*}
  \]
This is rather different from what we have found above for $N\geq1$. It
should be noticed that while the automorphism group of the first Weyl
algebra is considerably larger than that of our algebras $A$ with $N\geq1$,
but every one of its elements of finite order is conjugated to one of
the form of the automorphism~$\phi$ described here. On the other hand,
there are non-cyclic finite groups of automorphisms of the first Weyl
algebra.
