\begin{problem}[5]
  Given a countable sequence of sets $\set{A_i}_{i \in \omega}$,
  define the \emph{infinite join}
  $A = \bigjoin_{i \in \omega} A_i$ by
  \[
    A = \set{\vector{i, n} \mid n \in A_i}.  
  \]

  Prove that there are sequences $\set{A_i}_{i \in \omega}$ and $\set{B_i}_{i \in \omega}$
  such that $A_i \equiv_T B_i$ for all $i$ but $A \not\equiv_T B$.
  In other words, this operation is defined on sets, but not on
  degrees (unlike the finite joins). \\
  \emph{
    Hint: make $A$ computable but $B$ not computable.
  }

  \begin{answer}
    For each $i \in \omega$, let
    \[ A_i = \set{i} \] and
    % \[ B_i = \set{ e \given \varphi_{e, i}(e) \converges}. \]
    \[ B_i = \set{ e \quad \given \quad e = i \text{ and } \varphi_i(i) \converges } \]

    First, we show that each $A_i \equiv_T B_i$.
    \begin{enumarabic}
      \item $A_i \leq_T B_i$.
        Note that each $A_i$ is computable,
        since it is a singleton set containing $i$.
        Thus, for any $A_i$ and $B_i$, we can define $\Phi_e^{B_i}$
        to be an oracle machine that,
        given $n$, ignores $B_i$ and computes $\chi_A$
        as follows:
        % returns $1$ if $n = i$ or $0$ otherwise.
        \[ 
          \chi_{A_i}(n) = \begin{cases}
            1 & \text{if } n = i, \\
            0 & \text{otherwise}.
          \end{cases}
        \]

      \item $B_i \leq_T A_i$.
        Define $\Phi_e^{A_i}$ to be an oracle machine that,
        given $n$, first checks if $n \in A_i$ then simulates
        $\varphi_n(n)$ and checks if it converges.
        \[
          \chi_{B_i}(n) = \begin{cases}
            1 & \text{if } n \in A_i \land \varphi_n(n) \converges \\
            0 & \text{otherwise}.
          \end{cases}
        \]
    \end{enumarabic}

    Next, we show that $A \not\equiv_T B$.

    Let \[ A = \set{\vector{i, n} \mid n \in A_i} = \set{\vector{i, i} \given i \in \omega} \]
    and \[ B = \set{\vector{i, n} \mid n \in B_i} = \set{\vector{i, i} \given \varphi_i(i) \converges}
    = \set{\vector{i, i} \given i \in K}. \]

    $A$ is computable since it is a set of all pairs of the form $\vector{i, i}$.
    Given a pair $\vector{i, n}$, we can computably check if $n = i$.
    However, $B$ is not computable since it is the set of all pairs $\vector{i, i}$ such that
    $\varphi_i(i)$ converges, which is equivalent to the halting problem.
    Thus, $A \not\equiv_T B$.
  \end{answer}
\end{problem}
