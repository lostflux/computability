\begin{problem}
  Does Lemma 1 from the Noncomputability lecture hold if we  remove
  the word ``total''?  That is, $f$ partial computable
  if and only if its graph is a computable set?
  Justify your answer.
  \begin{answer}
    \begin{lemma}
      The graph of a (partial) function $f$, $\graph{f}$, is the set
      $\set{\vector{n, k} \mid f(n) = k}$.
      If $f$ is total, $\graph{f}$ is computable if and only if $f$ is computable.
    \end{lemma}

    \newterm{
      No, the lemma does not hold.
    }

    \subsubsection*{$\impliedby$ \cmark}
    Assume that $\graph{f}$ is computable. Consider the following turing machine:

    \step
    \begin{algorithm}[H]
      \SetAlgorithmName{$\mathsf{T} \mathsf{M}$}{}{}
      \caption{Compute $f(n)$}
      \For{$k = 1, 2, 3, \ldots$}{
        \If{$\vector{n, k} \in \graph{f}$}{
          \textbf{output} $k$
        }
      }
    \end{algorithm}

    \step
    For $n \in \dom{f}$, the turing machine will eventually check if
    $\vector{n, f(n)} \in \graph{f}$ and output $k$.
    However, for $n \not\in \dom{f}$, the turing machine will never halt.
    Thus, $f$ is partial computable.

    \step
    \subsubsection*{$\implies$ \xmark}
    However, $f$ being partial computable \emph{does not} imply that
    $\graph{f}$ is computable.
    Suppose we have a turing machine $M$ simulating $f$.
    Then the approach for determining if $\vector{n, k} \in \graph{f}$
    would be to run $M$ on input $n$, obtain the output $k_2$,
    and compare if $k = k_2$. But given $f$ is partial, $M$ may not halt
    for some inputs (specifically $\N \setminus \dom{f}$, which is a nonempty set).
    Thus, we cannot compute $\chi_{\graph{f}}$, so $\graph{f}$ is not
    necessarily computable.
  \end{answer}
\end{problem}
