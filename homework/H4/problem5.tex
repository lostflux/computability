\begin{problem}[5]
  Show that if $P$ is productive then $P$ contains an infinite c.e. set.

  \begin{definition}
    A set $P$ is productive if it has a productive function ---
    a (partial) computable function $\psi$ such that, whenever
    $W_e \subseteq P$, $\psi(e) \converges$ and $\psi(e) \in P \setminus W_e$.
    That is, a productive function is able to
    produce a witness to the fact that $P \neq W_e$ whenever $W_e \subseteq P$.
    Then it is immediate that productive sets are not c.e., so finding a c.e. set whose complement
    is productive will necessarily be a noncomputable c.e. set.
  \end{definition}

  \begin{answer}
    Let $P$ be a productive function, with $\psi$ as its productive function.
    We shall enumerate an infinite set
    $Y = \set{y_0, y_1, y_2, \ldots } \subseteq P$ as follows:

    \begin{enumarabic}
      \item Take $e_0$ to be the smallest index with $W_{e_0} = \emptyset \subseteq P$.
        Then $\psi(e_0) \converges = y_0$ for some $y_0 \in P \setminus \emptyset = P$.

      \item Inductively, for $n \geq 1$,
        select $e_n$ to be the smallest index
        such that $W_{e_n} = \set{y_0, y_1, \ldots, y_{n-1}} \subseteq Y$.
        
        Then $\psi(e_n) \converges = y_n$ for some
        $y_n \in P \setminus \set{y_0, y_1, \ldots, y_{n-1}}$.
        Particularly, $y_n \neq y_i$ for any $i < n$.
    \end{enumarabic}

    \step
    To show that $Y$ is c.e., we need to show that $Y = W_e$ for some $e$.
    Consider the following turing machine $\E_Y$ that enumerates $Y$:

    \step
    \begin{algorithm}[H]\label{tm:5.1}
      \SetAlgorithmName{$\mathsf{T} \mathsf{M}$}{}{}
      \caption{$\varphi_e : Y \to \N$}
      On input $n$: \\
      Initialize $Y \gets \emptyset$ \\
      \For{$i = 0, 1, 2, \ldots$}{
        Select $e_i$ as above \\
        Compute $y_i = \psi(e_i)$ \\
        \If{$y_i = n$}{
          \textbf{output} $1$
        }
      }
    \end{algorithm}

    On input $n$, the turing machine
    halts and outputs $1$ if $n \in Y$.
    If $n \not\in Y$, the turing machine will continue to loop, ad infinitum.

    Therefore, $\dom{\varphi_e} = Y$, so $Y$ is c.e.
  \end{answer}
\end{problem}
