\begin{problem}[2]
  Recall that $W_e$ is $\dom{\varphi_e}$, and that $X$ is c.e. if
  $X = W_e$ for some $e$. Show that it is equivalent to define
  the c.e. sets as those that are either finite or the range
  of a total, computable, injective function
  $f : \N \to \N$.

  \begin{answer}
    We can show the equivalence of the two definitions by proving that
    $X$ is c.e. if and only if $X$ is finite or the range of a
    total, computable, injective function.

    \subsection*{$\implies$}
      \begin{proof}
        Let $X$ be c.e. so that $X = W_e$ for some $e$.

        \step
        If $X$ is finite, then the condition is satisfied.

        \step
        Suppose $X$ is infinite.
        We shall construct a total, computable, injective function
        $f : \N \to \N$ whose range is $X$.
        Precisely, we construct $f$ such that, given input $n$,
        $f$ returns the $(n+1)-$th member of $X$ in the sequence
        \blue{$0, 1, 2, 3, \ldots$}

        \step
        \begin{algorithm}[H]
          \SetAlgorithmName{$\mathsf{TM}$}{}{}
          \caption{$\varphi_k$: compute $f$}
          On input $n$: \\
          $S \gets \emptyset$ \\
          \For{$i = 0, 1, 2, 3, \ldots$}{
            \If{$\varphi_e(i) \converges$}{
              $S \gets S \cup \set{\varphi_e(i)}$ \\
              \If{$\abs{S} = n+1$}{
                \textbf{output} $i$
              }
            }
          }
        \end{algorithm}
      \end{proof}

      Since $X$ is infinite, $\varphi_k(n)$
      will eventually halt and output the $(n+1)-$th element of $X$
      for every $n \in \N$.
      Therefore, $f$ is total.

      Similarly, since $\varphi_k$ only outputs an element
      of $X$ with the condition that $\varphi_e(i)$ converges,
      every element outputed by $f$ is a member of
      $\dom{\varphi_e} = W_e = X$. Therefore, $\range{f} = X$.

      Finally, $f$ is injective since it outputs a unique element
      for each $n \in \N$ since the set admits no duplicates.

    \subsubsection*{$\impliedby$}
    
    \begin{proof}
      For the two cases:

      \step
      \item \textbf{Suppose $X$ is finite}, then $X = \set{x_1, x_2, \ldots, x_n}$
      for some $n \in \N$.
      Define $g: \N \to \N$ by:
      \[ g(i) = \begin{cases}
        k & \text{if $i = x_k$ for some $k < n$ and $x_k \in X$} \\
        \diverges & \text{otherwise.}
      \end{cases} \]

      Let $e$ be the code of the turing machine that computes $g$.
      Then $\varphi_e(i) \converges$ \emph{if and only if}
      $i \in X$. Therefore, $\dom{\varphi_e} = W_e = X$,
      so $X$ is c.e.

      \step
      \textbf{Suppose $X$ is infinite and it is the range of a
      total, computable, injective function $f: \N \to \N$.}

      We show that $f$ is the domain of some function $g : \N \to \N$.
      Define $g$ as follows:
      \[ g(n) = \begin{cases}
        i & \text{if $f(i) = n$} \\
        \diverges & \text{otherwise.}
      \end{cases} \]

      \step
      \begin{algorithm}[H]
        \SetAlgorithmName{$\mathsf{TM}$}{}{}
        \caption{$g : \N \to \N$}
        On input $n$: \\
        \For{$i = 0, 1, 2, 3, \ldots$}{
          \If{$f(i) \converges = n$}{
            \textbf{output} $i$
          }
        }
      \end{algorithm}

      \step
      Let $e$ be the code of the turing machine that computes $g$
      as outlined above.
      Then $\varphi_e(n) \converges$ \emph{if and only if}
      $n = f(i)$ for some $i$, so $n \in \range{f} = X$.
      Therefore, $\range{f} = \dom{\varphi_e} = W_e = X$,
      and $X$ is c.e.
    \end{proof}
  \end{answer}
\end{problem}
