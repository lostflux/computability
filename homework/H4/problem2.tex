\begin{problem}
  Recall that $W_e$ is $\dom{\varphi_e}$, and that $X$ is c.e. if
  $X = W_e$ for some $e$. Show that it is equivalent to define
  the c.e. sets as those that are either finite or the range
  of a total, computable, injective function
  $f : \N \to \N$.

  \begin{answer}
    We can show the equivalence of the two definitions by proving that
    one is true if and only if the other is true.
    \subsubsection*{\emph{
        $X$ is c.e. $\implies$ $X$ is finite or the range of a
        total, computable, injective function.
      }
    }

      \begin{proof}
        Let $X$ be c.e. so that $X = W_e$ for some $e$.

        Then, we can define a total, computable, injective function
        $f : \N \to \N$ as follows:
        \[
          f(n) = \begin{cases}
            \varphi_e(n) & \text{if $\varphi_e(n) \converges$} \\
            0 & \text{otherwise}
          \end{cases}
        \]

        Then there exists a turing machine $\E_X$ that enumerates $X$ without repeating elements.
        Define $f : \N \to \N$ to be the function that pairs each input $n$ with
        the $n$th element of $X$ that is enumerated by $\E_X$.

        \begin{itemize}
          \item If $X$ is infinite, then $\E_X$ will never halt or repeat an element.
            Each $n \in \N$ will eventually be paired with an element of $X$,
            so $f$ is total, injective, and computable.
          \item On the contrary, if $X$ is not total then there must
            exist some $n \in \N$ that is not paired with an element of $X$.
            This means that $\E_X$ will halt after a finite number of elements,
            specifically before the $n$th element is enumerated.
            Therefore, $X$ must be finite.
          \item Similarly, if $X$ is not injective, then there must exist
            some $2$ elements $n_1, n_2 \in \N$ that are paired to the same
            $k \in X$. This is a contradiction to the fact that
            $E_X$ enumerates \emph{unique} elements of $X$.
        \end{itemize}
      \end{proof}

    \subsubsection*{$\impliedby$}

    \emph{
      If $X$ is finite or the range of a total, computable, injective function,
      then $X$ is c.e.
    }
    
    \begin{proof}
      For the two cases:
      \begin{enumarabic}
        \item If $X$ is finite, then $X = \set{x_1, x_2, \ldots, x_n}$ for some $n \in \N$.
        We can define a turing machine $\E_X$ that outputs $x_1, x_2, \ldots, x_n$
        in order and then halts.
        Similarly, given an input $x$, we can check if $x$ occurs in the list
        $x_1, x_2, \ldots, x_n$ in finite time and halt if it does.
        Therefore, $X$ is c.e.

        \item If $X$ is infinite and it is the range of a
        total, computable, injective function $f: \N \to \N$.
        Then $X = \set{f(1), f(2), \ldots}$.
        Since $f$ is computable, $f = \varphi_{e}$ for some $e$.
        We can therefore enumerate $X$ by running $\varphi_e$
        on $1, 2, 3, \ldots$ and outputting the corresponding values
        of $f(1), f(2), f(3), \ldots$ that are generated by $\varphi_e$.
      \end{enumarabic}
    \end{proof}
  \end{answer}
\end{problem}
