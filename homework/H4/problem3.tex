\begin{problem}
  Prove that a c.e. set is computable \emph{if and only if} it is the
  range of an increasing, total computable function.

  \begin{answer}
    \subsubsection*{$\implies$}
      Suppose $X$ is c.e. and computable.
      Then $X = W_e$ for some $e$.
      Since $X$ is computable, we can specify a turing machine to print
      the elements of $X$ in increasing order:

      \step
      \begin{algorithm}[H]
        \SetAlgorithmName{$\mathsf{T} \mathsf{M}$}{}{}
        \caption{Enumerate $X$ in increasing order}
        \For{$i = 0, 1, 2, \ldots$}{
          \If{$\chi_X(i) = 1$}{
            \textbf{print} $i$
          }
        }
      \end{algorithm}

      \step
      Define a function $f$ that, given input $n$, outputs the $n$th element
      listed in the increasing-order enumeration of $X$.
      Then $f$ is an increasing, total computable function whose
      range is $X$.

    \subsubsection*{$\impliedby$}
      Suppose $X$ is the range of an increasing, total, computable function $f: \N \to \N$.
      We shall show that $X$ is computable.

      Since $f$ is total and increasing, we have
      \[ \forall n_1, n_2 \in \N,\; n_1 \leq n_2 \implies f(n_1) \leq f(n_2). \]

      Furthermore, since $f$ is computable, $f = \varphi_e$ for some $e$.

      We can compute the characteristic function of $X$, $\chi_X$, as follows:

      \begin{algorithm}[H]
        \SetAlgorithmName{$\mathsf{T} \mathsf{M}$}{}{}
        \caption{Compute $\chi_{X}(n)$}
        \For{$x = 0, 1, 2, \ldots$}{
          \If{$f(x) = n$}{
            \textbf{output} $1$
          }
          \ElseIf{$f(x) > n$}{
            \textbf{output} $0$
          }
        }
      \end{algorithm}

      Since $f$ is total and increasing, the turing machine will eventually either
      reach an $x$ such that $f(x) = n$ and output $1$, or encounter a value
      of $x$ such that $f(x) > n$ and output $0$.
      Therefore, $X$ is computable.
  \end{answer}
\end{problem}
