\begin{problem}[4]
  Prove that $K$ (the halting set) is \newterm{not} an index set.

  \begin{enumarabic}
    \item $K = \set{ e \given \varphi_e(e) \converges }$.
    \item An index set is a set $X$ such that, for all $e$ and $k$,
      if $\varphi_e = \varphi_k$ then $e \in X$ if and only if $k \in X$.
  \end{enumarabic}

  % rice's theorem


  \begin{answer}

    To show that $K$ is not an index set,
    we shall find a code $e \in K$ and show that $k \not \in K$
    for some $\varphi_k = \varphi_e$.

    \step
    Define a function $f$ that converges only on its own code and diverges
    for all other $n \in \N$. That is, if $e$ is the code of the
    machine that computes $f$, then

    \[
      f(n) = \begin{cases}
        1 & \text{if } n = e \\
        \diverges & \text{otherwise.}
      \end{cases}
    \]

    \step
    We can do this because of he recursion theorem.
    Note that $e \in K$ since $\varphi_e(e) \converges$.
    By the \emph{padding lemma}, for any $e$, there are infinitely
    many $k \neq e$ such that $\varphi_e = \varphi_k$.
    Pick one such $k$.
    What happens when we run $\varphi_k(k)$?
    Since $k \neq e$, $\varphi_e(k) \diverges$ since $\varphi_e(k) \diverges$.
    Therefore, $k \not\in K$.

    \step
    This means that $K$ must not be an index set, since
    the condition
    \[ \varphi_e = \varphi_k \implies \parens{e \in K \iff k \in K} \]
    does not hold.
  \end{answer}
\end{problem}
