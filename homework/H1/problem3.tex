\begin{problem}
  Prove that the squaring function is computable by providing
  a register machine which takes in $n$ in $R_0$ and outputs
  $n^2$ in $R_1$.
  You may use the multiplication function $x(n, m)$ ---
  which starts with $n$ in $R_0$ and $m$ in $R_1$ and outputs
  $n \cdot m$ in $R_2$ --- as a black box function labeled $M$.

  \begin{answer}
    \subsection*{Register Machine}
    \begin{figure}[H]
      \centering
      \begin{tikzpicture}
        \node[state, initial] (start) at (0, 0) {\newterm{start}};

        \node (sub0) [subtraction, right of=start] {$R_0^-$};
        \node (add1) [addition, right of=sub0] {$R_1^+$};
        \node (add2) [addition, right of=add1] {$R_2^+$};
        \node (sub2) [subtraction, below of=sub0] {$R_2^-$};
        \node (add0) [addition, right of=sub2] {$R_0^+$};
        \node (mult) [blackbox,  below of=sub2] {$M$};
        \node (sub1) [subtraction, below of=mult] {$R_1^-$};
        \node (sub2-2) [subtraction, below of=sub1] {$R_2^-$};
        \node (add1-2) [addition, right of=sub2-2] {$R_1^+$};
        \node (stop) [state, below of=sub2-2] {\newterm{stop}};
        % \node[state, right of=q0] (q1) {stop};

        % EDGES
        \draw (start) edge[above] node {} (sub0)
              (sub0) edge[above] node {} (add1)
              (add1) edge[above] node {} (add2)
              (add2) edge[bend right, above] node {} (sub0)
              
              (sub0) edge[left] node {$e$} (sub2)
              (sub2) edge[bend right, above] node {} (add0)
              (add0) edge[bend right, above] node {} (sub2)
              (sub2) edge[left] node {$e$} (mult)
              (mult) edge[left] node {} (sub1)
              (sub1) edge[loop right, left] node {} (sub1)
              (sub1) edge[left] node {$e$} (sub2-2)
              (sub2-2) edge[bend right, above] node {} (add1-2)
              (add1-2) edge[bend right, above] node {} (sub2-2)
              (sub2-2) edge[left] node {$e$} (stop);
      \end{tikzpicture}
      \caption{Calculate $f: n \mapsto n^2$ for $n \in \N$.}
      \label{fig:square}
    \end{figure}

    \newpage
    \subsection*{Idea}
    I found it useful to categorize the machine into five general parts
    (by height in Figure~\ref{fig:square}):

    \begin{enumarabic}
      \item \newterm{start}
      \item Copy $n$ to $R_1$ and $R_2$, which sets $R_0 \gets 0$.
      \item Copy $n$ from $R_2$ back to $R_0$.
      \item Call $M$ to calculate $R_2 \gets R_0 \cdot R_1 = n \cdot n = n^2$.
      \item Set $R_1 \gets 0$.
      \item Move $n^2$ from $R_2$ to $R_1$.
      \item \newterm{stop}
    \end{enumarabic}
    % write algorithm
    \subsection*{Algorithm}
    I wrote this to convince myself that the machine works as I intended. \\
    Please disregard if irrelevant.

    \step
    \begin{algorithm}[H]
      \caption{Calculate $f: n \mapsto n^2$ for $n \in \N$.}
        \newterm{start} \\
        \While(\comment*[f]{Copy $n$ to $R_1$ and $R_2$, which sets $R_0 = 0$}){$(\newterm{status} \coloneq R_0 - 1) \neq e$}{
          $R_1 \gets R_1 + 1$ \\
          $R_2 \gets R_2 + 1$
        }
        \While(\comment*[f]{Move $n$ from $R_2$ back to $R_0$}) {$(\newterm{status} \coloneq R_2 - 1) \neq e$}{
          $R_0 \gets R_0 + 1$
        }
        \textbf{call} $M$ \comment*[f]{Call $M$ to populate $R_2$ with $n^2 \coloneq R_0 \cdot R_1$} \\
        \While(\comment*[f]{Set $R_1 \gets 0$}) {$(\newterm{status} \coloneq R_1 - 1) \neq e$}{
          % $R_1 \gets R_1 - 1$
          $\mathbf{loop}$
        }
        \While(\comment*[f]{Move $n^2$ from $R_2$ to $R_1$}) {$(\newterm{status} \coloneq R_2 - 1) \neq e$}{
          $R_1 \gets R_1 + 1$
        }
        \newterm{stop}
    \end{algorithm}
  \end{answer}
\end{problem}
