\begin{problem}
  Prove that a set $X$ is computable \emph{if and only if}
  it and its complement (that is, $\set{n : n \not \in X}$)
  are c.e.

  \begin{answer}
    \begin{definition}
      A set $X$ is computable if and only if its characteristic function $\chi_X : \N \to \set{0, 1}$
      or its principal function $\chi_X : \N \to \set{0, 1}$ is Turing computable.
    \end{definition}

    \begin{definition}
      A set $X$ is said to be \emph{computably enumerable} (or c.e.) if it is the domain
      of $\varphi_e$ for some $e$. Then we write $X = W_e$.
    \end{definition}

    \step
    We shall show that $X$ is computable if and only if
    $X$ and its complement are computably enumerable (c.e.)
    in two steps;
    first, by showing that if $X$ is computable then $X$ and its complement are c.e.,
    then by showing that if $X$ and its complement are c.e. then $X$ is computable.

    % \begin{claim}
      \subsection*{1. If $X$ is computable then $X$ and $X^\sfC$ are c.e.}

      \begin{proof}
        Suppose $X$ is computable.
        Then we can construct a Turing machine $M_X$ that computes $\chi_X : \N \to \set{0, 1}$,
        the characteristic function of $X$.

        Consider these two universal machines:

        \begin{algorithm}[H]\label{alg2:X}
          \SetAlgorithmName{$\mathsf{T} \mathsf{M}$}{}{}
          \caption{Compute $f_1 : n \mapsto \begin{cases}
            1 & \text{if } M_X(n) = 1 \\
            \uparrow & \text{otherwise.}
          \end{cases}$}
          On input $n$: \\
          Run $M_X$ on $n$. \\
          \If{$M_X$ outputs $1$}{
            \textbf{output} $1$
          }
          \ElseIf{$M_X$ outputs $0$}{
            \textbf{$\mathbf{diverge}$}
          }
        \end{algorithm}

        \step
        $\sfT \sfM$ ~\ref{alg2:X} halts and outputs $1$ on input $n$ when
        $M_X$ halts and outputs $1$ on input $n$.
        For all other values of $n$, it diverges.
        Thus, $X = W_e$ where $e$ is the index of $\sfT \sfM~\ref{alg2:X}$.
        
        \begin{algorithm}[H]\label{alg2:XC}
          \SetAlgorithmName{$\mathsf{T} \mathsf{M}$}{}{}
          \caption{Compute $f_2 : n \mapsto \begin{cases}
            1 & \text{if } M_X(n) = 0 \\
            \uparrow & \text{otherwise.}
          \end{cases}$.}
          On input $n$: \\
          Run $M_X$ on $n$. \\
          \If{$M_X$ outputs $1$}{
            \textbf{$\mathbf{diverge}$}
          }
          \ElseIf{$M_X$ outputs $0$}{
            \textbf{output} $1$
          }
        \end{algorithm}

        \step
        $\sfT \sfM$ ~\ref{alg2:XC} halts and outputs $1$ on input $n$ when
        $M_X$ halts and outputs $0$ on input $n$.
        For all other values of $n$, it diverges.
        Thus, $X^\sfC = W_{e'}$ where $e'$ is the index of $\sfT \sfM~\ref{alg2:XC}$.

        \step
        Therefore, $X$ and $X^\sfC$ are c.e.
      \end{proof}


    % \begin{claim}
    \subsection*{2. If $X$ and its complement are c.e. then $X$ is computable.}

      \begin{proof}
        Suppose $X$ and its complement are c.e.
        Then $X = W_e$ and $X^\sfC = W_{e'}$ for some $e, e'$.

        We will construct a Turing machine $M_X$ that computes $\chi_X : \N \to \set{0, 1}$,
        the characteristic function of $X$.

        \begin{algorithm}[H]\label{alg2:chiX}
          \SetAlgorithmName{$\mathsf{T} \mathsf{M}$}{}{}
          \caption{Compute $\chi_X : n \mapsto \begin{cases}
            1 & \text{if } n \in X \\
            0 & \text{otherwise.}
          \end{cases}$}
          On input $n$: \\
          Initialize $\varphi_e$ and $\varphi_{e'}$ on input $n$. \\
          \Loop{
            Run $\varphi_e$ on $n$ for one more incremental step. \\
            Run $\varphi_{e'}$ on $n$ for incremental step. \\
            \If {$\varphi_e$ \newterm{halts}}{
              \textbf{output} $1$
            }
            \ElseIf{$\varphi_{e'}$ \newterm{halts}}{
              \textbf{output} $0$
            }
          }
        \end{algorithm}
      \end{proof}

    Note that since $X$ and $X^\sfC$ are c.e.,
    for any given input $n$, either $\varphi_e$ or $\varphi_{e'}$ will \emph{eventually} halt.
    Therefore, $\sfT \sfM$ ~\ref{alg2:chiX} eventaully halts and output $1$ if $n \in X$,
    or $0$ if $n \not \in X$.
    Therefore, $\sfT \sfM$ ~\ref{alg2:chiX} computes $\chi_X$,
    so $X$ is computable.
  \end{answer}
\end{problem}
