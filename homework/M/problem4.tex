\begin{problem}
  Let $M$ be the set
  \[ \set{ x \given \forall y < x \parens{ \varphi_x \neq \varphi_y } }. \]
  That is, $M$ is the set of minimal indices of computable functions:
  the smallest indices which define a given (partial) computable function.

  \begin{enumalph}
    \item Is $M$ immune? Simple?
      \begin{answer}
        \newterm{$M$ is immune but not simple.}

        \subsection*{Immunity}
        A set is immune if it is infinite but has no infinite c.e. subset.

        While $M$ is infinite, it does not contain any infinite c.e. subset.
        For the sake of contradiction, suppose it did
        and let the subset be $S$.
        Then there exists a bijection $f : \N \to S$ (since $S$ is c.e.).
        By the recursion theorem, there exists a fixed point $e$
        such that $e \neq f(e)$ and $\varphi_e = \varphi_{f(e)}$.
        But since $f$ is a bijection, $f(e) \in S \subset M$,
        so either $e$ is not minimal or $f(e)$ is not minimal,
        and both cases contradict the constitution of $M$.

        \subsection*{Simplicity}
        A set is simple if it is c.e. and its complement is immune.

        $M$ is not simple since it is not c.e.
        Suppose it was c.e., let $f : \N \to M$ be a computable bijection.
        Then by the recursion theorem, there exists a fixed point $e$
        of $f$ such that $\varphi_e = \varphi_{f(e)}$.
        Then $f(e) \in M$ since $f$ is a bijection unto $M$,
        implying that either $e$ is not minimal or $f(e)$ is not minimal,
        and each case contradicts the constitution of $M$.
        
      \end{answer}
    \newpage
    \item Is $M$ productive? Creative?
      \begin{answer}

        \crim{
          I got stuck on proving productive/creative properties for $M$. 
          I am still trying to figure it out.
          Are you open you giving hints?
          That could be helpful.
        }

        % \subsection*{Productivity}
        % A set $P$ is productive if there exists a \emph{productive} function
        % --- a (partial) computable function $f$ such that whenever
        % $W_e \subseteq P$, $f(e) \converges$ and $f(e) \in P \setminus W_e$.

        % \step
        % $M$ is not productive.
        % Suppose for the sake of contradiction $M$ is productive,
        % let $f$ be its productive function.
        % Then for any $e$ with $W_e \subseteq M$,
        % $f(e) \converges$ and $f(e) \in M \setminus W_e$.
        % This means that $M \setminus W_e$ is nonempty.

        
        % then there would exist
        % a productive function $f$ such that whenever $W_e \subseteq M$,
        % $f(e) \converges$ and $f(e) \in M \setminus W_e$.

        % \subsection*{Creativity}
        % A set is creative if it is c.e. and its complement is productive.

        % $M$ is creative.
        % Since $M$ is c.e., we need to show that $M^c$ is productive.
        % Define a productive function for $M^c$ as:

        % \[
        %   p(e) = \begin{cases}
        %     e & \text{if $W_e \subseteq M^c$} \\
        %     \diverges & \text{otherwise}.
        %   \end{cases}
        % \]



      \end{answer}
  \end{enumalph}
\end{problem}
