\begin{problem}
  Give an example of a partial computable function $i(x, y)$ such that:
  \begin{itemize}
    \item $i(x, y) \converges$ implies $i(x, y) = 0$ or $i(x, y) = 1$.
    \item If $A$ is computable, then there exists an $e$ such that
      $i(e, n) = \chi_A(n)$ for all $n$.
    \item $I_x \coloneq \set{ y \given i(x, y) \converges > 0 }$ is computable
      for all $x$.
  \end{itemize}
  Why does $i(x, y)$ not contradict Homework 3, Question 2?

  \begin{answer}
    \subsection*{Definition of $i(x, y)$}
    Let $i(x, y)$ be the following partial computable function:
    \[
      i(x, y) = \begin{cases}
        1 & \text{if } \varphi_x(y) \converges= 1 \\
        0 & \text{if } \varphi_x(y) \converges= 0 \\
      \end{cases}
    \]
  \end{answer}
\end{problem}
