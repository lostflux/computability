\begin{problem}
  Give an example of a partial computable function $i(x, y)$ such that:
  \begin{itemize}
    \item $i(x, y) \converges$ implies $i(x, y) = 0$ or $i(x, y) = 1$.
    \item If $A$ is computable, then there exists an $e$ such that
      $i(e, n) = \chi_A(n)$ for all $n$.
    \item $I_x \coloneq \set{ y \given i(x, y) \converges > 0 }$ is computable
      for all $x$.
  \end{itemize}
  Why does $i(x, y)$ not contradict Homework 3, Question 2?

  \begin{answer}
    \subsection*{Construction of $i(x, y)$}
    Let $i(x, y)$ be the following partial computable function:
    \[
      i(x, y) = \begin{cases}
        1 & \text{if } \varphi_x(y) \converges= 1 \\
        0 & \text{if } \varphi_x(y) \converges= 0 \\
        \diverges & \text{otherwise}.
      \end{cases}
    \]

    \begin{enumarabic}
      \item Whenever $i(x, y) \converges$, then $i(x, y) = 0$ or $i(x, y) = 1$.
      \item When $A$ is computable, then there exists some index $e$
        such that $\varphi_e = \chi_A$.
        Then $i(e, n) = \varphi_e(n) = \chi_A(n)$ for all $n$.
      \item $I_x = \set{ y \given i(x, y) \converges > 0 }$
        $=\blue{\set{y \given \varphi_x(y) = 1}}$.
        For any fixed $x$, we can compute $I_x$ by iterating
        through all $y \in \N$ and checking if
        $i(x, y) = 1$.
    \end{enumarabic}

    \subsection*{Comparison with Homework 3, Question 2}

    Homework 3, Question 2 used a diagonalization argument
    to show that there is no uniform listing of all 
    characteristic functions of the computable sets.

    The function $i(x, y)$ does not contradict this fact because
    it does not exclusively list all characteristic functions
    of computable sets.
    Instead, it simulates every other function, althogh it only
    converges if the simulated function converges to $i \in \set{0, 1}$
    and diverges otherwise.
  \end{answer}
\end{problem}
