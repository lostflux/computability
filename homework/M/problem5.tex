\begin{problem}
  Prove that $W_e \leq_1 H$ for all $e$,
  then prove that $H \leq_1 K$.

  \begin{answer}
    \[ M = \set{ \vector{e, k, n} \given \varphi_e(k) \converges = n } \]
    \[ H = \set{ \vector{e, k} \given \varphi_e(k) \converges } \]
    \[ K = \set{ e \given \varphi_e(e) \converges } \]

    \subsection*{$W_e \leq_1 H$ for all $e$}

    Define the function $f : \N \to \N^2$ by
    \[
      f(k) = \vector{e, k}
    \]
    \begin{enumarabic}
      \item When $n \in W_e$, then $\varphi_e(n) \converges$,
        so $f(n) = \vector{e, n} \in H$.
      \item When $n \notin W_e$, then $\varphi_e(n) \diverges$,
        so $f(n) \diverges$.
      \item $f$ is an injection from $W_e$ to $H$ since $e$ is fixed
        after the definition of $f$, and every each $k' \in W_e$
        is mapped to a unique element $\vector{e, k'} \in H$.
    \end{enumarabic}

    Therefore, $n \in W_e$ \emph{if and only if} $f(n) \in H$,
    so $W_e \leq_1 H$ with witness $f$.
    
    \subsection*{$H \leq_1 K$}

    For any $e \in \N$ with $W_e \subseteq \N$ being the domain
    of $\varphi_e$, we can define the function $g : \N^2 \to \N$
    as follows:

    First, for each $(i, j) \in \N^2$, let $e_{(i, j)}$ be the code of
    the machine that computes $h_{(i, j)}$, where $h_{(i, j)} : \N \to \N$ is defined as:
    % \[
    %   h_{(i, j)}(n) = \begin{cases}
    %     e_{(i, j)} & \text{ if $n = e_{(i,j)}$ and $\varphi_i(j) \converges$ } \\
    %     \diverges & \text{otherwise}.
    %   \end{cases}
    % \]
    \[
      h_{(i, j)}(n) = \begin{cases}
        1 & \text{ if $n = e_{(i,j)}$ and $\varphi_i(j) \converges$ } \\
        \diverges & \text{otherwise}.
      \end{cases}
    \]
    The recursion theorem with parameters tells us that
    such a function $h_{(i, j)}$ exists and can be obtained computably.

    \begin{proof}
      To see this, let $f(y, x_0, x_1)$ be a total computable function,
      then the recursion theorem tells us that there exists an injective,
      total computable function $r(x_0, x_1)$ such that
      \[
        \varphi_{r(x_0, x_1)} = \varphi_{f(r(x_0, x_1), x_0, x_1)}.
      \]
      Let $e$ be an index for $f(r(x_0, x_1), x_0, x_1)$,
      and define $r$ via $r(x_0, x_1) = d(e, x_0, x_1)$.
      Then $r(x_0, x_1)$ is a fixed point.
    \end{proof}

    Define the function $g : \N^2 \to \N$ by
    % \[
    %   g(e, k) = h_{(e, k)}(e_{(e, k)})
    % \]
    \[
      g(e, k) = e_{(e,k)}
    \]

    \begin{enumarabic}
      \item When $\vector{e, k} \in H$, $\varphi_e(k) \converges$.
        Note that $\varphi_{(e, k)}(e_{(e, k)})$ simulates
        $\varphi_e(k)$, and $\varphi_e(k) \converges$,
        and since it received $\varphi_{(e, k)}$ as input,
        it halts and outputs $1$.
        Thus, $g(e, k) = e_{(e, k)} \in K$ when $\vector{e, k} \in H$.
      \item When $\vector{e, k} \notin H$, then
        $\varphi_{(e, k)}(e_{(e, k)})$ simulates
        $\varphi_e(k)$, and $\varphi_e(k) \diverges$,
        so $g(e, k) \diverges$.
        Thus, when $\vector{e, k} \notin H$,
        $g(e, k) = \varphi_{(e, k)} \notin K$.
      \item $g$ is an injection from $H$ to $K$ since
        $\vector{e, k} \in H$ is mapped to a unique element $e_{(e, k)} \in K$.
    \end{enumarabic}

    Therefore, $\vector{e, k} \in H$ \emph{if and only if} $g(e, k) \in K$,
    so $H \leq_1 K$ with witness $g$.

  \end{answer}
\end{problem}
