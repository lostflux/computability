\begin{problem}
  Prove that, for all $n$ and $f : \omega \to \omega$,
  there is a computable function $g : \omega^{n+1} \to \omega$
  such that
  \[
    f(x) =
      \lim_{s_0 \to \inf}
      \lim_{s_1 \to \inf}
      \cdots
      \lim_{s_{n-1} \to \inf}
      g(x, s_0, s_1, \ldots, s_{n-1})
  \]
  if and only if $f \leq_T \emptyset^{(n)}$.

  \begin{answer}
    % We will use the \newterm{limit lemma},
    % which states that a function $f : \omega \to \omega$
    % is limit computable if and only if $f \leq_T \emptyset'$.
    
    % Suppose
    % $\displaystyle f(x) = \lim_{s_0 \to \inf} \lim_{s_1 \to \inf} \cdots \lim_{s_{n-1} \to \inf} g(x, s_0, s_1, \ldots, s_{n-1})$.

    % First, we see that $f$ is $\Sigma^0_{2n+2}$:
    % \begin{alignat*}{10}
    %   &\exists s_0 \forall \parens{s_{0'} > s_0} \\
    %   &\qquad \exists s_1 \forall \parens{s_{1'} > s_1} \\
    %   &\qquad \qquad \ddots \\
    %   &\qquad \qquad \qquad \exists s_{n-1} \forall \parens{s_{n-1'} > s_{n-1}} \\
    %   &\qquad \qquad \qquad \qquad g(x, s_{0'}, s_{1'}, \ldots, s_{(n-1)'}) = f(x).
    % \end{alignat*}

    % \step
    % Similarly, $f$ is $\Pi^0_{2n+2}$:
    % \begin{alignat*}{10}
    %   &\forall s_0 \exists \parens{s_{0'} > s_0} \\
    %   &\qquad \forall s_1 \exists \parens{s_{1'} > s_1} \\
    %   &\qquad \qquad \ddots \\
    %   &\qquad \qquad \qquad \forall s_{n-1} \exists \parens{s_{n-1'} > s_{n-1}} \\
    %   &\qquad \qquad \qquad \qquad g(x, s_{0'}, s_{1'}, \ldots, s_{(n-1)'}) = f(x).
    % \end{alignat*}

    % \step
    % Thus, $f(x)$ is $\Delta^0_{2n+2}$.

    % By the limit lemma (Lemma 4),
    % $f$ is computable from $\varnothing'$
    % \emph{if and only if} $f$ is $\Delta^0_2$.
    % Since the limit lemma relativizes,
    % $f$ is computable from $\varnothing^{(n)}$
    % \emph{if and only if} $f$ is $\Delta^0_{n+1}$.

    % \step
    % Furthermore, since every computable function is c.e.,
    % and every c.e. function is limit computable,
    % $f$ is $n$-limit computable

    % \step
    % Finally, by relativizing Lemma 2. of the limit lemma,
    % $f$ is $n$-limit computable \emph{if and only if} $f \leq_T \varnothing^{(n)}$.

    We will use induction to show that $g$ is limit-computable.

    \begin{enumarabic}
      \item For $i = 1$, we show that $\displaystyle f = \lim_{s_0 \to \infty} g(x, s_0)$
        if and only if $f \leq_T \emptyset^{(1)}$.
        \begin{enumroman}
          \item $\implies$: \\
            Suppose $\displaystyle f = \lim_{s_0 \to \infty} g(x, s_0)$.
            Then there exists an $e$ such that $\Phi_e(x, s_0) = g(x, s_0)$
            and $\Phi_e(x, n) = f(x)$ for all $n \geq N$ for some $N \in \omega$.
            Therefore, $f \leq_T \emptyset^{(1)}$.
          \item Suppose $\displaystyle f = \lim_{s_0 \to \infty} g(x, s_0)$.
            Then $f$ is limit computable with $1$ limit.
            By definition, $f$ is a computable function, so $f$ is c.e.
            and therefore limit computable with a single limit.
          \item Suppose $f \leq_T \emptyset^{(0)}$.
            Then $f$ is limit computable with $0$ limits.
            By definition, $f$ is a computable function, so $f$ is c.e.
            and therefore limit computable with a single limit.
        \end{enumroman}
      
      $g$ is computable with one limit.
        By definition, $g$ is a computable function, so $g$ is c.e.
        and therefore limit computable with a single limit.
      \item For $1 < i \leq n$, we assume that $g$ is limit computable with $i-1$ limits.
        We show that $g$ is limit computable with $i$ limits.
        By the induction hypothesis, $g$ is limit computable with $i-1$ limits,
        so $g$ is c.e. and therefore limit computable with $i$ limits.
    \end{enumarabic}
  \end{answer}
\end{problem}
