\begin{problem}
  Show that HW3 Q5 relativizes.
  That is, show that $A$ is $X$-computable
  if and only if $A$ and $A^c$ are both $X$-ce.

  \begin{answer}
    \subsection*{($\Longrightarrow$)}
    Suppose $A$ is $X$-computable.
    Then there is an $e$ such that $\Phi_e^X = \chi_A$.
    This means that for each $n$,
    \[
      \Phi_e^X(n) = \begin{cases}
        1 & \text{if } n \in A \\
        0 & \text{if } n \not\in A \qquad \text{ (hence $n \in A^c$)}
      \end{cases}
    \]

    $A$ can be computably enumerated by a turing machine
    that goes through all $n = 1, 2, 3, \ldots$ and
    outputs $n$ if $\Phi_e^X(n) = 1$.

    \step
    \begin{algorithm}[H]
      \SetAlgorithmName{$\mathsf{T} \mathsf{M}$}{}{}
      \caption{Enumerate $A$}
      \For{$n = 0, 1, 2, \ldots$}{
        \If{$\Phi_e^X(n) = 1$}{
          \textbf{output} $n$
        }
      }
    \end{algorithm}

    \step
    Similarly, $A^c$ can be computably enumerated by a turing machine
    that goes through all $n = 1, 2, 3, \ldots$ and
    outputs $n$ if $\Phi_e^X(n) = 0$.

    \step
    \begin{algorithm}[H]
      \SetAlgorithmName{$\mathsf{T} \mathsf{M}$}{}{}
      \caption{Enumerate $A^c$}
      \For{$n = 0, 1, 2, \ldots$}{
        \If{$\Phi_e^X(n) = 0$}{
          \textbf{output} $n$
        }
      }
    \end{algorithm}

    \step
    Therefore, $A$ and $A^c$ are both $X$-ce.

    \clearpage

    \subsection*{($\Longleftarrow$)}
    Suppose $A$ and $A^c$ are both $X$-ce.
    Then $A$ is the domain of some $X$-computable function $f$,
    and $A^c$ is the domain of some $X$-computable function $g$.
    We can define a function $h$ that computes $A$ as follows:
    \[
      h(n) = \begin{cases}
        1 & \text{if } f(n) \text{ is defined} \\
        0 & \text{if } g(n) \text{ is defined}
      \end{cases}
    \]

    Specifically, let $f = \Phi_i^X$ and $g = \Phi_j^X$.

    Then we can define $\Phi_h^X$ as follows:

    \step
    \begin{algorithm}[H]
      \SetAlgorithmName{$\mathsf{TM}$}{}{}
      \caption{Compute $A$}
      On input $n$: \\
      \For{$k = 1, 2, 3, \ldots$}{
        \If{$\Phi_{i,k}^X(n) \downarrow$}{
          \textbf{output} $1$
        }
        \If{$\Phi_{j,k}^X(n) \downarrow$}{
          \textbf{output} $0$
        }
      }
    \end{algorithm}

    \step
    Since \emph{both} $A$ and $A^c$ are $X$-ce
    and $A \cup A^c = \omega$,
    for any $n \in \omega$, eventually either one of
    $\Phi_i^X(n)$ or $\Phi_j^X(n)$, simulated for some finite $k$ steps,
    will halt.
    Thus, the $\mathsf{TM}$ eventually halts and outputs either
    $1$ or $0$ for any $n \in \omega$, effectively computing
    $\chi_A$.

    \step
    Therefore, $A$ is $X$-computable.
  \end{answer}
\end{problem}
