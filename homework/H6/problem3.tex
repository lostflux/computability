\begin{problem}
  Give an example of a set $X$ such that $X \bot_T \emptyset^{(n)}$
  for all $n > 1$. \\
  \emph{
    Hint: we are only required to perform
    (priority) constructions computably.
  }

  \begin{answer}
    We use a priority construction to define $X$.

    % First, define $\displaystyle \emptyset^{(\omega_{>1})} = \bigcup_{n = 2, 3, 4, \ldots} \emptyset^{(n)}$.

    Define the requirements $R_e$ and $Q_e$ as follows:
    \[
      R_e : \chi_X \neq \Phi_e^{\emptyset^{(e)}}
    \]
    % \[
    %   Q_e : \chi_{\emptyset^{(\omega_{>1})}} \neq \Phi_e^X
    % \]

    Let $X_0 = \emptyset$.
    At each step $s+1$, pick $x \not \in X_s$.
    Simulate $\Phi_x^{\emptyset^{(s)}}(x)$.
    If $\Phi_x^{\emptyset^{(s)}}(x) \convergesto 0$, then
    set $X_{s+1} = X_s \cup \set{x}$.
    Otherwise, repeat this step until such an $x$ is found.

    \step
    For each $n > 1$, let $x_n$ be the $n$-th element that wasadded to $X$,
    then $\Phi_x^{\emptyset^{(n)}}(x_n) \convergesto 0$, so
    $\chi_X(x_n) = 1 \neq \Phi_x^{\emptyset^{(n)}}(x_n)$.
    Thus, $X \bot_T \emptyset^{(n)}$ for all $n > 1$.
  \end{answer}
\end{problem}
