\begin{problem}
  Describe informally what process you would use to determine
  if the register machine coded by $n$ contains a subtraction node.
  You may assume that the $n$ you are given is a valid code
  for a register machine. \\
  \emph{You do not need to provide a machine which runs your process.}

  \begin{answer}
    The encoding of nodes is done as follows:
    \begin{enumroman}
      \item If $N_i$ is an addition node $R_j^+$ with output node $N_k$,
        then $\#N_i = 3^j \times 5^k$.
      \item If $N_i$ is a subtraction node $R_j^-$ with output node $N_k$
        and empty output node $N_l$, then $\#N_i = 2 \times 3^j \times 5^k \times 7^l$.
      \item The encoding of $M$ is then computed as $\displaystyle \#M = \prod_{i=1}^n p_i^{\#N_i}$.
    \end{enumroman}

    In particular, a node $N_i$ is a subtraction node \emph{if and only if}
    its encoding $\#N_i$ is divisible by $2$.

    Given an encoding $\#M$ of a register machine $M$,
    we can determine if $M$ contains a subtraction node as follows:
    
    \begin{enumarabic}
      \item Compute the prime factorization of $\#M$.
      \item Iterate through each prime factor $p_i$ and check if its exponent,
        equivalent to $\#N_i$, is divisible by $2$.
        If it is, then node $N_i$ is a subtraction node.
      \item If none of the prime factors have an exponent divisible by $2$,
        then the register machine $M$ does not contain a subtraction node.
    \end{enumarabic}
    
  \end{answer}
\end{problem}
