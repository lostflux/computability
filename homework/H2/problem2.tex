\begin{problem}
  Show that the set of powers of $2$ is computable
  by building a Turing machine.

  \begin{answer}
    \subsection*{Idea}
    A number (in binary) is a power of $2$ if and only if it has exactly one $1$ bit.
    Therefore, we can construct a Turing machine as follows: that scans the input tape from left to right,
    counting the number of $1$ bits it encounters.
    \begin{enumarabic}
      \item The Turing Machine starts in state $q_0$.
      \item While in $q_0$, it reads the tape at the current position.
        \begin{enumalph}
          \item If it reads a $0$, it moves right and remains in $q_0$.
          \item If it reads a $1$, it moves right and transitions to $q_1$.
        \end{enumalph}
      \item While in $q_1$, it reads the tape at the current position.
        \begin{enumalph}
          \item If it reads a $0$, it moves right and remains in $q_1$.
          \item If it reads a $1$, it moves right and transitions to $q_2$.
          \item If it reads a blank symbol, it halts in $q_1$. \\
            \emph{
              This is an accepting scenario since the Turing machine has encountered \\
              exactly one $1$ bit in the entire binary string.
            }
        \end{enumalph}
      \item $q_2$ is a non-accepting state without any transitions. If in $q_2$,
        it does not matter what the Turing machine reads---the string is
        not a power of $2$ since it already has more than one $1$ bit.
        Thus, reading any symbol while in $q_2$ would cause it to halt
        and not accept the input as a power of $2$.
    \end{enumarabic}

    \subsection*{Turing Machine}

    \begin{enumarabic}
      \item $\vector{q_0, \sfzero, \sfR, q_0}$
      \item $\vector{q_0, \sfone, \sfR, q_1}$
      \item $\vector{q_1, \sfzero, \sfR, q_1}$
      \item $\vector{q_1, \sfone, \sfR, q_2}$
    \end{enumarabic}

    \begin{figure}[H]
      \centering
      \begin{tikzpicture}
        \node[state, initial] (q0) at (0, 0) {$q_0$};
        \node[state, , accepting, right of=q0] (q1) {$q_1$};
        \node[state, right of=q1] (q2) {$q_2$};

        \draw (q0) edge[above, loop above] node{$\vector{\sfzero, \sfR}$} (q0)
              (q0) edge[above] node{$\vector{\sfone, \sfR}$} (q1)
              (q1) edge[above, loop above] node{$\vector{\sfzero, \sfR}$} (q1)
              (q1) edge[above] node{$\vector{\sfone, \sfR}$} (q2);
      \end{tikzpicture}
    \end{figure}
  \end{answer}
\end{problem}
