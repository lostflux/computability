\begin{problem}[2]
  Show that the set of powers of $2$ is computable
  by building a Turing machine.

  \begin{answer}
    \subsection*{Idea}
    A number (in binary) is a power of $2$ if and only if it has exactly one $1$ bit.
    To show that the set is computable, we can construct a Turing machine that computes
    the characteristic function $\chi_{\text{pow}_2}(x)$, which is $1$ if $x$ is a power
    of $2$ and $0$ otherwise:

    \step
    To show that the 
    \begin{enumarabic}
      \item \newterm{start} in state $q_0$.
      \item While in $q_0$, read the tape at the current position.
        \begin{enumalph}
          \item If a $\sfzero$ is read, move right and remain in $q_0$.
          \item If a $\sfone$ is read, move right and transition to $q_1$.
          \item If tape is blank at current position, move left and transition to $q_3$.
        \end{enumalph}
      \item While in $q_1$, read the tape at the current position.
        \begin{enumalph}
          \item If a $\sfzero$ is read, it move right and remain in $q_1$.
          \item If a $\sfone$ is read, input cannot be a power of $2$. \textsc{Halt} and output $\sfzero$.
          \item If tape is blank at current position, input is a power of $2$.
            \textsc{Halt} and output $\sfone$.
        \end{enumalph}
    \end{enumarabic}

    \subsection*{Turing Machine}

    \begin{enumarabic}
      \item $\vector{q_0, \sfzero, \sfR, q_0}$
      \item $\vector{q_0, \sfone, \sfR, q_1}$
      \item $\vector{q_1, \sfzero, \sfR, q_1}$
      \item $\vector{q_1, \sfone, \sfR, q_2}$
    \end{enumarabic}

    \begin{figure}[H]
      \centering
      \begin{tikzpicture}
        \node[state, initial] (q0) at (0, 0) {$q_0$};
        \node[state, , accepting, right of=q0] (q1) {$q_1$};
        \node[state, right of=q1] (q2) {$q_2$};

        \draw (q0) edge[above, loop above] node{$\vector{\sfzero, \sfR}$} (q0)
              (q0) edge[above] node{$\vector{\sfone, \sfR}$} (q1)
              (q1) edge[above, loop above] node{$\vector{\sfzero, \sfR}$} (q1)
              (q1) edge[above] node{$\vector{\sfone, \sfR}$} (q2);
      \end{tikzpicture}
    \end{figure}
  \end{answer}
\end{problem}
