\begin{problem}
  Show that the set of multiples of $4$ is computable
  by building a Turing machine.

  \begin{answer}
    \subsection*{Idea}
    A number (in binary) is a multiple of $4$ if either
    the number is $0$ or the last two bits are $0$.
    We check this by reading the input from left to right, until we reach the end
    of the input, then checking the last two bits on the tape.
    \begin{enumarabic}
      \item The Turing machine starts in $q_0$. It reads the tape at the current position.
        \begin{enumalph}
          \item If it reads a $0$, it moves right and remains in $q_0$.
          \item If it reads a $1$, it moves right and remains in $q_0$.
          \item If it reads a blank ($\ast$), it moves left on the tape and transitions to $q_1$.
        \end{enumalph}
      \item While in $q_1$, it reads the tape at the current position.
        \begin{enumalph}
          \item If it reads a $0$, it moves left and transitions to $q_2$.
          \item If it reads a $1$, or a blank symbol ($\ast$), it halts in $q_1$ and does not accept.
        \end{enumalph}
      \item While in $q_2$, it reads the tape at the current position.
        \begin{enumalph}
          \item If it reads a $0$ or a blank $(\ast)$, it moves right and transitions to $q_3$. \\
            \emph{NOTE: we include the empty symbol $\ast$ here to account for the number $0$.}
          \item If it reads a $1$, it halts in $q_2$ and does not accept.
        \end{enumalph}
      \item $q_3$ is an accepting state. If the Turing machine reaches $q_3$, whatever it reads next
        halts the machine and accepts the input as a multiple of $4$.
    \end{enumarabic}

    \subsection*{Turing Machine}

    \begin{enumarabic}
      \item $\vector{q_0, \sfzero, \sfR, q_0}$
      \item $\vector{q_0, \sfone, \sfR, q_0}$
      \item $\vector{q_0, \ast, \sfL, q_1}$
      \item $\vector{q_1, \sfzero, \sfL, q_2}$
      \item $\vector{q_2, \sfzero, \sfL, q_3}$
      \item $\vector{q_2, \ast, \sfL, q_3}$
    \end{enumarabic}

    \begin{figure}[H]
      \centering
      \begin{tikzpicture}
        \node[state, initial] (q0) at (0, 0) {$q_0$};
        \node[state, right of=q0] (q1) {$q_1$};
        \node[state, right of=q1] (q2) {$q_2$};
        \node[state, right of=q2, accepting] (q3) {$q_3$};

        % EDGES
        \draw (q0) edge[above, loop above] node{$\vector{\sfzero, \sfR}$} (q0)
              (q0) edge[below, loop below] node{$\vector{\sfone, \sfR}$} (q0)
              (q0) edge[above] node{$\vector{\ast, \sfL}$} (q1)
              (q1) edge[above] node{$\vector{\sfzero, \sfL}$} (q2)
              (q2) edge[above] node{$\vector{\sfzero, \sfL}$} (q3)
              (q2) edge[below] node{$\vector{\ast, \sfL}$} (q3);

      \end{tikzpicture}
    \end{figure}
  \end{answer}
\end{problem}
