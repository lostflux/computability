% vim: set spell spelllang=en_us:
\documentclass[10pt, final]{article}
% \documentclass[11pt]{amsart}

\input{~/common}
% \input{~/globals/macros}

\hyphenation{auto-morphism auto-morphisms homo-geneous}

%%%%%%%%%%%%%%%%%%%%%%%%%%%%%%%%%%%%%%%%%%%%%%%%%%%%%%%%%%%%%%%%%%%%%%
%%%%%%%%%%%%%%%%%%%%%%%%%%%%%%%%%%%%%%%%%%%%%%%%%%%%%%%%%%%%%%%%%%%%%%
%%%%%%%%%%%%%%%%%%%%%%%%%%%%%%%%%%%%%%%%%%%%%%%%%%%%%%%%%%%%%%%%%%%%%%

% \title{%
%   On the derivations and automorphisms\\
%   of the algebra $\kk\langle x,y\rangle/(yx-xy-x^N)$
% }
% \author{Amittai Siavava\thanks{%
%     Undergraduate student at Dartmouth College.
%   }
% }
% \ifoptionfinal
%   {\date{\today}}
%   {\date{Started on March 2021; compiled \today}}

  
\begin{document}

% \setlength{\headheight}{13.0pt}
% \setlength{\footskip}{13.0pt}


% TITLE
\newdate{due-date}{12}{05}{2024}
\PSET{5 --- \displaydate{due-date}}
  {Spring 2024}
  {Miller}
  {Amittai Siavava}
  {Math 29: Computability Theory}

% CREDIT STATEMENT
% \CreditStatement{
%   I worked on these problems alone,
%   with reference to class notes and the following books:
%   \begin{enumarabic}
%     \item \textbf{\textit{Introduction to Analysis}} by Maxwell Rosenlicht
%   \end{enumarabic}
% }

% \bigskip

\begin{problem}
  Show that if $C$ is a computable set, then $C \leq_m X$ for any set
  which is empty and has nonempty complement.

  \begin{answer}
    
  \end{answer}
\end{problem}


\newpage
\begin{problem}
  Show that the set of powers of $2$ is computable
  by building a Turing machine.

  \begin{answer}
    \subsection*{Idea}
    A number (in binary) is a power of $2$ if and only if it has exactly one $1$ bit.
    Therefore, we can construct a Turing machine as follows: that scans the input tape from left to right,
    counting the number of $1$ bits it encounters.
    \begin{enumarabic}
      \item The Turing Machine starts in state $q_0$.
      \item While in $q_0$, it reads the tape at the current position.
        \begin{enumalph}
          \item If it reads a $0$, it moves right and remains in $q_0$.
          \item If it reads a $1$, it moves right and transitions to $q_1$.
        \end{enumalph}
      \item While in $q_1$, it reads the tape at the current position.
        \begin{enumalph}
          \item If it reads a $0$, it moves right and remains in $q_1$.
          \item If it reads a $1$, it moves right and transitions to $q_2$.
          \item If it reads a blank symbol, it halts in $q_1$. \\
            \emph{
              This is an accepting scenario since the Turing machine has encountered \\
              exactly one $1$ bit in the entire binary string.
            }
        \end{enumalph}
      \item $q_2$ is a non-accepting state without any transitions. If in $q_2$,
        it does not matter what the Turing machine reads---the string is
        not a power of $2$ since it already has more than one $1$ bit.
        Thus, reading any symbol while in $q_2$ would cause it to halt
        and not accept the input as a power of $2$.
    \end{enumarabic}

    \subsection*{Turing Machine}

    \begin{enumarabic}
      \item $\vector{q_0, \sfzero, \sfR, q_0}$
      \item $\vector{q_0, \sfone, \sfR, q_1}$
      \item $\vector{q_1, \sfzero, \sfR, q_1}$
      \item $\vector{q_1, \sfone, \sfR, q_2}$
    \end{enumarabic}

    \begin{figure}[H]
      \centering
      \begin{tikzpicture}
        \node[state, initial] (q0) at (0, 0) {$q_0$};
        \node[state, , accepting, right of=q0] (q1) {$q_1$};
        \node[state, right of=q1] (q2) {$q_2$};

        \draw (q0) edge[above, loop above] node{$\vector{\sfzero, \sfR}$} (q0)
              (q0) edge[above] node{$\vector{\sfone, \sfR}$} (q1)
              (q1) edge[above, loop above] node{$\vector{\sfzero, \sfR}$} (q1)
              (q1) edge[above] node{$\vector{\sfone, \sfR}$} (q2);
      \end{tikzpicture}
    \end{figure}
  \end{answer}
\end{problem}


\newpage
\begin{problem}
  Verify that the $f$ constructed in the High-Low lecture
  notes dominates every total computable function, but does not
  compute $K$.

  \begin{answer}

  \end{answer}
\end{problem}


\newpage
\begin{problem}[4]
  Prove that $K$ (the halting set) is \newterm{not} an index set.

  \begin{enumarabic}
    \item $K = \set{ e \given \varphi_e(e) \converges }$.
    \item An index set is a set $X$ such that, for all $e$ and $k$,
      if $\varphi_e = \varphi_k$ then $e \in X$ if and only if $k \in X$.
  \end{enumarabic}

  % rice's theorem


  \begin{answer}

    To show that $K$ is not an index set,
    we shall find a code $e \in K$ and show that $k \not \in K$
    for some $\varphi_k = \varphi_e$.

    \step
    Define a function $f$ that converges only on its own code and diverges
    for all other $n \in \N$. That is, if $e$ is the code of the
    machine that computes $f$, then

    \[
      f(n) = \begin{cases}
        1 & \text{if } n = e \\
        \diverges & \text{otherwise.}
      \end{cases}
    \]

    \step
    We can do this because of he recursion theorem.
    Note that $e \in K$ since $\varphi_e(e) \converges$.
    By the \emph{padding lemma}, for any $e$, there are infinitely
    many $k \neq e$ such that $\varphi_e = \varphi_k$.
    Pick one such $k$.
    What happens when we run $\varphi_k(k)$?
    Since $k \neq e$, $\varphi_e(k) \diverges$ since $\varphi_e(k) \diverges$.
    Therefore, $k \not\in K$.

    \step
    This means that $K$ must not be an index set, since
    the condition
    \[ \varphi_e = \varphi_k \implies \parens{e \in K \iff k \in K} \]
    does not hold.
  \end{answer}
\end{problem}


\newpage
\begin{problem}
  Describe informally what process you would use to determine
  if the register machine coded by $n$ contains a subtraction node.
  You may assume that the $n$ you are given is a valid code
  for a register machine. \\
  \emph{You do not need to provide a machine which runs your process.}

  \begin{answer}
    The encoding of nodes is done as follows:
    \begin{enumroman}
      \item If $N_i$ is an addition node $R_j^+$ with output node $N_k$,
        then $\#N_i = 3^j \times 5^k$.
      \item If $N_i$ is a subtraction node $R_j^-$ with output node $N_k$
        and empty output node $N_l$, then $\#N_i = 2 \times 3^j \times 5^k \times 7^l$.
      \item The encoding of $M$ is then computed as $\displaystyle \#M = \prod_{i=1}^n p_i^{\#N_i}$.
    \end{enumroman}

    In particular, a node $N_i$ is a subtraction node \emph{if and only if}
    its encoding $\#N_i$ is divisible by $2$.

    Given an encoding $\#M$ of a register machine $M$,
    we can determine if $M$ contains a subtraction node as follows:
    
    \begin{enumarabic}
      \item Compute the prime factorization of $\#M$.
      \item Iterate through each prime factor $p_i$ and check if its exponent,
        equivalent to $\#N_i$, is divisible by $2$.
        If it is, then node $N_i$ is a subtraction node.
      \item If none of the prime factors have an exponent divisible by $2$,
        then the register machine $M$ does not contain a subtraction node.
    \end{enumarabic}
    
  \end{answer}
\end{problem}


\newpage
\begin{problem}
  Suppose $C$ is the set of valid codes for machines based on
  our coding scheme defined in class. Give a total, computable
  bijection from $\omega$ to $C$,
  i.e. a computable function which associates every natural number
  to a unique machine. 

  \begin{answer}
    
  \end{answer}
\end{problem}


\vfill

\end{document}
