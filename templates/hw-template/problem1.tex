\begin{problem}
  Prove that $\displaystyle \int\limits_0^1 f(x) \d x = 0$ if
  $f(\frac{1}{n})= 1$ for all $n \in \N$ and $f(x) = 0$ for all other $x$.
\end{problem}

% \begin{answer}
\begin{claim}
  $f$ is Rieman integrable on $[0, 1]$.
\end{claim}

\begin{proof}
  Since $f = 0$ at all points $x \in [0, 1] \setminus \{1/n : n \in \N\}$,
  $f$ is continuous at all such $x$. Therefore, we can consider
  the points of the form $1/n, n \in \N$ as discontinuities of $f$.
  However, since $\N$ has measure zero (since it is countable),
  $\set{1/n : n \in \N}$ also has measure zero.
  By the Lebesgue criterion for Riemann integrability, $f$ is Riemann integrable,
  so $\displaystyle \int\limits_0^1 f(x) \d x$ exists.
\end{proof}

\begin{claim}
  $\displaystyle \int\limits_0^1 f(x) \d x = 0$.
\end{claim}
\begin{proof}
  Given a partition $P_n$ of $[0, 1]$ into $n$ subintervals,
  \blue{\[
    L(f, P) = \sum_{i=1}^n m_i \Delta x_i
            \leq \int \limits_0^1 f(x) \d x
            \leq \sum_{i=1}^n M_i \Delta x_i
            = U(f, P),
  \]}
  where $m_i = \inf\set{f(x) : x \in [x_{i-1}, x_i]}$ and
  $M_i = \sup\set{f(x) : x \in [x_{i-1}, x_i]}$.
  Furthermore;
  \begin{equation}
    L(f, P) = \sum_{i=1}^n m_i \Delta x_i
            = \sum_{i=1}^n 0 \cdot \Delta x_i
            = 0
  \end{equation}
  and
  \begin{equation}
    U(f, P) = \sum_{i=1}^n M_i \Delta x_i.
  \end{equation}
  Since the set of points where $f$ is nonzero has measure zero,
  as we make the partitions finer and finer, $M_i \Delta x_i$ will
  either be zero or approach $0$ for all $i$, so $U(f, P)$ will also approach $0$.
  On the other hand, $L(f, P)$ will always be $0$.
  Thus, we can make $U(f, P) - L(f, P)$ arbitrarily small,
  so $\displaystyle \int\limits_0^1 f(x) \d x = 0$.
\end{proof}
  % negative space
  % \vspace{-1.5em}
% \end{answer}
