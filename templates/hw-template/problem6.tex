\begin{problem}
  Let $g, f : \R \to \R$ be Riemann integrable on any interval $[a, b] \subset \R$.
  Is it true that $g \circ f$ is also Riemann integrable on any interval $[a, b] \subset \R$?\\
  \underline{Hint:} Consider $g$ such that $g(x) = 0$ if $x = 0$
  and $g(x) = 1$ if $x \neq 0$, and $f$ as in Problem \crim{$3$}:
  \[
    f(x) = \begin{cases}
      0 & \text{ if } x \in \R \setminus \Q \text{ (i.e. not rational).}\\
      \frac{1}{q} & \text{ if } x \colonequals \frac{p}{q} \in \Q
      \text{ with $p, q$ coprime and $q > 0$}.
    \end{cases}
  \]
\end{problem}

% \begin{answer}
  \begin{claim}
    The composition of two Riemann integrable functions is not necessarily Riemann integrable.
  \end{claim}
  \begin{proof}
    Consider the functions $g, f : \R \to \R$ as defined anove,
  and their composition $g \circ f$.
  Both $g$ and $f$ are Riemann integrable on any interval $[a, b] \subset \R$,
  as priorly shown. However, Let's look at $g \circ f$:

  \[
    (g \circ f)(x) = \begin{cases}
      0 & \text{ if } x \in \R \setminus \Q \text{ (i.e. not rational).}\\
      1 & \text{ if } x \colonequals \frac{p}{q} \in \Q
      \text{ with $p, q$ coprime and $q > 0$}.
    \end{cases}
  \]


  Note that $g \circ f$ is not continuous at any point $x \in \Q$,
  since there exists a rational number between any two distinct irrationals
  \footnote{
    \textbf{Proof that there exists a rational number between any two distinct irrationals.} \\
    Let $a, b \in \R \setminus \Q$ with $a < b$.
    Let $c = b - a$. By the properties of $\R$, there exists $n \in \N$
    such that $n > 1/c$, which implies that $cn > 1$.
    Since we took $c = b - a$, this implies that $nb - na > 1$.
    Therefore, there exists some integer $N$ such that $na < N < nb$.
    Dividing by $n$, we get $a < N/n < b$.
    Thus, $N/n$ is a rational number between $a$ and $b$. \\
  }
  and there exists an irrational number between any two distinct rationals
  \footnote {
    \textbf{Proof that there exists an irrational number between any two distinct rationals.} \\
    Let $a, b \in \Q$ with $a < b$.
    Then $b - a > 0$, $\displaystyle b - a > \frac{b-1}{\sqrt{2}}$,
    and $\displaystyle \frac{b - a}{\sqrt{2}} \not \in \Q$.
    Therefore $\displaystyle a + \frac{b - a}{\sqrt{2}} \in \R \setminus \Q$, and it is contained
    in the interval $(a, b)$. \\
  }.
  Consequently, if we take any $0 < \epsilon < 1$, then there is no value
  for $\delta > 0$ that satisfies the $\epsilon$-$\delta$ criterion for continuity
  at any point $x \in \R$ since. Take $x_2$ to be any number in the interval
  $(x - \delta, x + \delta)$, then:
  \begin{enumarabic}
    \item If $x \in \Q$ and $x_2 \in \R \setminus \Q$, then
      $(g \circ f)(x) = 1$ and $(g \circ f)(x_2) = 0$,
      so \[ \abs{(g \circ f)(x) - (g \circ f)(x_2)} = 1 > \epsilon. \]
    \item If $x \in \Q$ and $x_2 \in \Q$, then there exists some irrational number
      $x_3$ between $x$ and $x_2$, then $(g \circ f)(x) = 1$
      and $(g \circ f)(x_3) = 0$. Therefore,
      \[ \abs{(g \circ f)(x) - (g \circ f)(x_3)} = 1 > \epsilon. \]
    \item If $x \in \R \setminus \Q$ and $x_2 \in \Q$, then
      $(g \circ f)(x) = 0$ and $(g \circ f)(x_2) = 1$,
      so \[ \abs{(g \circ f)(x) - (g \circ f)(x_2)} = 1 > \epsilon. \]
    \item If $x \in \Q$ and $x_2 \in \Q$, then there exists some irrational number
      $x_3$ between $x$ and $x_2$, then $(g \circ f)(x) = 1$
      and $(g \circ f)(x_3) = 0$. Therefore,
      \[ \abs{(g \circ f)(x) - (g \circ f)(x_3)} = 1 > \epsilon. \]
  \end{enumarabic}

  \step
  Thus,  $g \circ f$ is not continuous at any point $x \in \R$,
  therefore not Riemann integrable on any interval $[a, b] \subset \R$.
\end{proof}
 
% \end{answer}
