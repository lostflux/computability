\begin{problem}
  Let $f : \R \to \R$ with
    \[
      f(x) = \begin{cases}
        0 & \text{ if } x \in \R \setminus \Q \text{ (i.e. not rational).}\\
        \frac{1}{q} & \text{ if } x \colonequals \frac{p}{q} \in \Q
        \text{ with $p, q$ coprime and $q > 0$}.
      \end{cases}
    \]
  Show that $\displaystyle \int\limits_0^1 f(x) \d x$ exists and is equal to $0$.\\
  \underline{Hint:} Use the Lebesgue criterion for integrability.
    In particular, you need to determine at what points $f$ is continuous.
\end{problem}

% \begin{answer}
  \begin{claim}
    $\displaystyle \int\limits_0^1 f(x) \d x$ exists.
  \end{claim}
  \begin{proof}
    We are given that $f$ is only nonzero for rational numbers of the form $p/q$,
    where $p, q \in \N$ and $p, q$ are coprime.
    Since the rational numbers are countable, $\Q$ has Lebesgue measure zero,
    meaning that the set of points where $f$ is nonzero,
    which is a subset of $\Q$, also has measure zero.
    By the Lebesgue criterion, $f$ is Riemann integrable on $[0, 1]$
    since the set of points where $f$ is discontinuous has measure zero,
    so $\displaystyle \int\limits_0^1 f(x) \d x$ exists.
  \end{proof}

  \begin{claim}
    $\displaystyle \int\limits_0^1 f(x) \d x = 0$.
  \end{claim}
  \begin{proof}
    Given a partition $P_n$ of $[0, 1]$ into $n$ subintervals,
    \blue{\[
      L(f, P) = \sum_{i=1}^n m_i \Delta x_i
              \leq \int \limits_0^1 f(x) \d x
              \leq \sum_{i=1}^n M_i \Delta x_i
              = U(f, P),
    \]}
    where $m_i = \inf\set{f(x) : x \in [x_{i-1}, x_i]}$ and
    $M_i = \sup\set{f(x) : x \in [x_{i-1}, x_i]}$.
    Furthermore;
    \begin{equation}
      L(f, P) = \sum_{i=1}^n m_i \Delta x_i
              = \sum_{i=1}^n 0 \cdot \Delta x_i
              = 0
    \end{equation}
    and
    \begin{equation}
      U(f, P) = \sum_{i=1}^n M_i \Delta x_i.
    \end{equation}
    Since the set of points where $f$ is nonzero has measure zero,
    as we make the partitions finer and finer, $M_i \Delta x_i$ will
    approach $0$ for all $i$, so $U(f, P)$ will also approach $0$
    (since only a smaller subset will have nonzero $M_i$).
    On the other hand, $L(f, P)$ will always be $0$.
    For any $\epsilon > 0$, take $P_\epsilon$ to be a partition such that
    $U(f, P_\epsilon) < \epsilon$, then $U(f, P_\epsilon) - L(f, P_\epsilon) < \epsilon$.
    Therefore, $\displaystyle \int\limits_0^1 f(x) \d x = 0$.
  \end{proof}
  % \end{answer}
