\section{Actions of Taft algebras}
\label{sect:taft}

Let now $n$,~$m\in\NN$ be such that $1<m$ and $m\mid n$, and let
$\lambda\in\kk^\times$ be a primitive $m$th root of unity. The
\newterm{generalized Taft algebra} $T=T_n(\lambda,m)$ is the algebra
freely generated by two letters $g$ and~$\xi$ subject to the relations
  \[ \label{eq:T:rels}
  g^n = 1,
  \qquad
  \xi^m = 0,
  \qquad
  g\xi = \lambda\xi g.
  \]
This algebra was originally studied by David Radford in~\cite{Radford}.
It is finite dimensional of dimension~$nm$, and the set $\{\xi^ig^j:0\leq
i<m,0\leq j<n\}$ is one of its bases. It is a Hopf algebra with
comultiplication $\Delta:T\to T\otimes T$ and augmentation
$\epsilon:T\to\kk$ such that
  \[
  \Delta(g) = g\otimes g,
  \qquad
  \Delta(\xi) = \xi\otimes1+g\otimes\xi,
  \qquad
  \epsilon(g) = 1,
  \qquad
  \epsilon(\xi) = 0.
  \]
When $n=m$ and $\theta=0$ this is the classical Taft Hopf algebra
constructed by Earl Taft in~\cite{Taft} with the purpose of exhibiting
finite dimensional Hopf algebras with antipode of arbitrarily large order.
We will use freely the Heyneman--Sweedler notation for the coproduct of~$T$
and even omit the sum: we will write $\Delta(a)$ in the form $a_1\otimes
a_2$.

\bigskip

We are interested in left $T$-module-algebra structures on our algebra~$A$,
that is, left $T$\nobreakdash-module structures $\lact:T\otimes A\to A$
such that the multiplication and unit map $A\otimes A\to A$ and $\kk\to A$
are both $T$-linear --- this means that $h\lact 1_A=\epsilon(h)1_A$ and
$h\lact ab =(h_1\lact a)(h_2\lact b)$ for all choices of~$a$ and~$b$
in~$A$ and $h$ in~$T$. We refer the reader to \cite{Montgomery}*{Chapter 4}
for general information about module-algebras over Hopf algebras.

We will further restrict our attention to $T$-module algebra structures
on~$A$ that are \newterm{inner-faithful} --- a notion introduced by
T.\,Banica and J.\,Bichon in~\cite{BB} ---  as these correspond to faithful
group actions on~$A$: the condition is that there be no non-zero Hopf
ideal~$I$ in~$T$ such that $I\lact A=0$. According to
\cite{Cline}*{Corollary 3.7}, in the specific case in which the Hopf
algebra is our generalized Taft algebra $T$ we have a handy criterion:

\begin{claim}\label{eq:faithful}
  A $T$-module-algebra structure on~$A$ is inner-faithful if and
  only if the group $G(T)=\gen{g}$ of group-like elements of~$T$ acts
  faithfully on~$A$ and $\xi\lact A\neq0$.
\end{claim}

As the algebra~$T$ is generated by~$g$ and~$\xi$ subject to the
relations in~\eqref{eq:T:rels}, giving a $T$-module structure on~$A$ is the
same as giving the two maps $\phi:a\in A\mapsto g\lact a\in A$ and
$\partial:a\in A\mapsto\xi\lact a\in A$ such that $\phi^n=\id_A$,
$\partial^m=0$ and $\phi\partial=\lambda\partial\phi$, and
that structure will be a $T$-module-algebra structure exactly when $\phi$ is
an automorphism of~$A$ and $\partial$ a $\phi$-twisted derivation $A\to A$.
The automorphism~$\phi$ will have finite order: according to
Proposition~\ref{prop:finite-subgroups}, up to conjugating the whole
module-algebra structure by an algebra automorphism of~$A$ we can suppose
then that there is a scalar~$\omega\in\kk$ such that $\phi(x)=\omega x$ and
$\phi(y)=\omega^{N-1} y$, and then the group of group-like
elements~$G(T)=\gen{g}$ of~$T$, which is cyclic of order~$n$, will clearly
act faithfully on~$A$ if and only if the scalar~$\omega$ is a primitive
$n$th root of unity.

We are left with the task of understanding the possibilities for the
map~$\partial$. Since it is a $\phi$-twisted derivation $A\to A$, we now
from Proposition~\ref{prop:der:xinner} that the map~$\partial$ will be, in
fact, the restriction of an inner $\tilde\phi$-twisted derivation of the
localization~$A_x$ that preserves~$A$. The following lemma imposes
significant restrictions on what can actually happen.

\begin{Lemma}\label{lemma:partial:inner}
Let $\omega$ be a primitive $n$th root of unity in~$\kk$, let $\phi:A\to A$
be the algebra automorphism such that $\phi(x)=\omega x$ and
$\phi(y)=\omega^{N-1}y$, and let $\tilde\phi:A_x\to A_x$ be the unique
extension of~$\phi$ to the localization~$A_x$. Let $u$ be a non-zero
element of~$A_x$ such that the inner $\tilde\phi$-twisted derivation
$\ad_{\tilde\phi}(u):A_x\to A_x$ preserves the subalgebra~$A$ and let
  \[
  \partial\coloneqq\ad_{\tilde\phi}(u)|_A:A\to A
  \]
be its restriction to~$A$, which is a $\phi$-twisted derivation.
\begin{thmlist}

\item The map~$\partial$ is non-zero.

\item If $\lambda$ is a scalar, then we have that
$\phi\partial=\lambda\partial\phi$ exactly when $\tilde\phi(u)=\lambda u$,
and when that is the case we have that $\lambda^n=1$.

\item If there are a scalar~$\lambda$ and a positive integer~$m$
such that $\tilde\phi(u)=\lambda u$ and $\partial^m=0$, then $n\leq m$.

\end{thmlist}
\end{Lemma}

\begin{proof}
\thmitem{1} We have $\partial(x)=ux-\omega xu=-\tilde\alpha(u)\neq0$,
according to Lemma~\ref{lemma:comm:phi}.

\thmitem{2} If $\lambda\in\kk$ is such that
$\phi\partial=\lambda\partial\phi$, then
  \begin{align}
  \tilde\phi(u)\omega x - \omega^2 x\tilde\phi(u)
       &= \tilde\phi(ux-\omega xu) 
        = \phi(ux-\omega xu) 
        = \phi(\partial(x))
        = \lambda\partial(\phi(x)) \\
       &= \lambda\partial(\omega x) 
        = \lambda(u\omega x-\omega^2xu),
  \end{align}
and therefore
  \[
  \tilde\alpha\bigl(\tilde\phi(u)-\lambda u\bigr) 
        = \omega x\bigl(\tilde\phi(u)-\lambda u\bigr)
          - \bigl(\tilde\phi(u)-\lambda u\bigr)x 
        = 0,
  \]
so that $\tilde\phi(u)=\lambda u$ because of Lemma~\ref{lemma:comm:phi}.
Conversely, if $\lambda$ is a scalar such that $\tilde\phi(u)=\lambda u$,
then a simple and direct calculation shows that we have that
$\phi\partial=\lambda\partial\phi$. Moreover, since the
automorphism~$\tilde\phi$ is manifestly diagonalizable and all its
eigenvalues are powers of~$\omega$, in that case we have that
$\lambda^n=1$.

\thmitem{3} Let us suppose that there are $\lambda\in\kk$ and $m\in\NN$
such that $\tilde\phi(u)=\lambda u$ and $\partial^m=0$. Since $u\neq0$,
there is an integer~$d\in\NN_0$ such that $u\in\tilde F_d\setminus\tilde
F_{d-1}$. An induction shows that if $l\in\NN_0$, $a\in A$ and $t\in\kk$
are such that $a\in F_l\setminus F_{l-1}$ and $\phi(a)=ta$, then for all
$k\in\NN_0$ we have that
  \[
  \partial^k(a) \equiv \prod_{i=0}^{k-1}(1-t\lambda^i)\cdot au^k
        \mod \tilde F_{l+kd-1}
  \]
and, in particular, since $\partial^m(a)=0$, that
  \[
  \prod_{i=0}^{m-1}(1-t\lambda^i)\cdot au^m \in \tilde F_{l+md-1}.
  \]
As the graded algebra~$\gr A_x$ for the filtration~$(\tilde F_i)_{i\geq-1}$
is an integral domain, this tells us that, in fact,
  \[
  \prod_{i=0}^{m-1}(1-t\lambda^i) = 0
  \]
and, then, that $t\in\{\lambda^{-i}:0\leq i<m\}$. In particular, for each
$j\in\{0,\dots,n-1\}$ we can take $a=x^j$, that has $\phi(a)=\omega^j a$,
and conclude that the $n$ pairwise different scalars
$\omega^0$,~\dots,~$\omega^{n-1}$ all belong to the set
$\{\lambda^{-i}:0\leq i<m\}$. This set has cardinal at most $m$, so
$n\leq m$, as the lemma claims.
\end{proof}

The obvious question after having proved Lemma~\ref{lemma:partial:inner}
is: when is the map~$\partial$ appearing there such that $\partial^n=0$?
The answer is simple: never. With the objective of proving this, let us
recall some standard notations from the theory of $q$-variants. If $q$ is a
variable, then for each $n\in\NN_0$ we let $[n]_q\coloneqq
1+q+\cdots+q^{n-1}$ and $[n]_q!\coloneqq[1]_q[1]_q\cdots[n]_q$, and
consider for each choice of $k$ and~$i$ in~$\NN_0$ such that $0\leq i\leq
k$ the \newterm{$q$-binomial} or \newterm{Gaussian binomial coefficient}
  \[
  \binom{k}{i}_q \coloneqq \frac{[k]_q!}{[i]_q!\cdot[k-i]_q!},
  \]
which is an element of~$\ZZ[q]$.

\begin{Lemma}\label{lemma:non-zero}
Let $\omega$ and~$\lambda$ be primitive $n$th roots of unity in~$\kk$, let
$\phi:A\to A$ be the algebra automorphism such that $\phi(x)=\omega x$ and
$\phi(y)=\omega^{N-1}y$, and let $\tilde\phi:A_x\to A_x$ be the unique
extension of~$\phi$ to the localization~$A_x$. If $u$ is a non-zero
element of~$A_x$ such that $\tilde\phi(u)=\lambda u$ and the inner
$\tilde\phi$-twisted derivation $\ad_{\tilde\phi}(u):A_x\to A_x$ preserves
the subalgebra~$A$, then the restriction
  \(
  \partial\coloneqq\ad_{\tilde\phi}(u)|_A:A\to A
  \),
which is a $\phi$-twisted derivation, has $\partial^n\neq0$.
\end{Lemma}

\begin{proof}
Let $u$ be a non-zero element of~$A_x$ such that $\tilde\phi(u)=\lambda u$
and $\ad_{\tilde\phi}(u)(A)\subseteq A$, and let us consider the
$\phi$-twisted derivation $\partial\coloneqq\ad_{\tilde\phi}(u)|_A:A\to A$.

Let $a$ in~$A$ and $t\in\kk$ be such that $\phi(a)=ta$. For all $k\in\NN_0$
we have that
  \[ \label{eq:pka}
  \partial^k(a) 
        = \sum_{i=0}^k (-1)^{k-i}
                       \lambda^{\binom{i}{2}}
                       \binom{k}{i}_\lambda 
                       t^i u^{k-i}au^i,
  \]
as can be proved by an obvious induction using the well-known analogue of
Pascal's identity,
  \[
  \binom{k}{i}_q = \binom{k-1}{i}_q + q^{k-i}\binom{k-1}{i-1}_q,
  \]
valid in~$\ZZ[q]$ whenever $0<i<k$. In particular, taking $k=n$
in~\eqref{eq:pka} we find that
  \[ \label{eq:pka:1}
  \partial^n(a) 
        = \sum_{i=0}^n 
                (-1)^{n-i}
                \lambda^{\binom{i}{2}}
                \binom{n}{i}_\lambda 
                t^i u^{n-i}au^i.
  \]
Now, according to the Cauchy $q$-binomial theorem we have that
  \[
  \prod_{i=0}^n (y+zq^i) = \sum_{i=0}^nq^{\binom{i}{2}}\binom{n}{i}_qy^{n-i}z^i
  \]
in the polynomial ring~$\ZZ[q,y,z]$. Extending scalars to~$\kk$ and
specializing $z$ at~$-1$ and $q$ at~$\lambda$, the left hand side of this
equality becomes $y^n-1$, so by looking at the coefficients of the powers
of~$y$ in the right hand side we see that $\binom{n}{i}_\lambda=0$ when
$0<i<n$. The equality~\eqref{eq:pka:1} therefore violently simplifies to
  \[
  \partial^n(a) = \lambda^{\binom{n}{2}}au^i + (-1)^nu^na,
  \]
since $t^n=1$, as $t$ is an eigenvalue of the automorphism~$\phi$ and thus
a power of~$\omega$. As $\lambda$ is a primitive $n$th root of unity,
we have that $\lambda^{\binom{n}{2}}=(-1)^{n+1}$ and, therefore, that
  \[
  \partial^n(a) = (-1)^n[u^n,a].
  \]

This equality holds whenever $a$ is an eigenvector of~$\phi$ in~$A$ and,
since the automorphism~$\phi$ is diagonalizable, it then follows
immediately that the equality holds in fact for all~$a$ in~$A$. In
particular, the map~$\partial^n$ is zero exactly when the element $u^n$
of~$A_x$ commutes with all elements of~$A$: this happens if and only if
$u^n$ is central in~$A_x$, so an element of~$\kk$ according to
Proposition~\ref{prop:center:x}, and clearly this can happen if and only if
$u\in\kk$. Now, as $\lambda\neq1$ and $\phi(u)=\lambda u$, we certainly
have that $u$ is not in~$\kk$, so $\partial^n\neq0$, as the lemma claims.
\end{proof}

\begin{Remark}
The vanishing of the Gaussian binomial coefficients at well-chosen roots of
unity that we used in this proof is a very special case of the $q$-Lucas
theorem, and it was in fact in that way that we originally proceeded. The
simpler recourse to the $q$-binomial theorem that we used above was
suggested by a comment of Richard Stanley on MathOverflow~\cite{A}. Let
us note that the $q$-Lucas theorem was first proved by Gloria Olive
in~\cite{Olive}, where the theorem appears as Equation (1.2.4). This
beautiful result, which generalizes the classical Lucas theorem for
binomial coefficients, has also been proved by Jacques Désarménien
in~\cite{Desarmenien}, by Volker Strehl in~\cite{Strehl}, by Bruce Sagan
in~\cite{Sagan}, and Donald E.\ Knuth and Herbert S.\ Wilf describe
in~\cite{KnW} a factorization of the Gaussian binomial coefficient
$\binom{n}{i}_q$ with $0<i<n$  which certainly includes the $n$th
cyclotomic polynomial among the factors.
\end{Remark}

After all this work we can state and proof the following markedly
disappointing result:

\begin{Proposition}\label{prop:taft-actions}
Let $n$ and~$m$ be integers such that $1<m$ and $m\mid n$, and let
$\lambda\in\kk^\times$ be a primitive $m$th root of unity in~$\kk$. There
is no inner-faithful action of the generalized Taft algebra
$T_n(\lambda,m)$ on~$A$.
\end{Proposition}

\begin{proof}
Let us suppose, in order to reach a contradiction, that there is an
inner-faithful module-algebra action of the Hopf algebra $=T_n(\lambda, m)$
on~$A$, and let us consider the maps $\phi:a\in A\mapsto g\lact a\in A$ and
$\partial:a\in A\mapsto \xi\lact a\in A$. The relations that define the
algebra~$T$ imply that $\phi^n=\id_A$, $\partial^m=0$ and
$\phi\partial=\lambda\partial\phi$, and the fact that what we have is a
module-algebra structure that $\phi$ is an automorphism and $\partial$ a
$\phi$-twisted derivation. 

The criterion~\eqref{eq:faithful} for inner-faithfulness from Cline's paper
\cite{Cline} implies that the order of~$\phi$ is exactly~$n$, since $g$ has
order~$n$ in~$T$, and then, according to
Proposition~\ref{prop:finite-subgroups}, up to conjugating the action
of~$T$ on~$A$ by an algebra automorphism we can suppose that there is a
primitive $n$th root of unity $\omega$ in~$\kk$ such that $\phi(x)=\omega
x$ and $\phi(y)=\omega^{N-1}y$. Proposition~\ref{prop:der:xinner} now tells
us that there is an element~$u$ in the localization~$A_x$ such that the
$\tilde\phi$-twisted derivation $\ad_{\tilde\phi}(u):A_x\to A_x$ preserves
the subalgebra~$A$ of~$A_x$ and the map $\partial$ coincides with its
restriction $\ad_{\tilde\phi}(u)|_A$ to~$A$. The second part of
Lemma~\ref{lemma:partial:inner} tells us tat $\tilde\phi(u)=\lambda u$ and,
since $\partial^m=0$, the third part of that lemma that $n\leq m$. Of
course, since $m$ divides $n$ we have in fact that $n=m$, and we are
therefore in the situation of Lemma~\ref{lemma:non-zero}: this is
absurd, for the lemma tells us that $\partial^n\neq0$.
\end{proof}

In~\cite{Radford} Radford considers a more general class of generalized
Taft algebras: given two positive integers $n$,~$m\in\NN$ such that $1<m$
and $m\mid n$, a primitive $m$th root of unity~$\lambda$ in~$\kk$, and an
arbitrary scalar~$\tau\in\kk$, he lets $T_n(\lambda,m,\tau)$ be the algebra
freely generated by two letters~$g$ and~$\xi$ subject to the relations
  \[ 
  g^n = 1,
  \qquad
  \xi^m = \tau(g^m-1),
  \qquad
  g\xi = \lambda\xi g,
  \]
which is a Hopf algebra with respect to the comultiplication and counit such that
  \[
  \Delta(g) = g\otimes g,
  \qquad
  \Delta(\xi) = \xi\otimes1+g\otimes\xi,
  \qquad
  \epsilon(g) = 1,
  \qquad
  \epsilon(\xi) = 0.
  \]
When $\tau=0$ we obtain the algebras we considered above, since
$T_n(\lambda,m,0)=T_n(\lambda,m)$, and the connection between the two
families is that $T_n(\lambda,m,0)$ is the graded Hopf algebra
corresponding to the coradical filtration of the pointed Hopf
algebra~$T_n(\lambda,m,\tau)$ --- we refer to Section 5.2 in Montgomery's
book \cite{Montgomery} for information of this. It is a natural guess,
after our last proposition, that there is also no inner-faithful
module-algebra action of these more general Hopf algebras on our
algebra~$A$, although a new idea is needed to verify this. For example, the
twisted derivation~$\partial$ that corresponds to the element~$\xi$ in an action
of~$T_n(\lambda,m,\tau)$ is locally finite --- in the sense that every
element of~$A$ is contained in a finite-dimensional subspace that is
invariant under~$\partial$ --- so classifying locally-finite twisted
derivations of~$A$ could well be helpful.
