\phantomsection
\addcontentsline{toc}{section}{Introduction}%
\section*{Introduction}

% In this paper we fix a field~$\kk$ of characteristic zero and a
% non-negative integer~$N$, and study the algebra~$A_N$ freely
% generated by two letters~$x$ and~$y$ subject to the relation
%   \[
%   yx-xy = x^N.
%   \]
% with the objective of computing as explicitly as it is possible (to us!)
% some of its invariants of homological nature.

A Turing category is a category $C$ equipped with:
\begin{itemize}
  \item cartesian products --- to pair (the codes of) data and programs,
  \item a notion of partiality --- to represent programs (morphisms)
    which do not necessarily halt,
  \item and a Turing object $A$ --– to represent the ``codes'' of all programs.
\end{itemize}

Turing categories provide an abstract framework for computability:
a ``category with partiality'' equipped with a ``universal computer'',
whose programs and codes thereof constitute the objects of interest.
\cite{AD}
