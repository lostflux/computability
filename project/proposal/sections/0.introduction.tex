\phantomsection
\addcontentsline{toc}{section}{Project Proposal}%
\section*{Project Proposal}

A Turing category is a category $C$ equipped with:
\begin{itemize}
  \item cartesian products --- to pair (the codes of) data and programs,
  \item a notion of partiality --- to represent programs (morphisms)
    which do not necessarily halt,
  \item and a \newterm{Turing object} $A$ --– to represent the
    ``codes'' of all programs.
    A Turing object is an object $A$ such that for any $X, Y \in C$,
    there is a universal application morphism
    $\tau_{X, Y} : A \times X \to Y$
    that represents the application of a program (in $A$) to data (in $X$)
    to produce a result (in $Y$).~\cite{TURING-CATEGORIES}
\end{itemize}

Turing categories provide an abstract framework for computability:
a ``category with partiality'' equipped with a ``universal computer'',
whose programs and codes thereof constitute the objects of interest.
\cite{TURING-CATEGORIES}

I am interested in doing an expository study on
Turing categories with the goal of explaining
the main ideas it builds on
and demonstrating some of the main
results it achieves.

My main references will be:

\begin{enumarabic}
  \item \newterm{Basic category theory}~\cite{CATEGORY-THEORY-TEXT},
    a textbook by Tom Leinster accessible on the internet
    \href{https://arxiv.org/abs/1612.09375}{here}.
  \item Introductory paper on \newterm{Turing Categories}
    ~\cite{TURING-CATEGORIES},
    accessible on the internet \href{https://doi.org/10.1016/j.apal.2008.04.005}{here}.
  \item Introductory notes on
    \newterm{Effective Applicative Structures}~\cite{APPLICATIVE-STRUCTURES},
    accessible on the internet
    \href{https://link.springer.com/chapter/10.1007/3-540-60164-3_21#citeas}{here}.
  \item Enderton's textbook on mathematical logic~\cite{ENDERTON},
    accessible on the internet \href{https://doi.org/10.1007/978-3-319-20451-2}{here}
    and discussed \href{https://www.logicmatters.net/tyl/booknotes/enderton/}{here}.
\end{enumarabic}

The final deliverable will be a paper (\LaTeX)
that explains the main ideas of Turing categories and
demonstrates how it can be used to derive some of the main results
in computability.

I am thinking of two possible directions for the presentation:
\begin{enumarabic}
  \item A video presentation of the paper using \verb|manim|,
    a Python library for making mathematical animations
    (developed by Grant Sanderson of \verb|3Blue1Brown|)..
    I've experimented a little with the library and
    I think it could be an exciting challenge.
    More information on \verb|manim| can be found
    \href{https://docs.manim.community/en/stable/tutorials/quickstart.html}{here}.
  \item If the video doesn't work out, my alternative will be presenting
    the paper in class.
\end{enumarabic}

\clearpage
