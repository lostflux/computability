\phantomsection
\addcontentsline{toc}{section}{Introduction}%
\section*{Introduction}

In this paper, we construct a turing category
$\kk$ and study the resulting implications
on computability.

In mathematics, it is often the case that structures
present in one field are closely mirrored in another.
For example, the idea of sets, maps and bijective maps
in set theory is mirrored by the notion of
groups, homomorphisms and isomorphisms in group theory,
topological spaces, continuous maps and homeomorphisms
in topology, vector spaces, linear maps and linear isomorphisms
in linear algebra, and so on.
\begin{alignat*}{5}
  &\underbrace{S}_{set} \xlto{f} \underbrace{T}_{set}
  \qquad&&\qquad
  &&\underbrace{G}_{group} \xlto{\varphi} \underbrace{H}_{group}
  \qquad&&\qquad
  \underbrace{X}_{topological space} \xlto{\sigma} \underbrace{Y}_{topological space}
\end{alignat*}

Category theory as a field of mathematics studies such structures
and relationships between them, with the aim that,
often, withdrawing from the specifics within a particular field
can reveal more general shared structures, properties, and relationships~\cite{BRADLEY}.

In this paper, we explore some of the core concepts in category theory,
then hone in on the notion of a turing category---an abstract
model of computation based on category theory---and study some
of its properties and their implications on computability.

% \clearpage
