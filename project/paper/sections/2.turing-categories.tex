\section{Turing Categories}

A Turing category is a cartesian restriction category
$\cat{T}$ equipped with:
\begin{enumarabic}
  \item cartesian products --- to pair (the codes of) data and programs;
  \item a restriction structure representing the
    notion of partiality --- for represent programs (morphisms)
    which do not necessarily halt;
  \item and a \newterm{Turing object} $A$ --– to represent the
    ``codes'' of all programs.
    A Turing object is an object $A$ such that for any $X, Y \in \cat{T}$,
    there is a universal application morphism
    $\tau_{X, Y} : A \times X \to Y$
    that represents the application of a program (in $A$) to data (in $X$)
    to produce a result (in $Y$).~\cite{TURING-CATEGORIES}
    % $\tau_{X, Y}$ is referred to as a \newterm{Turing morphism}.
\end{enumarabic}

Turing categories provide an abstract framework for computability:
a ``category with partiality'' equipped with a ``universal computer'',
whose programs and codes thereof constitute the objects of interest.
\cite{TURING-CATEGORIES}

\clearpage
\subsection{Basic Properties of Turing Categories}

\begin{definition}
  Given two objects $A, B \in \cat{C}$, $A$ is a \newterm{retract} of $B$
  if there exist morphisms $s : A \to B$ and $r : B \to A$ such that
  $r \circ s = \id_A$.
  $s$ is called a \newterm{section} and $r$ is called a \newterm{retraction}.
  \[
    \begin{tikzcd}
      A \arrow[r, "s", bend left] &B \arrow[l, "r", bend left]
    \end{tikzcd}
  \]
\end{definition}

\begin{lemma}
  In a Turing category $\cat{C}$ with a Turing object $A$,
  every object $B \in \cat{C}$ is a retract of $A$.
\end{lemma}

\begin{examples}
  Here are some examples of Turing categories:

  \begin{enumarabic}
    \item The classical recursion category $\cat{R}$,
      where objects are sets and morphisms are partial computable (``recursive'') functions.
      Since Turing machine (or register-machine)--computable functions are exactly
      the partial computable functions, one may consider the codes of Turing machines
      corresponding to the partial computable functions as the objects of $\cat{R}$,
      with $\varphi_i : \N \to \N$ representing the machine with code $i$.~\cite{TURING-CATEGORIES}
      When $f = \varphi_e$, we say that $e$ is a \emph{code} for $f$.

      Key properties of classical recursion include:
      \begin{enumalph}
        \item The existence of a universal partial computable function
          $\Phi$ such that for each $e \in \N$,
          \[
            \Phi(e, x_1, x_2, \ldots, x_n) = \varphi_e(x_1, x_2, \ldots, x_n).
          \]
        \item The \textbf{s-m-n Theorem}: There are computable and injective functions
          $s_m^n$ for each $m, n > 0$ such that
          \[
            \varphi_e(x_1, x_2, \ldots, x_m, y_1, y_2, \ldots, y_n)
            = \varphi_{s_m^n(e, x_1, x_2, \ldots, x_m)}(y_1, y_2, \ldots, y_n).
          \]
      \end{enumalph}

      Define \begin{align*}
        \dotp : \N \times \N &\to \N \\
        (e, x) &\mapsto \varphi_e(x).
      \end{align*}

      \def \Comp {\mathsf{Comp}}
      Consider the category $\Comp(\N)$ with the following properties:
      \begin{enumalph}
        \item $\ob{\Comp(\N)} = \set{ \N^i \given i \in \N }$;
        \item $f : N^k \to \N^m$ is an $m$-tuple of partial computable functions
          of $k$ variables each.
      \end{enumalph}
  \end{enumarabic}
\end{examples}

Some of the key results in computability theory
carry over to Turing categories, including
the following.

\begin{theorem}
  (smn) For $\varphi : A \to B$ partial computable, there exists a
  partial computable function $s : \N \to \N$ such that
  $\varphi(\vector{i, n}) = \varphi_{s(i)}(n)$ for all $x \in \dom{\varphi}$.
\end{theorem}


