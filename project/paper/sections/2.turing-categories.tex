\section{Turing Categories}

A \emph{turing category} is a specific categorification\footnote[1]{
  Categorification is the generalization of set-theoretic
  structures and properties to category theory
  (or from category theory to higher category theory).
}
of \newterm{partial combinatory algebras}\footnote[2]{
  A partial combinatory algebra is a generalization
  of untyped lambda calculus (equivalently, Turing machines),
  allowing for application to be only partially defined.
}
based on restriction categories.
Precisely, a Turing category is a cartesian restriction category
$\cat{T}$ equipped with:
\begin{enumarabic}
  \item cartesian products --- to pair (the codes of) data and programs;
  \item a restriction structure representing the
    notion of partiality, since programs may not terminate on all inputs;
  \item and a \newterm{Turing object} $A$ --– as some execution context
    or universal machine that understands and executes codes (morphisms)
    within the category.
    For any $X, Y \in \ob{\cat{T}}$,
    there is a universal application morphism
    $\tau_{X, Y} : A \times X \to Y$
    that represents the application of a program (in $A$) to data (in $X$)
    to produce a result (in $Y$).~\cite{TURING-CATEGORIES}
    \def \TM {\mathsf{TM}}
    \begin{remark}
      For example, in a Turing category $\TM$ of Turing machines,
      the Turing object $A$ would be a universal Turing machine
      that takes as input some code $e$ and data $x$,
      and simulates $\varphi_e(x)$.
    \end{remark}
\end{enumarabic}

Turing categories provide an abstract framework for computability:
a ``category with partiality'' equipped with a ``universal computer'',
whose codes constitute the objects of interest.
\cite{TURING-CATEGORIES}

% \clearpage
\subsection{Properties of Turing Categories}

\begin{definition}
  Given two objects $A, B \in \cat{C}$, $A$ is a \newterm{retract} of $B$
  if there exist morphisms $s : A \to B$ and $r : B \to A$ such that
  $r \circ s = \id_A$.
  \[
    \begin{tikzcd}
      A \arrow[r, "s", bend left] &B \arrow[l, "r", bend left]
    \end{tikzcd}
  \]
  In this case, $s$ is called a \newterm{section} or \emph{right inverse} of $r$,
  and $r$ is called a \newterm{retraction} or \emph{left inverse} of $s$.
  \cite{APPLICATIVE-STRUCTURES}
\end{definition}

\begin{lemma}
  In a Turing category $\cat{C}$ with a Turing object $A$,
  then $A$ is a universal object in the sense that
  every object $B \in \cat{C}$ is a retract of $A$.
  \begin{proof}
    To show that every object $B \in \cat{C}$ is a retract of $A$,
    we need to exhibit a section $s : B \to A$ and a retraction $r : A \to B$.
    Since $A$ is a Turing object, there exists a morphism
    $\tau_{B, A} : A \times B \to A$ that represents the application
    of a program (in $A$) to data (in $B$) to produce a result (in $A$).
    Define $s = \pi_2 : B \to A$ and $r = \pi_1 : A \to B$
    such that $r \circ s = \pi_1 \circ \pi_2 = \id_A$. \\
    \crim{revisit}
  \end{proof}
\end{lemma}

% Some of the key results in computability theory
% carry over to Turing categories, including
% the following.

% \begin{theorem}
%   (smn) For $\varphi : A \to B$ partial computable, there exists a
%   partial computable function $s : \N \to \N$ such that
%   $\varphi(\vector{i, n}) = \varphi_{s(i)}(n)$ for all $x \in \dom{\varphi}$.
% \end{theorem}

\begin{theorem}
  Given a partial combinatory algebra $\A$,
  the computable maps in $\A$ form a Turing category.

  \begin{proof}
    The proof is by construction.
    Given a partial combinatory algebra $\A$,
    we construct a Turing category $\cat{T}$ as follows:
    \begin{enumarabic}
      \item The objects of $\cat{T}$ are the elements of $\A$.
      \item The morphisms of $\cat{T}$ are the computable maps in $\A$.
      \item The Turing object $A$ is the universal machine in $\A$.
      \item The application morphism $\tau_{X, Y} : A \times X \to Y$
        is the application of a program (in $A$) to data (in $X$)
        to produce a result (in $Y$).
    \end{enumarabic}
    The resulting category $\cat{T}$ is a Turing category.
  \end{proof}
\end{theorem}

% This theorem means that whenever we have a construct with (partial) computable maps
% or functions and a known way of applying these functions to data,
% then we can construct a Turing category around this construct
% by imposing extra structure.

\begin{theorem} (Recognition Criterion for Turing Categories)
  Given a category $C$, the following are equivalent:
  \begin{enumarabic}
    \item $C$ is a Turing category.
    \item There exists an object $A \in \ob{C}$ of which every other
      Turing object is a retract
  \end{enumarabic}
\end{theorem}

\step
% \begin{examples}
\subsection{Examples of Turing Categories}
  % Here are some examples of Turing categories:

\subsubsection{Classical Recursion Category}

  Consider an enumeration
  of partial computable functions $f : \omega \to \omega$ as
  $\varphi_0, \varphi_1, \varphi_2, \ldots$
  since each such function consists of a finite number of
  instructions that can be encoded as a natural number.
  Alternatively, since the partial computable functions
  are exactly the Turing machine-computable ones, one may
  consider a coding for Turing machines.

  Similarly, one can consider an enumeration of $n$-ary functions
  $f : \omega^n \to \omega$ as $\varphi_0^n, \varphi_1^n, \varphi_2^n, \ldots$.

  We define the classical recursion category $\cat{R}$ with:
  \begin{enumarabic}
    \item $\ob{\cat{R}} = \set{\omega, \omega^2, \omega^3, \ldots}$;
    \item The Turing object is a universal machine $\Phi$;
    \item Morphisms are codes for partial computable (``recursive'') functions
      with each $i$ representing $\varphi_i$.
  \end{enumarabic}

  To distinguish between unary, binary, ternary, etc., functions,
  we denote $\Phi^{(n)}(e, x_1, \ldots, x_n)$ when the universal machine
  $\Phi$ interprets $e$ as a code for an $n$-ary function
  and applies it to the arguments $x_1, \ldots, x_n$.

  \begin{remark}
    For the category to be well-formed, the coding scheme must be consistent.
    For example, if the Turing object interprets codes as Turing machines,
    then each code must encode a turing machine.
    Mixing codes for Turing machines with codes for register machines
    or codes for function instructions,
    for example, would break the structure of the category.
  \end{remark}
  
  \begin{remark}
    When convenient (and purely for semantic reasons), we sometimes interchange
    the functions and machines as the morphisms in $\cat{R}$.
  \end{remark}

  Key properties of classical recursion include:
  \begin{enumalph}
    \item By the construction of the category, the \textbf{\emph{Universality Theorem}}
      holds. That is: for each $e, n \in \omega$,
      \[
        \Phi^{(n)}(e, x_1, x_2, \ldots, x_n) = \varphi_e(x_1, x_2, \ldots, x_n).
      \]
    \item The \textbf{\emph{s-m-n Theorem}} also holds.
      That is, there exists computable and injective functions
      $s_m^n$ for each $m, n \in \omega$ such that
      \[
        \Phi^{(m+n)}(x_1, x_2, \ldots, x_m, y_1, y_2, \ldots, y_n)
        = \Phi^{(n)}(s_m^n(e, x_1, x_2, \ldots, x_m), y_1, y_2, \ldots, y_n).
      \]
  \end{enumalph}

\subsubsection{Variations of Classical Recursion}

A slight variation of the classical recursion category above is obtained
when we consider the category obtained by considering
the c.e. subsets of $\omega^n, k > 0$ as objects
and tuples of partial computable functions as morphisms.
This category is denoted $\cat{R}_n$.

% \subsubsection{Reflexive Turing Category}

%   \begin{definition}
%     A \newterm{reflexive object} in a category $\cat{C}$ is an object $A$
%     such that there exists a morphism $r : A \to A$ such that $r \circ r = r$.
%   \end{definition}

\subsubsection{A Non-Example}

  When does a category not form a Turing category?
  One of the important requirements is that the morphisms
  within the category have a \emph{suitable} enumeration.
  For example, consider the subcategory of total computable functions
  within the classical recursion category.
  This subcategory does not form a Turing category because no suitable
  enumeration can be constituted for the total functions.

  \begin{proof}
    Suppose we enumerate the total computable functions as
    $\varphi_0, \varphi_1, \varphi_2, \ldots$,
    Then, by diagonalization, we can always find a total computable function
    not in the enumeration: $f(e) = \varphi_e(e) + 1$.
  \end{proof}

\subsection{(Some) Applications of Turing Categories}

\subsubsection{Turing Equivalences}

  \begin{lemma}~\label{lemma:turing-equivalence}
    Given $A$ is a Turing object in $C$, then $B$ is a Turing object
    if and only if $A$ is a retract of $B$.

    \begin{proof}
      Since all Turing-complete models of computation are equivalent, 
      given any two Turing objects $A$ and $B$,
      there exists some computable isomorphism between them.

      For example, for each register machine simulating some
      function $f$, there exists
      some Turing machine that simulates the same
      sequence of instructions needed
      to compute $f$.

      For each $i \in \omega$, suppose $\phi_i$ is the Turing machine
      with coding $i$, and $\psi_i$ is the register machine with coding $i$,
      then for each $\phi_i$ there exists some $\psi_k$
      (with $k$ not necessarily equal to $i$) such that $\phi_i \equiv \psi_k$.
      Define $s$ and $t$ such that $s(\phi_i) = \psi_k$ and $t(\psi_k) = \phi_i$.
      Then, $s$ and $t$ satisfy the conditions for a section and retraction.
    \end{proof}

    This means that the coding scheme for any Turing object
    has to be consistent with the coding scheme for other valid
    Turing objects, since the models of computation are equivalent.

    Furthermore, even when a category contains a finite finite number
    of codes,
    the Turing object itself must be capable of representing
    an infinite number of codes for it to be Turing-complete.
  \end{lemma}

  \step
  \begin{definition}
    Two categories $\cat{C}$ and $\cat{D}$ are \newterm{equivalent}
    if there exists a pair of functors $F : \cat{C} \to \cat{D}$ and
    $G : \cat{D} \to \cat{C}$ such that $F \circ G \cong \id_{\cat{D}}$
    and $G \circ F \cong \id_{\cat{C}}$, and all the laws of functors
    are satisfied.
  \end{definition}

  Following from Lemma~\ref{lemma:turing-equivalence},
  any two Turing categories $A$ and $B$ encoding the same set
  of morphisms are equivalent. This means that
  whenever $\ob{A} \equiv \ob{B}$\footnote[1]{
    Not necessarily the same codes, but simulating the same functions.
  } then all properties consistent in $A$ are also consistent in $B$.

  Turing equivalences provide a way to compare the computability
  of different structures and constructs, and effectively
  equate different models of computation.
