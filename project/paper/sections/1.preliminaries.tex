\section{Preliminaries}

% In mathematics, it often pays to study constucts and results from



% a construct or result from one field
% has surprising utility in a different field. For example;

% \begin{itemize}

%   % non-euclidean geometry and relativity

%   % galois theory unsolvability of quintic
%   \item Galois 
% \end{itemize}

\subsection{Categories}

\begin{definition}
  A \newterm{category} $\cat{A}$ consists of:
  \begin{enumarabic}
    \item A collection $\ob{\cat{A}}$ of objects;
    \item For each pair of objects $A,B\in\ob{\cat{A}}$, a set
      $\cat{A}(A,B)$ of \newterm{arrows} or \newterm{morphisms}
      or \newterm{maps} from $A$ to $B$;
    \item For each $A, B, C \in \ob{\cat{A}}$, a function
      \begin{align*}
        \circ_{A,B,C} : \cat{A}(B,C) \times \cat{A}(A,B) &\to \cat{A}(A,C) \\
        (f,g) &\mapsto f \circ g
      \end{align*}
      called \newterm{composition};
      where $(f \circ g)(x) = f(g(x))$ for all $x \in A$.
    \item For each $A \in \ob{\cat{A}}$, an arrow
      $\id_A \in \cat{A}(A,A)$ called the \newterm{identity} on $A$;
  \end{enumarabic}
  such that the following axioms hold:
  \begin{enumarabic}
    \item \textbf{associativity}: for all $f \in \cat{A}(A,B)$,
      $g \in \cat{A}(B,C)$, and $h \in \cat{A}(C,D)$,
      $(h \circ g) \circ f = h \circ (g \circ f)$.
    \item \textbf{identity laws}: for all $f \in \cat{A}(A,B)$,
        $f \circ \id_A = f = \id_B \circ f$.
  \end{enumarabic}
\end{definition}

\begin{remark}
  As simplifications, we write:
  \begin{enumalph}
    \item $A \in \cat{A}$ to mean $A \in \ob{\cat{A}}$;
    \item $f : A \to B$ or $A \xlongrightarrow{f} B$ to mean
      $f \in \cat{A}(A,B)$;
    \item $fg$ for $f \circ g$;
  \end{enumalph}
\end{remark}


% \setmathfont[range={\mathscr,\mathbfscr}]{Tex Gyre Pagella Math}

\begin{examples}
  
  \begin{enumarabic}
    \item There is a category $\mathsf{Set}$, where
      \begin{enumalph}
        \item $\ob{\mathsf{Set}}$ is the collection of all sets;
        \item $\mathsf{Set}(A,B)$ is the set of all functions from $A$ to $B$;
        \item composition is ordinary function composition;
        \item the identity on $A$ is the identity function on $A$.
      \end{enumalph}
    \item There is a category $\mathsf{Grp}$, where
      \begin{enumalph}
        \item $\ob{\mathsf{Grp}}$ is the collection of all groups;
        \item $\mathsf{Grp}(G,H)$ is the set of all group homomorphisms from $G$ to $H$;
        \item composition is ordinary function composition;
        \item the identity on $G$ is the identity homomorphism on $G$.
      \end{enumalph}
    \item There is a category $\mathsf{Top}$ of topological space and continuous maps.
    \item For each field $k$, there is a category $\mathsf{Vect}_k$ of vector spaces over $k$
      and linear maps between them.
  \end{enumarabic}
\end{examples}

\begin{definition}
  A map $f : A \to B$ in a category $\cat{A}$ is an \newterm{isomorphism} if there exists
  a map $g : B \to A$ such that $fg = \id_A$ and $gf = \id_B$.
  Ee call $g$ the \newterm{inverse} of $f$ and write $f^{-1} = g$,
  and say that $A$ and $B$ are \newterm{isomorphic}
  if there exists an isomorphism between them.

  \begin{examples}
    \begin{enumarabic}
      \item In $\mathsf{Set}$, isomorphisms are bijections.
      \item In $\mathsf{Grp}$ and $\mathsf{Ring}$, isomorphisms are
        group and ring isomorphisms respectively.
      \item In $\mathsf{Vect}_k$, isomorphisms are linear isomorphisms.
    \end{enumarabic}
  \end{examples}

\end{definition}

% restriction categories
\subsection{Restriction Categories}
\begin{definition}
  A \newterm{restriction category} is a category $\cat{A}$ with a
  \newterm{restriction} operation that assigns to each arrow $f : A \to B$
  an arrow $\bar{f} : A \to A$ such that:
  \begin{enumarabic}
    \item $\bar{f} \circ f = f$;
    \item $\bar{f} \circ \bar{g} = \bar{g} \circ \bar{f}$
      whenever $\dom{f} = \dom{g}$;
    \item $\overline{f \circ \bar{g}} = \bar{g} \circ \bar{f}$
      whenever $\dom{f} = \dom{g}$.
    \item $\bar{g} \circ f = \bar{g} \circ f \circ \bar{g}$
      whenever $\dom{f} = \range{g}$.
  \end{enumarabic}
\end{definition}

\begin{remark}
  It follows from the definition that $\bar{f}$ is \newterm{idempotent}.
  That is, $\bar{f} \circ \bar{f} = \bar{f}$.

  Furthermore, the operation $f \mapsto \bar{f}$ is also monotonic,
  with $\bar{\bar{f}} = \bar{f}$.
\end{remark}

\begin{examples}
  Here are a few examples of restriction categories.~\cite{CATEGORY-THEORY-TEXT}
  \begin{enumarabic}
    \item All categories admit the trivial restriction operation
      that maps $f : A \to B$ to $\bar{f} = \id_{A}$.
    \item The category $\mathsf{Par}$ of partial functions between sets
      admits a restriction operation that maps
      $f : A \parto B$ to $\bar{f} = \id_{\dom{f}}$.
  \end{enumarabic}
\end{examples}
